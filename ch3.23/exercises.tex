\documentclass[12pt,letterpaper]{article}
\usepackage[pdftex]{graphicx}
\usepackage{alltt}
\usepackage[margin=1in]{geometry}
\usepackage{amsmath, amsthm, amssymb}
\usepackage{verbatim}
\usepackage{ragged2e}
\usepackage{enumitem}
\usepackage{xfrac}
\setlist{parsep=0pt,listparindent=\parindent}
\setlength{\RaggedRightParindent}{\parindent}
\newcommand{\degree}{\ensuremath{^\circ}}
\newcommand{\n}{\break}
\let\oldemptyset\emptyset
\let\emptyset\varnothing
\newcommand{\Wlog}{without loss of generality}
\newcommand{\WLOG}{Without loss of generality}
\usepackage{accents}
\let\thinbar\bar
\newcommand\thickbar[1]{\accentset{\rule{.4em}{.8pt}}{#1}}
\let\bar\thickbar
\usepackage{standalone}
\usepackage{hyperref}
\newcommand{\R}{\ensuremath{\mathbb{R}}}
\usepackage{mathtools}
\DeclarePairedDelimiter{\ceil}{\lceil}{\rceil}
\DeclarePairedDelimiter{\floor}{\lfloor}{\rfloor}
\DeclarePairedDelimiter\abs{\lvert}{\rvert}
\DeclarePairedDelimiter\norm{\lVert}{\rVert}
%%%%%%%%%%%%%%%%%%%%%%%%%%%%%%%%%%%%%%%%%%%%%%%%%%%%%
%TOPOLOGY DOCUMENTS ONLY%
\newcommand{\T}{\ensuremath{\mathcal{T}}}
%%%%%%%%%%%%%%%%%%%%%%%%%%%%%%%%%%%%%%%%%%%%%%%%%%%%%

\begin{document}
\RaggedRight
\begin{enumerate}
  \item Let $\T$ and $\T'$ be two topologies on $X$. If $\T'\supset\T$, what does connectedness of $X$ in one topology imply about connectedness in the other?\\
  If $\T$ is connected, then the courser $\T'$ is clearly connected. The reverse relationship does not imply anything about connectivity, because a finer set can always be made by adjoining a separation of $X$ to $\T$.
  \item Let $\{A_n\}$ be a sequence of connected subspaces of $X$, such that $A_n\cap A_n+1 \neq \emptyset$ for all $n$. Show that $\bigcup A_n$ is connected. \\
  $A_1$ and $A_2$ are connected by \hyperref[thm:unionConnected]{Theorem 23.3}, since their intersection is not null. Likewise, the union of the sets $(A_!\cup A_2)$ and $A_3$ is connected, since $A_2\cap A_3$ is connected. Proceeding by induction, $\bigcup A_n$ is connected.
  \item Let $\{A_\alpha\}$ be a collection of connected subspaces of $X$; let $A$ be a connected subspace of $X$. Show that if $A\cap A_\alpha\neq\emptyset$ for all $\alpha$, then $A\cup(\bigcup(A_\alpha))$ is connected.\\
  For each $\alpha$, $A$ intersects $A_\alpha$, so $A\cup A_\alpha$ is connected. Similarly, for $\beta\neq\alpha$, $A\cup A_\alpha \cup A_\beta$ is connected, and so by induction we get that $A\cup(\bigcup A_\alpha)$ is connected.
  \item Show that if $X$ is an infinite set, it is connected in the \hyperref[dfn:finiteComplementTopology]{finite complement topology}. \\
  Suppose it is not connected, there exists a subset $U$ of $X$ which is both open and closed. However, for $X-U$ to be finite, $U$ must be an infinite set. Therefore, $X-U$ is not open, since $X-U$ is finite, so $U$ is not closed, a contradiction. $X$ is connected.
  \item A space is totally disconnected if its only connected subspaces are one-point sets. Show that if $X$ has the discrete topology, then $X$ is totally disconnected. Does the converse hold? \\
  Let $A$ be a connected subspace of $X$, suppose $A$ contains more than one distinct point, let $x$ be an element $A$. $\{x\}$ and $X-\{x\}$ are open sets of $X$, and when intersected with $A$ remain open sets. They are nonempty, and so form a separation of $X$, or when intersected with $A$, form a separation of $A$. A one point subset is obviously connected, and it is the only type of connected subspace of $X$.\\
  Conversly, suppose $X$ is totally disconnected. Let $A$ be a subset of $X$ that contains more than one point. By hypothesis, there exist disjoint nonempty sets $U$ and $V$ which are open in the subspace topology on $A$, and whose union is $A$. If $A$ is not open, then there is no need for $U$ or $V$ to be open in $X$. If they are not open, $X$ is not discrete. Still messy.\\
  \item Let $A\subset X$. Show that if $C$ is a connected subspace of $X$ that intersects both $A$ and $X-A$, then $C$ intersects $\text{Bd } A$. [\hyperref[dfn:boundary]{Boundary}].\\
  Consider the subsets of $C$, $U=A\cap C$ and $V=(X-A)\cap C$. $U$ and $V$ are clearly a pair of disjoint nonempty sets whose union is $C$. If $x$ is a limit point of both $A$ and $X-A$, then $x$ will be in the closure of each of these sets, and thus in the intersection of their closure, and so in the boundary of $A$, by the definition of boiundary. $C$ must contain at least one such $x$; it must be in one of $U$ or $V$, otherwise neither will contain a limit point of the other which would imply that $C$ is not connected - \hyperref[dfn:subspaceSeparation]{Lemma 23.1}. $C$ is connected by hypothesis, so it intersects the boundary of $A$.
  \item Is the space \hyperref[dfn:lowerLimitTopology]{$\R_\ell$} connected? \\
  It is not. The set $(-\infty,0)$ is open in $\R_\ell$, it is the union of basis elements $[-n,-n+1)$ for $n\in\mathbb{Z}_+$. $[0,\infty)$ is also open, constructed similarly. Together these form a separation of $\R_\ell$.
  \item Determine whether or not $\R^\omega$ is connected in the uniform topology. \\
  We can write $\R^\omega$ as the union of $U$, the set of all bounded sequences of real numbers, and $V$, the set of unbounded sequences of real numbers. These sets are disjoint. We show that each is open in the uniform topology.\\
  Suppose $x$ is a bounded sequence. Then $B_{\bar{p}}(x,\epsilon)$ for some $\epsilon<1$ is an open set containing $x$ which only contains bounded sequences, since each the $i$th coordinate of each element can be no farther than $\epsilon$ away from the $i$th coordinate of $x$. If the bound on $x$ is $M$, then a bound on each element of this $\epsilon$-ball is $M+\epsilon$. The union of such an $\epsilon$-ball around every bounded $x$ is the set of all bounded sequences of reals, which is therefore open. On the other hand, if $x$ is unbounded, then $B_{\bar{p}}(x,\epsilon)$ will be a set containing only unbounded sequences, and again the union of all such $\epsilon$-balls will give the set of all unbounded sequences. Therefore, $U$ and $V$ are open, and so form a separation of $\R^\omega$ in the uniform topology.
  \item Let $A$ be a proper subset of $X$, and let $B$ be a proper subset of $Y$. If $X$ and $Y$ are connected, show that $(X\times Y) - (A\times B)$ is connected. \\
  Intuitively, $X\times Y - A\times B$ is a plane with a hole in it, clearly connected. %We know by \hyperref[thm:finiteCartesianConnected]{Theorem 23.6} that $X\times Y$ is connected. Suppose $X-A$ and $Y-B$ are connnected, then by the same theorem, we are done. Suppose $X-A$ is not connected,
  Let $U_x=x\times Y$ for all $x\in X-A$, each $U_x$ is a subset of $X\times Y - A\times B$ and each $U_x$ is connected, since it is homeomorphic with $Y$. $V_y=X\times y$ for $y\in Y-B$ has similar properties. The union of all $U_x$ and $V_y$ is the entire set $X\times Y-A\times B$, and is a connected space, since each $V_y$ intersects all of the $U_x$s, and vice versa. 
  \item Let $\{X_\alpha\}_{\alpha\in J}$ be an indexed family of connected spaces; let $X$ be the product space $X=\prod_{\alpha\in J}X_\alpha$. Let $a=(a_\alpha)$ be a fixed point of $X$.
  \begin{enumerate}
    \item Given any finite subset $K$ of $J$, let $X_K$ denote the subspace of $X$ consisting of all points $x=(x_\alpha)$ such that $x_\alpha=a_\alpha$ for $a\not\in K$. Show that $X_K$ is connected. \\
    $X_K$ is homeomorphic to the finite cartesian product, $\prod_{\alpha\in K}X_\alpha$, just as the space $\bar{\R}^n$ is homeomorphic to $\R^n$ in example 7. Since finite cartesian products of connected spaces are connected, $X_K$ is connected.
    \item Show that the union $Y$ of spaces $X_K$ is connected. \\
    Each $X_k$ contains $a$, so the union of each connected $X_K$ is also connected, so $Y$ is connected.
    \item Show that $X$ equals the closure of $Y$; conclude that $X$ is connected.\\
    For each basis element $U=\prod U_\alpha$ of $X$, there is some $n$ for which $U_{\alpha>n}=X_\alpha$. The point $(x_1,\dots,x_n,a_{n+1},a_{n+2},\dots)$ belongs to both $Y$ and $X$, therefore each $U$ intersects $Y$, so the closure of $Y$ is $X$, so $X$ is connected.
  \end{enumerate}
  \item Let $p:X\rightarrow Y$ be a \hyperref[dfn:quotientMap]{quotient map}. Show that if each set $p^{-1}(\{y\})$ is connected, and $Y$ is connected, then $X$ is connected. \\
  Suppose $A,B$ is a separation of $X$. $p$ is surjective, so $p(A)\cup p(B) = Y$. By hypothesis, $Y$ is connected, so $p(A),p(B)$ are not a separation of $Y$. Then either $p(A)$ and $p(B)$ are not both open, or they are not disjoint. Suppose first that they are not disjoint, then there is a $y$ which is an element of both $p(A)$ and $p(B)$. However, $p^{-1}(y)$ is connected, therefore by \hyperref[thm:subspaceOfConnected]{Lemma 23.2} $p^{-1}(y)$ must lie entirely in $A$ or $B$. This is a contradiction, since if that were true $p(A)$ and $p(B)$ could not both contain $y$.\\
  Now, $p(A)$ and $p(B)$ must be disjoint. If they are not a separation of $Y$, one of them must not be open, suppose without loss of generality that $p(A)$ is not open. We have shown in the previous part that if $y\in p(A)$, then $p^{-1}(y)\in A$. Therefore, $A$ is saturated with respect to $p$. By hypothesis, $A,B$ are a separation of $X$, so they are open, so $A$ is a saturated open set, therefore $p(A)$ must be open. Similarly, $p(B)$ must be open. Thus, if $A,B$ is a separation of $X$, it must be a separation of $Y$, a contradiction, since $Y$ is connected by hypothesis. Therefore, $X$ is connected.
  \item Let $Y\subset X$; let $X$ and $Y$ be connected. Show that if $A$ and $B$ form a separation of $X-Y$, then $Y\cup A$ and $Y\cup B$ are connected.\\
  Suppose that $Y\cup A$ is not connected, it has a separation $C,D$. $Y$ is connected, so if $C$ intersects $Y$, it must be entirely in $Y$, likewise with $D$. The union of $C$ and $D$ must be $Y\cup A$, and neither $C$ nor $D$ can be empty, so exactly one of $C,D$ must be exactly $Y$, and the other must be exactly $A-Y$. Without loss of generality, suppose that $C=Y$ and $D=A-Y$. $A$ and $B$ form a separation of $X-Y$, each is a subset of $X-Y$, so neither intersects $Y$. Therefore, $D=A-Y=A$. Thus, if $Y\cup A$ is not connected, then its separation is $Y,A$. Similarly, if $Y\cup B$ is not connected, its separation is $Y,B$. \\
  Furthermore, $A$ is open in both $X-Y$ and $Y\cup A$. There exists an open set $A'\subset X$ such that $A' \cap (Y\cup A) = A$, so that $A$ will be open in the subspace $Y\cup A$. Therefore, $A'=A$, so $A$ is open in $X$. Likewise, $B$ and $Y$ are open in $X$. Therefore, $A\cup B$ and $Y$ form a separation of $X$, a contradiction.
  %$A\cup B = X - Y$, so $A\cup B\cup Y = X$. $A$ and $B$ are disjoint, so $A = X - Y - B$, and $A\cup Y = X - B$.
\end{enumerate}
\end{document}