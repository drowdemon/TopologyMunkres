\documentclass[12pt,letterpaper]{article}
\usepackage[pdftex]{graphicx}
\usepackage{alltt}
\usepackage[margin=1in]{geometry}
\usepackage{amsmath, amsthm, amssymb}
\usepackage{verbatim}
\usepackage{ragged2e}
\usepackage{enumitem}
\usepackage{xfrac}
\setlist{parsep=0pt,listparindent=\parindent}
\setlength{\RaggedRightParindent}{\parindent}
\newcommand{\degree}{\ensuremath{^\circ}}
\newcommand{\n}{\break}
\let\oldemptyset\emptyset
\let\emptyset\varnothing
\newcommand{\Wlog}{without loss of generality}
\newcommand{\WLOG}{Without loss of generality}
\usepackage{accents}
\let\thinbar\bar
\newcommand\thickbar[1]{\accentset{\rule{.4em}{.8pt}}{#1}}
\let\bar\thickbar

%%%%%%%%%%%%%%%%%%%%%%%%%%%%%%%%%%%%%%%%%%%%%%%%%%%%%
%THIS DOCUMENT ONLY%
\newcommand{\T}{\ensuremath{\mathcal{T}}}
%%%%%%%%%%%%%%%%%%%%%%%%%%%%%%%%%%%%%%%%%%%%%%%%%%%%%

\begin{document}
\RaggedRight
\begin{enumerate}
  \item Let $\mathcal{C}$ be a collection of subsets of the set $X$. Suppose that $\emptyset$ and $X$ are in $\mathcal{C}$, and that finite unions and arbitrary intersections of elements of $\mathcal{C}$ are in $\mathcal{C}$. Show that the collection $$\T = \{X-C\; |\; C\in\mathcal{C}\}$$ is a topology on $X$.\n
  \indent See Theorem 17.1, proof proceeds similarly, just looking at the opposite side of the equations.
  \item Show that if $A$ is closed in $Y$ and $Y$ is closed in $X$, then $A$ is closed in $X$.\n
  \indent Since $A$ is closed in $Y$, by theorem 17.2, there exists a set $D$ closed in $X$ such that $A=Y\cap D$.
  $Y$ and $D$ are both closed in $X$, and by Theorem 17.1 intersections of closed sets are closed, so $Y\cap D$ is closed, hence $A$ is closed in $X$.
  \item Show that if $A$ is closed in $X$ and $B$ is closed in $Y$, then $A\times B$ is closed in $X\times Y$.\n
  \indent Consider the set $\pi_1^{-1}(X-A)$. This is equivalent to the open set $(X-A)\times Y$, which is open since both $X-A$ and $Y$ are open. Similarly, $\pi_2^{-1}(Y-B)$ is open. The union of these is an open set, since the union of any two open sets is open.
  $$((X-A)\times Y)\cup (X\times (Y-B))$$
  which is precisely the complement of $A\times B$.
  \item Show that if $U$ is open in $X$ and $A$ is closed in $X$, then $U-A$ is open in $X$, and $A-U$ is closed in $X$.\n
  \indent $U,A\subset X$, therefore $U-A = U\cap(X-A)$. $U$ and $X-A$ are both open in $X$, so their intersection is open, so $U-A$ is open.\hspace{5in}\n
  Similarly, $A-U=A\cap (X-U)$. $X-U$ and $A$ are both closed sets, thus their intersection is closed, so $A-U$ is closed. \hspace{5in}
  \item Let $X$ be an ordered set in the order topology. Show that $\overline{(a,b)} \subset [a,b]$. Under what conditions does equality hold? \n
  \indent If $X$ has a minimum or a maximum, denote the minimum by $m$ and the maximum by $n$. Then $X-(a,b) = [m,a) \cup (b, n]$. Otherwise, $X-(a,b)=(-\infty,a]\cup[b,\infty)$. Generally, neither of these sets are open, as an open set in the order topology is an open interval, except at the endpoints. Thus the closure includes $a$ and $b$. The only case where it does not is when $a$ has an immediate successor and $b$ has an immediate predecessor. Then the set $[m,a]$ can be written as $[m,a+1)$, likewise with $b$. The union of these is open, thus in this case $(a,b)$ is it's own closure. Therefore, $\overline{(a,b)}=[a,b]$ if $a$ has no immediate successor and $b$ has no immediate predecessor.\n
  \item Let $A, B$, and $A_a$ denote subsets of a space $X$. Prove the following: 
  \begin{enumerate}
    \item If $A\subset B$, then $\bar{A}\subset\bar{B}$. 
    \item $\overline{A\cup B} = \bar{A}\cup \bar{B}$.
    \item $\overline{\bigcup A_a} \supset \bigcup \bar{A}_a$; give an example where equality fails.
  \end{enumerate}
  Solutions:
  \begin{enumerate}
    \item Proceeding by contradiction, $\bar{A}$ contains at least one element $x$ which is not an element of $\bar{B}$, implying by theorem 17.5 that every open set containing $x$ interesects $A$. Since $A\subset B$, then every open set containing $x$ intersects $B$, implying that $x\in B$, a contradiction.
    \item Consider $x\subset \bar{A}\cup\bar{B}$. Every open set that contains $x$ intersects $A$ or $B$, equivalently it intersects either $A\cup B$. Thus, it is in $\overline{A\cup B}$. Therefore, $\bar{A}\cup\bar{B} \subset \overline{A\cup B}$.\n
    \indent Consider the other direction. Let $C=A\cup B$. Then $\bar{C}\subset\bar{A}\cup\bar{B}$, so $C\cup C' \subset A \cup B \cup A' \cup B'$, implying that $C' \subset A' \cup B'$ is equivalent to what is to be proven. By contradiction, suppose $x$ is an element of $C'$ but neither $A'$ nor $B'$. Presume $x$ is in neither $A$ nor $B$, as that is a trivial case. Then there exists a neighborhood $U$ of $x$ in $X$ that intersects $A$ but not $B$, and a neighborhood $V$ of $X$ that intersects $B$ but not $A$. Consider the intersection of these neighboorhoods, $U\cap V$. This is an open set containing $X$ an no elements of either $A$ or $B$, thus it does not intersect $A\cup B$, so $x$ is not in $\bar{C}$, so $x$ is not in $C'$, a contradiction. Thus $\overline{A\cup B}\subset \bar{A}\cup\bar{B}$, implying that $\overline{A\cup B}=\bar{A}\cup\bar{B}$.
    \item If $x$ is in $\bigcup \bar{A}_a$, then every open set that contains $x$ intersects some element $A_a$, thus it intersects $\bigcup A_a$, thus it is in $\overline{\bigcup A_a}$, thus $\overline{\bigcup A_a} \supset\bigcup\bar{A}_a$. For a case where equality fails, consider the case when $\bigcup A_a$ is a union of infinitely many sets. Then there can exist an $x$ such that all of its neighborhoods intersect $\bigcup A_a$, but not all of its neighborhoods intersect any particular $A_a$. For example, $A_n=\{1/n\}$: $\overline{\bigcup A_a}$ contains 0.
  \end{enumerate}
  \item Criticize the following ``proof'' that $\overline{\bigcup A_a} \subset \bigcup \bar{A}_a$: if $\{A_a\}$ is a collection of sets in $X$ and if $x\in\overline{\bigcup A_a}$, then every neighborhood $U$ of $x$ intersects $\bigcup A_a$. Thus $U$ must intersect some $A_a$, so that $x$ must belong to the closure of some $A_a$. Therefore, $x\in\bigcup\bar{A_a}$.\n
  \indent $U$ must intersect \emph{some} $A_a$, but not every $U$ intersects the same $A_a$. 
  \item Let $A, B$, and $A_a$ denote subsets of a space $X$. Determine whether the following equations hold; if an equality fails, determine whether one of the inclusions $\supset$ or $\subset$ holds.
  \begin{enumerate}
    \item $\overline{A\cap B} = \bar{A}\cap\bar{B}$.\hspace{5in}\n
    \indent Consider an element $x$ of $\overline{A \cap B}$. Every neighborhood of $x$ intersects $A\cap B$, thus intersecting both $A$ and $B$, thus $x$ is in both $\bar{A}$ and $\bar{B}$. Therefore, $\overline{A\cap B}\subset \bar{A}\cap\bar{B}$. In the other direction, consider $x$ in $\bar{A}\cap\bar{B}$. Specifically, consider an $x$ that is not in $A$ or $B$, but is in both $\bar{A}$ and $\bar{B}$. Such an element might not be in $\overline{A\cap B}$. For example, consider $A=\mathbb{R}_+$ and $B=\mathbb{R}_-$. Both closures contain $0$, but the intersection of these sets is empty. Thus equality does not hold, only $\overline{A\cap B}\subset \bar{A}\cap{B}$.
    \item $\overline{\bigcap A_a} = \bigcap\bar{A}_a$.\hspace{5in}\n
    \indent The same argument as above applies. $\overline{\bigcap A_a}\subset\bigcap\bar{A}_a$.
    \item $\overline{A-B}=\bar{A}-\bar{B}.$ \hspace{5in}\n
    \indent Consider an $x$ that lies in $\bar{A}$, $\bar{B}$, and $\overline{A-B}$. This $x$ won't exist in $\bar{A}-\bar{B}$. Once again, the example of $A=\mathbb{R}_+$, $B=\mathbb{R}_-$ illustrates this. The difference of the closures is just $\mathbb{R}_+$, while the closure of the difference is $\bar{\mathbb{R}}_+=\mathbb{R}_+ + \{0\}$. Thus at best $\overline{A-B}\supset\bar{A}-\bar{B}$. To prove this direction, consider an $x$ in $\bar{A}-\bar{B}$. All neighborhoods of $x$ intersect $A$, but there exists some neighborhood $U$ that does not intersect $B$. Suppose there is a neighborhood $V$ that does not intersect $A-B$. Then $U\cap V$ must intersect $A$, since it's a neighborhood of $x$. Since it does not intersect $A-B$, the only way it could intersect $A$ is if it intersected $B$, i.e. if it intersected $A\cap B$. But $U$ does not intersect $B$, a contradiction. Thus we conclude that $\overline{A-B}\supset\bar{A}-\bar{B}$.
  \end{enumerate}
  \item Let $A\subset X$ and $B\subset Y$. Show that in the space $X\times Y$, $\overline{A\times B}=\bar{A}\times\bar{B}$.\n
  \indent Consider an $x_0\times y_0\in \bar{A} \times \bar{B}$. Every neighborhood of $x_0$ must intersect $A$, and every neighborhood of $y_0$ must intersect $B$. Identically, for an element $x_1\times y_1\in\overline{A\times B}$, every neighborhood of $x_1$ intersects $A$ and every neighborhood of $y_1$ intersects $B$. 
  \item Show that every order topology is Hausdorff.\n
  \indent Consider the pair of distinct points $a,b$ in a topological space $X$. \WLOG, assume $a<b$. Then either there exists a point $c$ such that $a<c<b$, or $b$ is the immediate successor of $a$. In the first case, the neighborhoods $(-\infty, c)$ and $(c,\infty)$ are disjoint neighborhoods of $a$ and $b$ respectively, so the space is Hausdorff. In the second case, $(-\infty, b)$ and $(a,\infty)$ are the disjoint neighborhoods. 
  \item Show that the product of two Hausdorff spaces $X$ and $Y$ is Hausdorff.\n
  \indent For every $x_1,x_2$ in $X$ there exist disjoint neighborhoods $U_1$ and $U_2$. Similarly, for every $y_1,y_2$ in $Y$ there exist disjoint neighborhoods $V_1$ and $V_2$. Thus in the product, for every $x_1\times y_1$ and $x_2\times y_2$ in $X\times Y$, there exist disjoint neighborhoods $U_1\times V_1$ and $U_2\times V_2$.
  \item Show that the subspace $A$ of a Hausdorff space $X$ is Hausdorff.\n
  \indent For every $x_1,x_2$ in $X$ there exist disjoint neighborhoods $U,V$. For such elements, the neighborhoods $U\cap A, V\cap A$ will be disjoint in the subspace topology.
  \item Show that $X$ is Hausdorff if and only if the diagonal $\Delta = \{x\times x\; |\; x\in X\}$ is closed in $X\times X$.\hspace{5in}\n
  \indent If $X$ is Hausdorff, for every pair of distinct points $x_1,x_2\in X$, they have disjoint neighborhoods $U,V$. Clearly $x_1 \times x_2$ is in the open set $U\times V$. Furthermore, $U\cap V=\emptyset$ implies that $U\times V \cap \Delta=\emptyset$, since if the latter intersection was not empty, that would imply there was an element $c\times c$ in $U\times V$, i.e. $c\in U$ and $c\in V$, which contradicts the previous assertion. Since every element of $X\times X - \Delta$ lies in an open set, $X\times X - \Delta$ is open, so $\Delta$ is closed.\n
  \indent In the other direction proceeds similarly: if the diagonal is closed, then it's complement is open, i.e. every pair of distinct points $x_1,x_2$ lies in an open set $U\times V$. $x_1\in U$ and $x_2\in V$. If $U\cap V\neq \emptyset$, then there exists an element $c$ such that $c\in U$ and $c\in V$, so $c\times c\in U\times V$, but $c\times c\in \Delta$, a contradiction as $U$ and $V$ are subsets of the complement of $\Delta$. Thus $U,V$ are disjoint, so $X\times X$ is Hausdorff.
  \item In the finite complement topology on $\mathbb{R}$ to what point or points does the sequence $x_n=1/n$ converge?\hspace{5in}\n
  \indent The finite complement topology on $\mathbb{R}$ is defined such that all finite sets are closed. Consider a finite set $A\in\mathbb{R}$. For any $N$, the set of $x_{n>N}$ is infinite, so there must be a smallest $x_n$ that $A$ contains (if it contains any of the $x_n$s), and $A$ will not contain any of the $x_n$ below that point. Therefore the complement of $A$ will be an open set that contains all $x_{n>N}$. This describes every open set, so every this sequence converges to every real.
  \item Show that the $T_1$ axiom is equivalent to the condition that for each pair of points of $X$, each has a neighborhood not containing the other. ($T_1$ axiom: all finite point sets are closed). \n
  \indent Suppose that $X$ is a space satisfying the $T_1$ axiom. Then, for any pair of points $a,b$, the sets $\{a\}$ and $\{b\}$ are closed. The complement of $\{a\}$ is a neighborhood of $a$ that does not contain $b$, and the complement of $\{b\}$ is a neighborhood of $b$ that does not contain $a$.\n
  \indent In the other direction, suppose that each pair of points $a,b$ has a neighborhood, $U,V$ respectively, that does not contain the other point. The corresponding closed sets are $\mathbb{R}-U\subset \mathbb{R}-\{a\}$, with the same format for $V$. Now, for every $c$ in that closed set except $c=b$, take the closed set $C$ that is the complement of the open set that does not contain $b$. The intersection of each such $C$ with $U$ will yield a closed set $\{b\}$. Thus every one point set is closed, and since finite unions are allowed, every finite set is closed.
  \item Consider the following five topologies on $\mathbb{R}$: $\T_1$ - the standard topology, $\T_2$ - the topology of $\mathbb{R}_K$, $\T_3$ - the finite complement topology, $\T_4$ - the upper limit topology, and $\T_5$ - the topology having all sets $(-\infty,a)$ as a basis. For each one, 
  \begin{enumerate}
    \item Determine the closure of the set $K=\{1/n\;|\;n\in\mathbb{Z}_+\}$
    \item Is it Hausdorff? Does it satisfy the $T_1$ axiom?
  \end{enumerate}
  \begin{enumerate}[label=\roman*)]
    \item $\T_1$ - standard topology
    \begin{enumerate}[label=(\alph*)]
      \item $\bar{K}=K\cup \{0\}$. Every neighborhood of 0 intersects $K$ at a point other than 0, and the next part shows that there is only one limit point of the sequence $x_n=1/n$.
      \item The standard topology is Hausdorff. For any $a$ and $b$, $a<b$ \Wlog . Then, the intervals $(a-1, (a+b)/2)$ and $((a+b)/2, b+1)$ are disjoint open neighborhoods of $a,b$ respectively.
    \end{enumerate}
    \item $\T_2$ - $\mathbb{R}_K$
    \begin{enumerate}[label=(\alph*)]
      \item $K$ is its own closure. 0 is not a limit point, since the set $(-1, 1) - K$ is a neighborhood of 0 that does not intersect $K$.
      \item This is finer than the standard topology, $\T_1$ is Hausdorff, so this is too.
    \end{enumerate}
    \item $\T_3$ - the finite complement topology
    \begin{enumerate}[label=(\alph*)]
      \item all reals, by exercise 14.
      \item $T_1$, in text. Also, by previous part obviously not Hausdorff, and $T_1$ by definition.
    \end{enumerate}
    \item $\T_4$ - the upper limit topology
    \begin{enumerate}[label=(\alph*)]
      \item The set is its own closure. 0 is not a limit point because the interval $(-1,0]$ is one of its neighborhoods.
      \item Finer than $\T_1$, which is Hausdorff, so this is Hausdorff.
    \end{enumerate}
    \item $\T_5$ - the topology having all sets $(-\infty, a)$ as a basis
    \begin{enumerate}[label=(\alph*)]
      \item $\bar{K}=[0,\infty)$, since for all points $x$ in that interval, all of their open neighborhoods must include an element larger than $x$, and all elements smaller than $x$. 
      \item Given points $a,b$ suppose, \Wlog , that $a<b$. All neighborhoods of $b$ include everything less than $b$, including $a$. This space does not follow the $\T_1$ axiom, because the set $\{a\}$ is not closed for any $a$. 
    \end{enumerate}
  \end{enumerate}
  \item Consider the lower limit topology on $\mathbb{R}$ and the topology generated by the basis $\mathcal{C}=\{[a,b)\;|\; a<b,\; a,b \text{ are rational}\}$. Determine the closures of the intervals $A=(0,\sqrt{2})$ and $B=(\sqrt{2},3)$ in these topologies.
  \begin{enumerate}
    \item Lower limit topology
    \begin{enumerate}
      \item $\bar{A} = A\cup \{0\}$. The neighborhood $\left[\sqrt{2},2\right)$ of $\sqrt{2}$ does not lie in $A$.
      \item $\bar{B} = B\cup \{\sqrt{2}\}$.
    \end{enumerate}
    \item The topology generated by $\mathcal{C}$.
    \begin{enumerate}
      \item $\bar{A} = A\cup \{0,\sqrt{2}\}$, since $\left[\sqrt{2}, b\right)$ is not an open set.
      \item $\bar{B} = B\cup \{\sqrt{2}\}$.
    \end{enumerate}
  \end{enumerate}
  \item Determine the closures of the following subsets of the ordered square $I^2_o = [0,1]\times [0,1]$
  \begin{enumerate}
    \item $A=\{(1/n)\times 0 \;|\; n\in\mathbb{Z}_+\}$\hspace{5in}\n
    \indent $A\cup \{0\times 0\}$
    \item $B=\{(1-1/n)\times \sfrac{1}{2}\;|\;n\in\mathbb{Z}_+\}$\hspace{5in}\n
    \indent $B\cup \{1\times \sfrac{1}{2}\}$
    \item $C=\{x\times 0 \;|\; 0<x<1\}$\hspace{5in}\n
    \indent $[0,1]\times 0$
    \item $D=\{x\times \sfrac{1}{2} \;|\; 0<x<1\}$\hspace{5in}\n
    \indent $[0,1]\times \sfrac{1}{2}$
    \item $E=\{\sfrac{1}{2} \times y \;|\; 0<y<1\}$\hspace{5in}\n
    \indent $[\sfrac{1}{2}\times 0, \sfrac{1}{2}\times 1]$
  \end{enumerate}
  \item If $A\subset X$ we define the boundary of $A$ by the equation $\text{Bd } A = \bar{A} \cap \overline{(X-A)}$
  \begin{enumerate}
    \item Show that Int $A$ and Bd $A$ are disjoint, and $\bar{A}=\text{Int } A\ \cup\ \text{Bd } A$.\hspace{5in}\n
    \indent Suppose there exists an element $x$ that is in both Int $A$ and Bd $A$. Then $x$ lies in an open set $U$ of $X$ that is a subset of $A$.
    $x$ must also lie within $\overline{(X-A)}$. $x$ is an element of $A$, so it cannot be in $X-A$, so it must be a limit point of $X-A$, i.e. every neighborhood of $x$ intersects $X-A$ at a point other than itself. However, $U$ is a subset of $A$, and thus is a neighborhood of $x$ that does not intersect $X-A$, a contradition. Thus the two are disjoint.\n
    \indent Bd $A = \bar{A} \cap \overline{(X-A)}$, therefore $\bar{A} = (X - \overline{(X-A)})\cup \text{Bd } A$. $X - \overline{(X-A)} = (X - (X - A)) \cap (X - (X-A)') = A \cap (X-(X-A)') = (A\cap X) - (A\cap (X-A)') = A - (A\cap(X-A)') = A - (X-A)'$\hspace{3in}\n
    This is equivalent to Int $A$. Obviously no points outside $A$ are in Int $A$. Now, suppose that there exists an $x$ in Int $A$ that is also a limit point of $X-A$, i.e. every neighborhood of $x$ intersects $X-A$ at a point other than $x$, and $x$ lies in an open set of $A$. The open set of $A$ that $x$ must lie in to be in its interior does not intersect $X-A$. Thus, Int $A = A - (X-A)'$ and $\bar{A} = \text{Int } A \cup \text{Bd } A$
    \item Show that Bd $A = \emptyset \leftrightarrow A$ is both open and closed.\hspace{5in}\n
    \indent Bd $A = \bar{A} \cap \overline{(X-A)} = (A \cup A') \cap ((X-A)\cup (X-A)') = (A\cap (X-A)') \cup (A'\cap (X-A)) \cup (A' \cap (X-A)')$ For any $U$, $U$ and $U'$ are disjoint, so the previous expression is equal to $(X-A)' \cup A' \cup (A'\cap (X-A)') = (X-A)' \cup A' = \text{Bd } A$.\n
    If $(X-A)'\cup A'=\emptyset$, then $A'=\emptyset$, implying that the set is closed. Furthermore, this implies that $\bar{A}=A$. In the previous part, we found $\bar{A}=\text{Int } A \cup \text{Bd } A$, so Int $A=A$, implying that $A$ is open. Thus $A$ is open and closed.\n
    \indent In the other direction, suppose $A$ is both open and closed. Then Int $A = A = \bar{A}$. Since $\bar{A} = $ Int $A\ \cup\ $Bd $A$, and since Int $A$ and Bd $A$ are disjoint, Bd $A$ must be empty.
    \item Show that $U$ is open $\leftrightarrow$ Bd $U = \bar{U} - U$.\hspace{5in}\n
    \indent $\bar{U} - U = U'$ and Bd $U = U' \cup (X-U)'$. So the hypothesis is equivalent to $U$ is open $\leftrightarrow (X-U)'\subset U'$. These two sets are disjoint (suppose there exists an $x$ in both. $x$ is not in $X-U$ and $x$ is not in $U$, a contradiction.) So the hypothesis is $U$ is open $\leftrightarrow (X-U)'=\emptyset$.\hspace{5in}\n
    If $U$ is open, then $(X-U)$ is closed, so it is its own closure, so $(X-A)'=\emptyset$.
    In the other direction, if $(X-U)'=\emptyset$, then $(X-U)$ is it's own closure, so it is closed, so $U$ is open.
    \item If $U$ is open, is it true that $U = \text{Int } \bar{U}$?\hspace{5in}\n
    \indent No. Consider $\mathbb{R} - \{0\}$. It's an open set in the standard topology, but the interior of the closure is the interior of $\mathbb{R}$ is just $\mathbb{R}\neq \mathbb{R}-\{0\}$.
  \end{enumerate}
  \item Find the boundary and the interior of each of the following subsets of $\mathbb{R}^2$
  \begin{enumerate}
    \item $A=\{x\times 0\}$\hspace{5in}\n
    \indent Bd $A$ = $A$. Int $A = \emptyset$
    \item $B=\{x\times y\;|\; x>0 \wedge y\neq 0\}$\hspace{5in}\n
    \indent Bd $B = \{x\times 0 \cup 0\times y \cup 0\times 0\}$. Int $B = B$.
    \item $C=A\cup B$\hspace{5in}\n
    \indent Bd $C = \{x\times 0\;|\; x<0 \cup 0\times y \}$. Int $C = \{x\times y\;|\; x>0\}$.
    \item $D=\{x\times y\;|\;x\in\mathbb{Q}\}$\hspace{5in}\n
    \indent Bd $D = \{x\times y\}$. Int $D = \emptyset$
    \item $E=\{x\times y\;|\;0<x^2-y^2\leq 1\}$\hspace{5in}\n
    \indent Bd $E = \{x\times y\;|\; x^2=y^2 \vee x^2=1+y^2$. Int $E=\{x\times y\;|\; 0<x^2-y^2<1\}$.
    \item $F=\{x\times y\;|\;x\neq 0 \wedge y\leq \frac{1}{x}\}$\hspace{5in}\n
    \indent Bd $F=\{x\times y\;|\; x=0 \vee y=\frac{1}{x}\}$. Int $F=\{x\times y\;|\; x\neq 0 \wedge y<\frac{1}{x}\}$.
  \end{enumerate}
  \item (Kuratowski) Consider the collection of all subsets $A$ of the topological space $X$. The operations of closure and complementation are functions from this collection to itself.
  \begin{enumerate}
    \item Show that starting from a given set $A$, one can form no more than 14 distinct sets by applying these operations successively.\n
    \indent Taking either operation twice on the same set obviously yields the original set.\n
    \indent The sets $A$, $X-A$, $\bar{A}$, $\overline{X-A}$ and $X-\bar{A}$ can obviously be distinct. For example the with the set $(0,1]$, these five resultant sets are distinct. Furthermore, applying these operations to one of the first three sets does not give a distinct sets, nor does the closure of the fourth, nor does the complement of the fifth.
    So, currently we need to examine $X-\overline{X-A}$ and $\overline{X-\bar{A}}$. The former, with the same example, yields a new set. The latter is identical to $\overline{X-A}$ in this case. \n
    So far, the sets we have, in order, with the running example are $(0,1]; (-\infty,0]\cup (1,\infty); [0,1]; (-\infty,0]\cup [1,\infty); (-\infty,0)\cup(1,\infty); (0,1)$.
    We now try $\overline{X-\overline{X-A}}$. This is identical to $\bar{A}$ in this case.
    %(2,3)\cup (3,4) is interesting, since the complement has a finite closed set {3}. Maybe that'll come to something. 
    \item Find a subset $A$ of $\mathbb{R}$ (in its usual topology) for which the maximum of 14 is reached.
  \end{enumerate}
\end{enumerate}
\end{document}
