\documentclass[12pt,letterpaper]{article}
\usepackage[pdftex]{graphicx}
\usepackage{alltt}
\usepackage[margin=1in]{geometry}
\usepackage{amsmath, amsthm, amssymb}
\usepackage{verbatim}
\usepackage{ragged2e}
\usepackage{enumitem}
\usepackage{xfrac}
\setlist{parsep=0pt,listparindent=\parindent}
\setlength{\RaggedRightParindent}{\parindent}
\newcommand{\degree}{\ensuremath{^\circ}}
\newcommand{\n}{\break}
\let\oldemptyset\emptyset
\let\emptyset\varnothing
\newcommand{\Wlog}{without loss of generality}
\newcommand{\WLOG}{Without loss of generality}
\usepackage{accents}
\let\thinbar\bar
\newcommand\thickbar[1]{\accentset{\rule{.4em}{.8pt}}{#1}}
\let\bar\thickbar

%%%%%%%%%%%%%%%%%%%%%%%%%%%%%%%%%%%%%%%%%%%%%%%%%%%%%
%THIS DOCUMENT ONLY%
\newcommand{\T}{\ensuremath{\mathcal{T}}}
%%%%%%%%%%%%%%%%%%%%%%%%%%%%%%%%%%%%%%%%%%%%%%%%%%%%%

\begin{document}
\RaggedRight
\begin{enumerate}
  \item Let $\mathcal{C}$ be a collection of subsets of the set $X$. Suppose that $\emptyset$ and $X$ are in $\mathcal{C}$, and that finite unions and arbitrary intersections of elements of $\mathcal{C}$ are in $\mathcal{C}$. Show that the collection $$\T = \{X-C\; |\; C\in\mathcal{C}\}$$ is a topology on $X$.\n
  \indent See Theorem 17.1, proof proceeds similarly, just looking at the opposite side of the equations.
  \item Show that if $A$ is closed in $Y$ and $Y$ is closed in $X$, then $A$ is closed in $X$.\n
  \indent Since $A$ is closed in $Y$, by theorem 17.2, there exists a set $D$ closed in $X$ such that $A=Y\cap D$.
  $Y$ and $D$ are both closed in $X$, and by Theorem 17.1 intersections of closed sets are closed, so $Y\cap D$ is closed, hence $A$ is closed in $X$.
  \item Show that if $A$ is closed in $X$ and $B$ is closed in $Y$, then $A\times B$ is closed in $X\times Y$.\n
  \indent Consider the set $\pi_1^{-1}(X-A)$. This is equivalent to the open set $(X-A)\times Y$, which is open since both $X-A$ and $Y$ are open. Similarly, $\pi_2^{-1}(Y-B)$ is open. The union of these is an open set, since the union of any two open sets is open.
  $$((X-A)\times Y)\cup (X\times (Y-B))$$
  which is precisely the complement of $A\times B$.
  \item Show that if $U$ is open in $X$ and $A$ is closed in $X$, then $U-A$ is open in $X$, and $A-U$ is closed in $X$.\n
  \indent $U,A\subset X$, therefore $U-A = U\cap(X-A)$. $U$ and $X-A$ are both open in $X$, so their intersection is open, so $U-A$ is open.\hspace{5in}\n
  Similarly, $A-U=A\cap (X-U)$. $X-U$ and $A$ are both closed sets, thus their intersection is closed, so $A-U$ is closed. \hspace{5in}
  \item Let $X$ be an ordered set in the order topology. Show that $\overline{(a,b)} \subset [a,b]$. Under what conditions does equality hold? \n
  \indent If $X$ has a minimum or a maximum, denote the minimum by $m$ and the maximum by $n$. Then $X-(a,b) = [m,a) \cup (b, n]$. Otherwise, $X-(a,b)=(-\infty,a]\cup[b,\infty)$. Generally, neither of these sets are open, as an open set in the order topology is an open interval, except at the endpoints. Thus the closure includes $a$ and $b$. The only case where it does not is when $a$ has an immediate successor and $b$ has an immediate predecessor. Then the set $[m,a]$ can be written as $[m,a+1)$, likewise with $b$. The union of these is open, thus in this case $(a,b)$ is it's own closure. Therefore, $\overline{(a,b)}=[a,b]$ if $a$ has no immediate successor and $b$ has no immediate predecessor.\n
  \item Let $A, B$, and $A_a$ denote subsets of a space $X$. Prove the following: 
    \begin{enumerate}
      \item If $A\subset B$, then $\bar{A}\subset\bar{B}$. 
      \item $\overline{A\cup B} = \bar{A}\cup \bar{B}$.
      \item $\overline{\bigcup A_a} \supset \bigcup \bar{A}_a$; give an example where equality fails.
    \end{enumerate}
    Solutions:
    \begin{enumerate}
    \item Proceeding by contradiction, $\bar{A}$ contains at least one element $x$ which is not an element of $\bar{B}$, implying by theorem 17.5 that every open set containing $x$ interesects $A$. Since $A\subset B$, then every open set containing $x$ intersects $B$, implying that $x\in B$, a contradiction.
    \item Consider $x\subset \bar{A}\cup\bar{B}$. Every open set that contains $x$ intersects $A$ or $B$, equivalently it intersects either $A\cup B$. Thus, it is in $\overline{A\cup B}$. Therefore, $\bar{A}\cup\bar{B} \subset \overline{A\cup B}$.\n
      \indent Consider the other direction. Let $C=A\cup B$. Then $\bar{C}\subset\bar{A}\cup\bar{B}$, so $C\cup C' \subset A \cup B \cup A' \cup B'$, implying that $C' \subset A' \cup B'$ is equivalent to what is to be proven. By contradiction, suppose $x$ is an element of $C'$ but neither $A'$ nor $B'$. Presume $x$ is in neither $A$ nor $B$, as that is a trivial case. Then there exists a neighborhood $U$ of $x$ in $X$ that intersects $A$ but not $B$, and a neighborhood $V$ of $X$ that intersects $B$ but not $A$. Consider the intersection of these neighboorhoods, $U\cap V$. This is an open set containing $X$ an no elements of either $A$ or $B$, thus it does not intersect $A\cup B$, so $x$ is not in $\bar{C}$, so $x$ is not in $C'$, a contradiction. Thus $\overline{A\cup B}\subset \bar{A}\cup\bar{B}$, implying that $\overline{A\cup B}=\bar{A}\cup\bar{B}$.
    \item If $x$ is in $\bigcup \bar{A}_a$, then every open set that contains $x$ intersects some element $A_a$, thus it intersects $\bigcup A_a$, thus it is in $\overline{\bigcup A_a}$, thus $\overline{\bigcup A_a} \supset\bigcup\bar{A}_a$. For a case where equality fails, consider the case when $\bigcup A_a$ is a union of infinitely many sets. Then there can exist an $x$ such that all of its neighborhoods intersect $\bigcup A_a$, but not all of its neighborhoods intersect any particular $A_a$. For example, $A_n=\{1/n\}$: $\overline{\bigcup A_a}$ contains 0.
    \end{enumerate}
  \item Criticize the following ``proof'' that $\overline{\bigcup A_a} \subset \bigcup \bar{A}_a$: if $\{A_a\}$ is a collection of sets in $X$ and if $x\in\overline{\bigcup A_a}$, then every neighborhood $U$ of $x$ intersects $\bigcup A_a$. Thus $U$ must intersect some $A_a$, so that $x$ must belong to the closure of some $A_a$. Therefore, $x\in\bigcup\bar{A_a}$.\n
    \indent $U$ must intersect \emph{some} $A_a$, but not every $U$ intersects the same $A_a$. 
  \item Let $A, B$, and $A_a$ denote subsets of a space $X$. Determine whether the following equations hold; if an equality fails, determine whether one of the inclusions $\supset$ or $\subset$ holds.
    \begin{enumerate}
    \item $\overline{A\cap B} = \bar{A}\cap\bar{B}$.\hspace{5in}\n
      \indent Consider an element $x$ of $\overline{A \cap B}$. Every neighborhood of $x$ intersects $A\cap B$, thus intersecting both $A$ and $B$, thus $x$ is in both $\bar{A}$ and $\bar{B}$. Therefore, $\overline{A\cap B}\subset \bar{A}\cap\bar{B}$. In the other direction, consider $x$ in $\bar{A}\cap\bar{B}$. Specifically, consider an $x$ that is not in $A$ or $B$, but is in both $\bar{A}$ and $\bar{B}$. Such an element might not be in $\overline{A\cap B}$. For example, consider $A=\mathbb{R}_+$ and $B=\mathbb{R}_-$. Both closures contain $0$, but the intersection of these sets is empty. Thus equality does not hold, only $\overline{A\cap B}\subset \bar{A}\cap{B}$.
    \item $\overline{\bigcap A_a} = \bigcap\bar{A}_a$.\hspace{5in}\n
      \indent The same argument as above applies. $\overline{\bigcap A_a}\subset\bigcap\bar{A}_a$.
    \item $\overline{A-B}=\bar{A}-\bar{B}.$ \hspace{5in}\n
      \indent Consider an $x$ that lies in $\bar{A}$, $\bar{B}$, and $\overline{A-B}$. This $x$ won't exist in $\bar{A}-\bar{B}$. Once again, the example of $A=\mathbb{R}_+$, $B=\mathbb{R}_-$ illustrates this. The difference of the closures is just $\mathbb{R}_+$, while the closure of the difference is $\bar{\mathbb{R}}_+=\mathbb{R}_+ + \{0\}$. Thus at best $\overline{A-B}\supset\bar{A}-\bar{B}$. To prove this direction, consider an $x$ in $\bar{A}-\bar{B}$. All neighborhoods of $x$ intersect $A$, but there exists some neighborhood $U$ that does not intersect $B$. Suppose there is a neighborhood $V$ that does not intersect $A-B$. Then $U\cap V$ must intersect $A$, since it's a neighborhood of $x$. Since it does not intersect $A-B$, the only way it could intersect $A$ is if it intersected $B$, i.e. if it intersected $A\cap B$. But $U$ does not intersect $B$, a contradiction. Thus we conclude that $\overline{A-B}\supset\bar{A}-\bar{B}$.
    \end{enumerate}
  \item Let $A\subset X$ and $B\subset Y$. Show that in the space $X\times Y$, $\overline{A\times B}=\bar{A}\times\bar{B}$.\n
    \indent Consider an $x_0\times y_0\in \bar{A} \times \bar{B}$. Every neighborhood of $x_0$ must intersect $A$, and every neighborhood of $y_0$ must intersect $B$. Identically, for an element $x_1\times y_1\in\overline{A\times B}$, every neighborhood of $x_1$ intersects $A$ and every neighborhood of $y_1$ intersects $B$. 
  \item Show that every order topology is Hausdorff.\n
    \indent Consider the pair of distinct points $a,b$ in a topological space $X$. \WLOG, assume $a<b$. Then either there exists a point $c$ such that $a<c<b$, or $b$ is the immediate successor of $a$. In the first case, the neighborhoods $(-\infty, c)$ and $(c,\infty)$ are disjoint neighborhoods of $a$ and $b$ respectively, so the space is Hausdorff. In the second case, $(-\infty, b)$ and $(a,\infty)$ are the disjoint neighborhoods. 
  \item Show that the product of two Hausdorff spaces $X$ and $Y$ is Hausdorff.\n
    \indent For every $x_1,x_2$ in $X$ there exist disjoint neighborhoods $U_1$ and $U_2$. Similarly, for every $y_1,y_2$ in $Y$ there exist disjoint neighborhoods $V_1$ and $V_2$. Thus in the product, for every $x_1\times y_1$ and $x_2\times y_2$ in $X\times Y$, there exist disjoint neighborhoods $U_1\times V_1$ and $U_2\times V_2$.
  \item Show that the subspace $A$ of a Hausdorff space $X$ is Hausdorff.\n
    \indent For every $x_1,x_2$ in $X$ there exist disjoint neighborhoods $U,V$. For such elements, the neighborhoods $U\cap A, V\cap A$ will be disjoint in the subspace topology.
  \item Show that $X$ is Hausdorff if and only if the diagonal $\Delta = \{x\times x\; |\; x\in X\}$ is closed in $X\times X$.\hspace{5in}\n
    % \indent If $X$ is Hausdorff, then $X\times X$ is Hausdorff. Therefore every element $x\times x$ is closed, by theorem 17.8.
    % If $\Delta$ is closed, then $C=X\times X - \Delta$ is open. 
    \indent If $X$ is Hausdorff, for every pair of distinct points $x_1,x_2\in X$, they have disjoint neighborhoods $U,V$. Clearly $x_1 \times x_2$ is in the open set $U\times V$. Furthermore, $U\cap V=\emptyset$ implies that $U\times V \cap \Delta=\emptyset$, since if the latter intersection was not empty, that would imply there was an element $c\times c$ in $U\times V$, i.e. $c\in U$ and $c\in V$, which contradicts the previous assertion. Since every element of $X\times X - \Delta$ lies in an open set, $X\times X - \Delta$ is open, so $\Delta$ is closed.\n
    \indent In the other direction proceeds similarly: if the diagonal is closed, then it's complement is open, i.e. every pair of distinct points $x_1,x_2$ lies in an open set $U\times V$. $x_1\in U$ and $x_2\in V$. If $U\cap V\neq \emptyset$, then there exists an element $c$ such that $c\in U$ and $c\in V$, so $c\times c\in U\times V$, but $c\times c\in \Delta$, a contradiction as $U$ and $V$ are subsets of the complement of $\Delta$. Thus $U,V$ are disjoint, so $X\times X$ is Hausdorff.
\end{enumerate}
\end{document}
