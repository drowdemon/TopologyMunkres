\documentclass[12pt,letterpaper]{article}
\usepackage[pdftex]{graphicx}
\usepackage{alltt}
\usepackage[margin=1in]{geometry}
\usepackage{amsmath, amsthm, amssymb}
\usepackage{verbatim}
\usepackage{ragged2e}
\usepackage{enumitem}
\usepackage{xfrac}
\setlist{parsep=0pt,listparindent=\parindent}
\setlength{\RaggedRightParindent}{\parindent}
\newcommand{\degree}{\ensuremath{^\circ}}
\newcommand{\n}{\break}
\let\oldemptyset\emptyset
\let\emptyset\varnothing
\newcommand{\Wlog}{without loss of generality}
\newcommand{\WLOG}{Without loss of generality}
\usepackage{accents}
\let\thinbar\bar
\newcommand\thickbar[1]{\accentset{\rule{.4em}{.8pt}}{#1}}
\let\bar\thickbar
\usepackage{standalone}
\usepackage{hyperref}
\newcommand{\R}{\ensuremath{\mathbb{R}}}
\usepackage{mathtools}
\DeclarePairedDelimiter{\ceil}{\lceil}{\rceil}
\DeclarePairedDelimiter{\floor}{\lfloor}{\rfloor}
\DeclarePairedDelimiter\abs{\lvert}{\rvert}
\DeclarePairedDelimiter\norm{\lVert}{\rVert}
%%%%%%%%%%%%%%%%%%%%%%%%%%%%%%%%%%%%%%%%%%%%%%%%%%%%%
%TOPOLOGY DOCUMENTS ONLY%
\newcommand{\T}{\ensuremath{\mathcal{T}}}
%%%%%%%%%%%%%%%%%%%%%%%%%%%%%%%%%%%%%%%%%%%%%%%%%%%%%

\begin{document}
\RaggedRight
\begin{enumerate}
  \item Prove theorem 19.2, which is as follows: Suppose the topology on each space $X_\alpha$ is given by a basis $\mathcal{B}_\alpha$. The collection of all sets of the form $\prod_{\alpha\in J}B_\alpha$ where $B_\alpha \in \mathcal{B}_\alpha$ for each $\alpha$, will serve as a basis for the box toplogy on $\prod_{\alpha\in J} X_\alpha$. The collection of all sets of the same form, where $B_\alpha \in \mathcal{B}_\alpha$ for finitely many indices $\alpha$ and $B_\alpha=X_\alpha$ for all remaining indices, will serve as a basis for the product topology $\prod_{\alpha\in J}X_\alpha$.\hspace{5in}\n
  \indent Let the collection that is to be proven to be a basis for the topology in question over the product space $X=\prod_{\alpha\in J}X_\alpha$ be denoted $\mathcal{C}$. Let $x=(x_\alpha)$ be any point of $\prod X_\alpha$. We show that $\mathcal{C}$ is a basis. \n
  \indent For each $x_\alpha$ there exists a basis element $B_\alpha$ containing it (or $X_\alpha$ obviously contains it), since each $\mathcal{B}_\alpha$ is a basis; call each such element $C_\alpha$. Then $\prod_\alpha C_\alpha$ gets an element of $\mathcal{C}$ which contains $x$. Hence there is at least a basis element containing any $x\in X$.
  Next, suppose $x$ belongs to the intersection of two basis elements, $C_1$ and $C_2$. This is equivalent to $\prod_\alpha C_{1_\alpha} \cap C_{2_\alpha}$. Since each $x_\alpha$ must belong to $C_{1_\alpha} \cap C_{2_\alpha}$, there must be a corresponding basis element $C_{3_\alpha}\in\mathcal{B}_\alpha$ such that $x\in C_{3_\alpha}\subset C_{1_\alpha} \cap C_{2_\alpha}$ since each $\mathcal{B}_\alpha$ is a basis. Then $\prod_\alpha C_{3_\alpha}$ is an element of $\mathcal{C}$. Hence, $\mathcal{C}$ is a basis.\n
  \indent Now we show that the topology generated by $\mathcal{C}$ is the correct topology.
  We show that $\mathcal{C}$ is equivalent to the basis that generally forms the box topology, the collection of all sets of the form $\prod_\alpha U_\alpha$, where $U_\alpha$ is open in $X_\alpha$ for all $\alpha\in J$. Consider an element of the defining basis, $U=\prod_\alpha U_\alpha$. Since each $\mathcal{B}_\alpha$ is a basis, each $U_\alpha$ can be formed by some union of elements of $\mathcal{B}_\alpha$. Hence $U$ can be formed by unions of elements of $\mathcal{C}$. The same argument applies when looking at the basis generated by the defining subbasis of the product topology.
  \item Prove theorem 19.3, which is as follows: Let $A_\alpha$ be a subspace of $X_\alpha$ for each $\alpha \in J$. Then $A=\prod A_\alpha$ is a subsapce of $X=\prod X_\alpha$ if both products are given either the box topology or the product topology.\hspace{5in}\n
  \indent Let $\T_s$ be the topology $A$ inherits from $X$ in the product topology. Hence, $\T_s$ is generated by $\mathcal{B}_s = \{ B \cap A | B\in \mathcal{B} \}$, where $\mathcal{B}$ is the basis of the $X$ with the product topology. Let $\T_p$ be the product topology on $A$, with the basis $\mathcal{B}_p=\prod U_\alpha$, where $U_\alpha$ is open in $A_\alpha$ and equal to $A_\alpha$ except for finitely many values of $\alpha$. \n
  \indent Consider an element $B$ of $\mathcal{B}_s$. $B=\prod B_\alpha \cap A_\alpha$. Since $B_\alpha \cap A_\alpha$ is open in $A_\alpha$, and where $B_\alpha$ is equal to $X_\alpha$, $B_\alpha \cap A_\alpha = A_\alpha$, $B\in \mathcal{B}_p$.\hspace{5in}\n
  \indent Now consider an element $B$ of $\mathcal{B}_p$. Any open element $U_\alpha$ of $A_\alpha$ must be equal to some $V_\alpha \cap A_\alpha$, where $V_\alpha$ is an open set of $A_\alpha$, because $A_\alpha$ is a subspace of $X_\alpha$. Thus, $B=\prod V_\alpha \cap A_\alpha$. For all but a finite number of $\alpha$s, $V_\alpha$ may be $X_\alpha$, since the corresponding $U_\alpha$ is $A_\alpha$ for all but a finite number of $\alpha$s. Hence, $B\in \mathcal{B_s}$, since the corresponding $B_\alpha$ of a $B\in \mathcal{B}$ may be any open set of $X_\alpha$. The proof follows similarly for the box topology.
  \item Prove Theorem 19.4, which is: If each space $X_\alpha$ is a Hausdorff space, then $X=\prod X_\alpha$ is a Hausdorff space in both the box and the product topologies. \hspace{5in} \n
  \indent For any distinct points $x', x'' \in X$ there exist disjoint neighborhoods $U_\alpha', U_\alpha''$ for $x', x''$ respectively, for each $\alpha \in J$ where $x_\alpha' \neq x_\alpha''$, in the other cases, let $U_\alpha'=U_\alpha''$ be any neighborhood of $x_\alpha'$. Thus there exist disjoint neighborhoods $U' = \prod U_\alpha',\; U''=\prod U_\alpha''$, thus $X$ is hausdorff.
  \addtocounter{enumi}{1}
  \item One of the implications stated in Theorem 19.6 holds for the box topology. Which one?\n
  \indent If $f$ is continuous, then $f_\alpha$ is continuous for each $\alpha$.
  \item Let $x_1, x_2, \dots$ be a sequence of points of the product space $X=\prod X_\alpha$. Show that this sequence converges to the point $x$ if and only if the sequence $\pi_\alpha(x_1), \pi_\alpha(x_2), \dots$ converges to $\pi_\alpha(x)$ for each $\alpha$. Is this fact true if one uses the box topology instead of the product topology? \hspace{5in} \n
  \indent Suppose the sequence converges to $x$. Then every neighborhood $U$ of $x$ contains every $x_n$ for all $n$ larger than some $m_U$. Hence, every neighborhood $\pi_\alpha(U)$ of $\pi_\alpha(x)$ contains every $\pi_\alpha(x_n)$ for all $n$ larger than some $m_U$, for each $\alpha$. Furthermore, every neighborhood $V$ of $\pi_\alpha(x)$ has a corresponding neighborhood of $x$: $\pi^{-1}_\alpha(x)$. Thus, since neighborhood $\pi_\alpha(U)$ of $\pi_\alpha(x)$ is every neighorhood $\pi_\alpha(x)$. Therefore, every sequence $\pi_\alpha(x_1), \pi_\alpha(x_2),\dots$ converges to $\pi_\alpha(x)$.\n
  \indent Now suppose every sequence $\pi_\alpha(x_1), \pi_\alpha(x_2)\dots$ converges to $\pi_\alpha(x)$. Any open neighborhood $U_\alpha$ of $\pi_\alpha(x)$ in $X_\alpha$ contains all $\pi_\alpha(x_n)$ greater than some $m_{U_\alpha}$. Use finitely many $U_\alpha$s to construct an open neighborhood of $x$; label these by $i$. $U=\prod U_\alpha$, where $U_\alpha=U_i$ for finitely many values of $\alpha$, and $U_\alpha=X_\alpha$ for the others. Then, for all values $n$ over $\max_\alpha(m_{U_\alpha})$, $U$ will contain $x_n$, thus the sequence converges to $x$.\n
  \indent In the box topology, the second result does not hold, since if the value $U_{m_\alpha}$ increases unboundedly, the final sequence does not converge. For example, in the space $\mathbb{R}^\omega$, consider $x=(0,0,\dots)$, its neighborhood $((-1,1),\allowbreak (-1/2,1/2),\allowbreak (-1/4,1/4),\allowbreak (-1/8,1/8),\allowbreak (-1/16,1/16)\dots)$, and the sequence $x_n = (1/n,1/n,1/n,\dots)$.
  \item Let $\mathbb{R}^\infty$ be the subset of $\mathbb{R}^\omega$ consisting of all sequences that are ``eventually zero,'' that is, all sequences $(x_1, x_2,\dots)$ such that $x_i\neq 0$ for only finitely many values of $i$. What is the closure of $\mathbb{R}^\infty$ in $\mathbb{R}^\omega$ in the box and product topologies?\\
  Suppose $x=x_1,x_2,\dots$ is a point in the product topology, and let $U$ be a neighborhood of $x$. $U=\prod U_i$, where for some $n$, all $U_{i>n}=\R$. Therefore, the point $y = (x_1,x_2,\dots,x_n,0,0,\dots)\in U$ and $y\in R^\infty$, so $U$ intersects $\R^\infty$, so $x$ is a limit point of $\R^\infty$, so $R^\omega$ is the closure of $\R^\infty$\\
  In the box topology, $\R^\infty$ is its own closure. Suppose a point $x=(x_1,x_2,\dots)\in\R^\omega$ outside of $\R^\infty$ were a limit point. But $U=(\prod_i \min(x_i/2, 2x_i),\max(2x_i,x_i/2))$ is an open set in the box topology that does not intersect $R^\infty$. Correction: if $x_i=0$, replace $U_i$ by $(-1,1)$, it is currently $(0,0)$ which would not work.
  \item Given sequences $(a_1, a_2, \dots)$ and $(b_1, b_2,\dots)$ of real numbers $a_i > 0$ for all $i$, define $h: \mathbb{R}^\omega \rightarrow \mathbb{R}^\omega$ by the equation $h((x_1, x_2, \dots)) = (a_1x_1+b_1, a_2x_2+b_2, \dots)$. Show that if $\mathbb{R}^\omega$ is given the product topology, $h$ is a homeomorphism of $\mathbb{R}^\omega$ with itself. What happens if $\mathbb{R}^\omega$ is given the box topology?\hspace{5in}\n
  \indent First, to prove that $h$ is a homeomorphism, we prove that it is bijective. To do that, we first show that it is injective. Suppose the opposite, there exist distinct $x$ and $y$ such that $h(x)=h(y)$. Then $a_1x_1+b_1 = a_1y_1 + b_1$, so $x=y$, a contradiction. Now we show that $h$ is surjective, again by contradiction. Suppose there is a point $y$ such that there exists no $x$ such that $h(x)=y$. However, the point $x=((y_1-b_1)/a_1, (y_2-b_2)/a_2, \dots)$ is such a point. Thus, $h$ is bijective. \hspace{5in}\n
  \indent Now we show that $h$ is continuous. Each function $h_1(x) = a_1x_1+b_1, h_2(x)=a_2x_2+b_2, \dots$ is clearly continuous, thus, by \hyperref[thm:MapsProducts19.6]{theorem 19.6}, $h$ is continous. Similarly, $h^{-1}$ is continuous, thus $h$ is homeomorphic.\hspace{5in}\n
  \indent In the box topology, theorem 19.6 does not hold, $h$ would not be continuous, and so would not be a homeomorphism. For example, if $a=1$ and $b=0$, we get exactly the function Munkres uses to show that this theorem does not hold for the box topology.
  \item Show that the \hyperref[thm:AxiomChoice]{axiom of choice} is equivalent to the statement that for any indexed family $\{A_\alpha\}_{\alpha\in J}$ of nonempty sets, with $J\neq 0$, the cartesian product $\prod_{\alpha\in J} A_\alpha$ is not empty.\n
  \indent Intuitively, each element of the cartesian product requires the choice of an arbitrary element from each set $A_\alpha$.
  \item Let $A$ be a set; let $(X_\alpha)_{\alpha\in J}$ be an indexed family of spaces; and let $(f_\alpha)_{\alpha\in J}$ be an indexed family of functions $f_\alpha: A \rightarrow X_\alpha$.
  \begin{enumerate}
    \item Show that there is a unique coarsest topology $\T$ on $A$ relative to which each of the functions $f_\alpha$ is continuous.\hspace{5in}\n
    \indent Suppose there were two such coarsest topologies, $\T$ and $\T'$. Each would contain at least one open set, $U$ and $U'$ respectively which the other did not contain. There then exist two $\alpha$, $i$ and $j$, and two open sets $V$ and $V'$ of $X_i$ and $X_j$ such that $f_i^{-1}(V)=U$ and $f_j^{-1}(V')=U'$. However, for $f_j$ to be continuous, $T'$ must contain $U$ and vice versa, hence there is a single unique coarsest topology. NOTE: This shows that the subbasis for such a topology must be unique, not the topology itself. \n
    \indent Consider the intersection of $\T$ and $\T'$. Each $f_\alpha$ is continuous in this intersection, since it is continuous in each topology, but the intersection is necessarily coarser. Thus there is only one possible coarsest topology. 
    \item Let $\mathcal{S}_\beta=\{f_\beta^{-1}(U_\beta) | U_\beta \text{ is open in } X_\beta\}$, and let $\mathcal{S}=\bigcup \mathcal{S}_\beta$. Show that $\mathcal{S}$ is a subbasis for $\T$.\hspace{5in}\n
    \indent For each function to be continuous, it is necessary that each $f_\alpha^{-1}(U_\alpha)$ be open, where $U_\alpha$ is an open set of $X_\alpha$. This subbasis allows for exactly that, without including any other sets. The inverse of each open set must be part of a subbasis to be sure that it forms a topology, they cannot simply be the topology directly.
    \item Show that a map $g: Y\rightarrow A$ is continuous relative to $\T$ if and only if each map $f_\alpha \circ g$ is continuous.\hspace{5in}\n
    \indent If $g$ is continuous, then each $f_\alpha \circ g$ is continuous, since the composition of continuous functions is continuous. Conversely, suppose each $f_\alpha \circ g$ is continuous, but $g$ is not continous. %Then there exists some open set $V\in A$ such that $g^{-1}(V)$ is not open in $Y$. Denote $f_\alpha(V)$ as $U_\alpha$. Then, each $g^{-1}(f^{-1}_\alpha(U_\alpha))$ is open in $Y$.
    % $V$ is some union of some finite intersection sets in $\mathcal{S}$.
    However, consider each subbasis element $U\in\mathcal{S}$. For each one, there exists some $\alpha$ such that there exists an open set $V\in X_\alpha$ with $f_\alpha^{-1}(V)=U$. $g^{-1}(f^{-1}_\alpha(V))$ is open by hypothesis, so $g^{-1}(U)$ is open. It is sufficient to show that the inverse image of each subbasis element is open to show that a function is continuous, thus $g$ is continuous.
    \item Let $f: A\rightarrow \prod X_\alpha$ be defined by the equation $f(a) = (f_\alpha(a))_{\alpha\in J}$; let $Z$ denote the subspace $f(A)$ of the product space $\prod X_\alpha$. Show that the image under $f$ of each element of $\T$ is an open set of $Z$. \hspace{5in}\n
    \indent %In the subspace $f(A)$, $f$ is obviously surjective. Identically, each $\pi_\alpha\circ f = f_\alpha$ maps to the space $\pi_\alpha(f(A)) = f_\alpha(A)$, so each $f_\alpha$ is surjective.
    Let $U$ belong to the subbasis $\mathcal{S}$ of $\T$, $U$ is the preimage of some open set $V\in X_\beta$ for some $\beta$. $f_\beta(U) = f_\beta(f_\beta^{-1}(V)) = V \cap f_\beta(A)$, thus it is an open set in the subspace $f_\beta(A)$.\n
    % Now, consider the set $B=\prod_{\alpha\in J} V_\alpha$, where $V_\alpha = f_\alpha(A)$ for all $\alpha \neq \beta$, and $U_\alpha = f_\alpha(U)$ when $\alpha = \beta$.
    %Let $B=f(U)$, and let $x$ be an element of $B$. %$B$ must be open in $\prod X_\alpha$, because
    The set $\prod V_\alpha$ where $V_\alpha = X_\alpha$ for all $\alpha \neq \beta$ and $V_\alpha=V$ when $\alpha=\beta$ is open in $\prod X_\alpha$. The intersection $B = \prod V_\alpha\cap f(A)$ is exactly $f(U)$. Proof proceeds: consider an $x\in U$. $f_\beta(x) \in V$, and for any other $\alpha$, $f_\alpha(x)$ is obviously in $X_\alpha$, so $f(x)\in B$. In the other direction, consider an $x\in B$. $\pi_\beta(x) \in f_\beta(U)$. Suppose there were a $y$ such that $f_\beta(y)=\pi_\beta(x)$, but $f(y)\not\in f(U)$. However, $y\in f^{-1}_\beta(\pi_\beta(x))$, equivalently, $y\in f^{-1}_\beta(f_\beta(U))$, so $y\in U$. Thus, $\pi_\beta(x)$ being an element of $f_\beta(U)$ is a sufficient condition for $x\in U$. %However, $y$ must be in $f^{-1}_\beta(V)$, and therefore in $U$, since $U=f^{-1}_\beta(V)$, and $\pi_\beta(x) \in f^{-1}(U)$, so $f(\pi_\beta(x)) \in U$ \n
    $B$ is clearly an open set of $Z$, so $f(U)$ is an open element of $Z$, where $U$ is an arbitrary subbasis element of $\T$. To extend this to arbitrary basis elements, simply note that these are finite intersections of subbasis elements, so if $U$ were a basis element, then $U$ would be intersection of the preimage of some finite sets $V_1, V_2,\dots$ of $X_{\beta_1},X_{\beta_2},\dots$ for some finite number of subscripts $\beta$. Since a finite number of the open sets comprising a basis element of the product topology can be any open set, the proof proceeds similarly. The function of a union is the union of the function of each set, thus we are done.
  \end{enumerate}
\end{enumerate}
\end{document}