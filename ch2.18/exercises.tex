\documentclass[12pt,letterpaper]{article}
\usepackage[pdftex]{graphicx}
\usepackage{alltt}
\usepackage[margin=1in]{geometry}
\usepackage{amsmath, amsthm, amssymb}
\usepackage{verbatim}
\usepackage{ragged2e}
\usepackage{enumitem}
\usepackage{xfrac}
\setlist{parsep=0pt,listparindent=\parindent}
\setlength{\RaggedRightParindent}{\parindent}
\newcommand{\degree}{\ensuremath{^\circ}}
\newcommand{\n}{\break}
\let\oldemptyset\emptyset
\let\emptyset\varnothing
\newcommand{\Wlog}{without loss of generality}
\newcommand{\WLOG}{Without loss of generality}
\usepackage{accents}
\let\thinbar\bar
\newcommand\thickbar[1]{\accentset{\rule{.4em}{.8pt}}{#1}}
\let\bar\thickbar

%%%%%%%%%%%%%%%%%%%%%%%%%%%%%%%%%%%%%%%%%%%%%%%%%%%%%
%THIS DOCUMENT ONLY%
\newcommand{\T}{\ensuremath{\mathcal{T}}}
%%%%%%%%%%%%%%%%%%%%%%%%%%%%%%%%%%%%%%%%%%%%%%%%%%%%%

\begin{document}
\RaggedRight
\begin{enumerate}
  \setcounter{enumi}{1}
\item Suppose that $f: X\rightarrow Y$ is continuous. If $x$ is a limit point of the subset $A$ of $X$, is it necessarily true that $f(x)$ is a limit point of $f(A)$.\n
\indent Consider the topology on the set $X=\{a,b,c\}$: $\T_1 = \{\{a,b\}, \{b,c\},\{a,b,c\},\emptyset\}$, and the topology on the set $Y=\{d,e\}$: $\T_2 = \{\emptyset, \{d,e\}\}$. The mapping $F: X\rightarrow Y$ is defined such that $f(a)=d,\ f(b)=e,\ f(c)=e$. The limit points of $\{b\}$ are $a$ and $c$. However, the limit point of $\{f(b)\}=\{e\}$ is $d$. $f(c)=e$ cannot be a limit point, since the open set containing it, $\{e\}$, only intersects $\{f(b)\}$ at itself, not any other point. That the mapping is continuous is easy to check.\n
\indent In general, let $V$ be a neighborhood of $f(x)$. Then $f^{-1}(V)$ is an open set of $X$ containing $x$, so it must intersect $A$ at some point $y$. $f(y)\in V$ and $f(y)\in f(A)$, therefore $V$ intersects $f(A)$ at $f(y)$. If $f(y)\in f(x)$ for all such points $y$ then $f(x)$ is not a limit point, but if this is false for all open sets $V$, $f(x)$ will be a limit point of $A$.
\item Let $X$ and $X'$ denote a single set in the two toplogies $\T$ and $\T'$, respectively. Let $i: X'\rightarrow X$ be the identity function.
  \begin{enumerate}
  \item Show that $i$ is continuous $\leftrightarrow \T'$ is finer than $\T$.\hspace{5in}\n
    \indent Clearly, $i$ is it's own inverse. Then, suppose $i$ is continuous. Then for an open set $V$ of $X$, $i(V)=V$ is an open set of $X'$. Hence, every open set $V$ of $X$ is an open set of $X'$, i.e. $\T'$ is finer than $\T$. \n
    \indent In the other direction, suppose that $\T'$ is finer than $\T$. Then for an open set $V$ of $X$, $i(V)=V$ is in $X'$, so $i$ is continuous.
  \item Show that $i$ is a homeomorphism $\leftrightarrow \T'=\T$.\hspace{5in}\n
    \indent Suppose $i$ is a homeomorphism. Then $i$ is bijective. This implies that $X=X'$. This also implies that every open sey $U$ of $X$ is open in $X'$, so the topologies must be equivalent. The argument is similar in the other direction.
  \end{enumerate}
\item Given $x_0\in X$ and $y_0\in Y$, show that the maps $f: X\rightarrow X\times Y$ and $g: Y\rightarrow X\times Y$ defined by $f(x)=x\times y_0$ and $g(y)=x_0\times y$ are imbeddings. \hspace{5in}\n
\indent Let $Z=X\times y_0$. Then $f(x)$ is just a line such that the $x$-coordinate is $x$ and the $y$ coordinate is $y_0$. This is equivalent to the identity function cross $y_0$, so it is homeorphic when considered on the range $X \times c$, and an imbedding in $X\times Y$. The same goes for $g$.




\item Let $\{A_\alpha\}$ be a collection of subsets of $X$; let $X=\bigcup_\alpha A_\alpha$. Let $f: X\rightarrow Y$; suppose that $f|A_\alpha$ is continuous for each $\alpha$. 
\begin{enumerate}
\item Show that if the collection $\{A_\alpha\}$ is finite and each set $A_\alpha$ is closed, then $f$ is continuous.\hspace{5in}\n
\indent The local formulation of continuity shows that this would be true if each $A$ were open. Furthermore, by part 3 of theorem 18.1, just like for open sets, if $f$ is continuous, then for every closed set $B$ of $Y$, the set $f^{-1}(B)$ is closed in $X$. Consider such a closed set $B$ of $Y$. $f^{-1}(V)\cap A_\alpha = (f|A_\alpha)^{-1}(B)$: both expressions are exactly those points $x$ in $A_\alpha$ such that $f(a)\in B$. Since $f|A_\alpha$ is continuous, $(f|A_\alpha)^{-1}(B)$ is closed. $f^{-1}(B) = \bigcup_\alpha(f|A_\alpha)^{-1}(B)$. The finite union of closed sets is closed, so $f^{-1}(B)$ is closed, so $f$ is continuous.
\item Find an example where the collection $\{A_\alpha\}$ is countable and each $A_\alpha$ is closed, but $f$ is not continuous.\hspace{5in}\n
\indent Consider 
$$A_0={0} \text{ and } A_{n>0}=\{\frac{1}{n}\text{, and } f(x)=\begin{cases} x & \quad x\neq 0 \\ 10 & \quad x=0\end{cases}$$
\item An indexed family of sets $\{A_\alpha\}$ is said to be $\emph{locally finite}$ is each point $x$ of $X$ has a neighborhood that intersects $A_\alpha$ for only finitely many values of $\alpha$. Show that if the family $\{A_\alpha\}$ is locally finite and each $A_\alpha$ is closed, then $f$ is continuous.
\end{enumerate}
\end{enumerate}
\end{document}
