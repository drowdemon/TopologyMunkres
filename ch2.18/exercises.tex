\documentclass[12pt,letterpaper]{article}
\usepackage[pdftex]{graphicx}
\usepackage{alltt}
\usepackage[margin=1in]{geometry}
\usepackage{amsmath, amsthm, amssymb}
\usepackage{verbatim}
\usepackage{ragged2e}
\usepackage{enumitem}
\usepackage{xfrac}
\setlist{parsep=0pt,listparindent=\parindent}
\setlength{\RaggedRightParindent}{\parindent}
\newcommand{\degree}{\ensuremath{^\circ}}
\newcommand{\n}{\break}
\let\oldemptyset\emptyset
\let\emptyset\varnothing
\newcommand{\Wlog}{without loss of generality}
\newcommand{\WLOG}{Without loss of generality}
\usepackage{accents}
\let\thinbar\bar
\newcommand\thickbar[1]{\accentset{\rule{.4em}{.8pt}}{#1}}
\let\bar\thickbar
\usepackage{standalone}
\usepackage{hyperref}
\newcommand{\R}{\ensuremath{\mathbb{R}}}
\usepackage{mathtools}
\DeclarePairedDelimiter{\ceil}{\lceil}{\rceil}
\DeclarePairedDelimiter{\floor}{\lfloor}{\rfloor}
\DeclarePairedDelimiter\abs{\lvert}{\rvert}
\DeclarePairedDelimiter\norm{\lVert}{\rVert}
%%%%%%%%%%%%%%%%%%%%%%%%%%%%%%%%%%%%%%%%%%%%%%%%%%%%%
%TOPOLOGY DOCUMENTS ONLY%
\newcommand{\T}{\ensuremath{\mathcal{T}}}
%%%%%%%%%%%%%%%%%%%%%%%%%%%%%%%%%%%%%%%%%%%%%%%%%%%%%

\begin{document}
\RaggedRight
\begin{enumerate}
  \setcounter{enumi}{1}
  \item Suppose that $f: X\rightarrow Y$ is continuous. If $x$ is a limit point of the subset $A$ of $X$, is it necessarily true that $f(x)$ is a limit point of $f(A)$.\n
  \indent Consider the topology on the set $X=\{a,b,c\}$: $\T_1 = \{\{a,b\}, \{b,c\},\{a,b,c\},\emptyset\}$, and the topology on the set $Y=\{d,e\}$: $\T_2 = \{\emptyset, \{d,e\}\}$. The mapping $F: X\rightarrow Y$ is defined such that $f(a)=d,\ f(b)=e,\ f(c)=e$. The limit points of $\{b\}$ are $a$ and $c$. However, the limit point of $\{f(b)\}=\{e\}$ is $d$. $f(c)=e$ cannot be a limit point, since the open set containing it, $\{e\}$, only intersects $\{f(b)\}$ at itself, not any other point. That the mapping is continuous is easy to check.\n
  \indent In general, let $V$ be a neighborhood of $f(x)$. Then $f^{-1}(V)$ is an open set of $X$ containing $x$, so it must intersect $A$ at some point $y$. $f(y)\in V$ and $f(y)\in f(A)$, therefore $V$ intersects $f(A)$ at $f(y)$. If $f(y)=f(x)$ for all such points $y$ then $f(x)$ is not a limit point, but if this is false for all open sets $V$, $f(x)$ will be a limit point of $f(A)$.
  \item Let $X$ and $X'$ denote a single set in the two toplogies $\T$ and $\T'$, respectively. Let $i: X'\rightarrow X$ be the identity function.
  \begin{enumerate}
    \item Show that $i$ is continuous $\leftrightarrow \T'$ is finer than $\T$.\hspace{5in}\n
    \indent Clearly, $i$ is it's own inverse. Then, suppose $i$ is continuous. Then for an open set $V$ of $X$, $i(V)=V$ is an open set of $X'$. Hence, every open set $V$ of $X$ is an open set of $X'$, i.e. $\T'$ is finer than $\T$. \n
    \indent In the other direction, suppose that $\T'$ is finer than $\T$. Then for an open set $V$ of $X$, $i(V)=V$ is in $X'$, so $i$ is continuous.
    \item Show that $i$ is a homeomorphism $\leftrightarrow \T'=\T$.\hspace{5in}\n
    \indent Suppose $i$ is a homeomorphism. Then $i$ is bijective. This implies that $X=X'$. This also implies that every open sey $U$ of $X$ is open in $X'$, so the topologies must be equivalent. The argument is similar in the other direction.
  \end{enumerate}
  \item Given $x_0\in X$ and $y_0\in Y$, show that the maps $f: X\rightarrow X\times Y$ and $g: Y\rightarrow X\times Y$ defined by $f(x)=x\times y_0$ and $g(y)=x_0\times y$ are imbeddings. \hspace{5in}\n
  \indent Let $Z=X\times y_0$. Then $f(x)$ is just a line such that the $x$-coordinate is $x$ and the $y$ coordinate is $y_0$. This is equivalent to the identity function cross $y_0$, so it is homeorphic when considered on the range $X \times c$, and an imbedding in $X\times Y$. The same goes for $g$.


  \item Show that the subspace $(a,b)$ of $\mathbb{R}$ is homeomorphic with $(0,1)$ and the subspace $[a,b]$ of $\mathbb{R}$ is homeomorphic with $[0,1]$.\n
  \indent Consider the function from $(a,b)$ to $(0,1)$: $f(x)=\frac{x-a}{b-a}$. The inverse is $f^{-1}(x)=x*(b-a)+a$. That this is the inverse and that $f(f^{-1}(x)) = x$ and $f^{-1}(f(x))=x$ is easily checked, so $f$ is bijective. For any open set $U$ of $(a,b)$, $f(U)$ will be open, the converse also holds, so it is homeomorphic. The same function will work for the second example.
  \item Find a function $f: \mathbb{R}\rightarrow \mathbb{R}$ that is continuous at precisely one point.\hspace{5in}\n
  \indent Consider the function $f(x) = \begin{cases} x & \text{if } x\in\mathbb{Q} \\ 0 & \text{if } x\not\in\mathbb{Q}\end{cases}\quad$ Then, $f$ is continuous only at the point 0. \hspace{3in}\n
  Consider a neighborhood $V$ of $f(0)$ of the form $(a,b)$, where $a<0$ and $b>0$. $f^{-1}(V)$ is all of the rationals in the set $(a,b)$ together with all of the irrationals, which clearly contains the set $(a,b)$. So $f$ is continuous at 0. Now consider $f$ at some other point, $x_0$, and a neighborhood $V$ of $f(x_0)$. Suppose that $x_0$ is rational, and that $V$ does not contain 0 - this is possible since any interval $(a,b)$ containing $x_0\neq 0$ can be made smaller until it does not contain 0. Then, $f^{-1}(V)$ does not contain any irrationals, but $V$ does, so $f$ is not continuous at $x_0$. If $x_0$ is irrational, then $f(x_0)=0$, so $V$ is a neighborhood of $0$. Then, consider a $V$ that contains only elements closer to 0 than $x$. Then $f^{-1}(V)$ will not contain any rationals close to $x_0$, so it will not contain any neighborhood of $x_0$, so $f$ is not continuous at $x_0$.
  \item
  \begin{enumerate}
    \item Suppose that $f: \mathbb{R} \rightarrow \mathbb{R}$ is ``continuous from the right,'' that is, $$\lim_{x\rightarrow a^+} f(x)=f(a),$$ for each $a\in\mathbb{R}$. Show that $f$ is continuous when considered as a function from $\mathbb{R}_\ell$ to $\mathbb{R}$.\hspace{5in}\n
    \item Can you conjecture what functions $f:\mathbb{R}\rightarrow \mathbb{R}$ are continuous when considered as maps from $\mathbb{R}$ to $\mathbb{R}_l$? As maps from $\mathbb{R}_\ell$ to $\mathbb{R}_\ell?$ We shall return to this question in chapter 3.
  \end{enumerate}
  \item Let $Y$ be an ordered set in the order topology. Let $f,g: X\rightarrow Y$ be continuous.
  \begin{enumerate}
    \item Show that the set $B=\{x\;|\; f(x)\leq g(x)\}$ is closed in $X$.\hspace{5in}\n
    \indent Define $A=X-B$ For any $x\in A$, for each of its neighborhoods $V_f=f(x)$, $U_f=f^{-1}(V_f)$ is open, and for each of its neighborhoods $V_g=g(x)$, $U_g=g^{-1}(V_g)$ is open. It is sufficient to consider only neighborhoods that are basis elements. Then, since $x\in A$, there exist neighborhoods $V_f$ and $V_g$ such that the smallest element in $V_f$ is greater than the largest element in $V_g$. $U_f$ is an open set containing $x$, and furthermore, since the smallest element in $V_f$ is greater than the largest element in $V_g$, every element in $U_f$ is contained in $A$. The union of open sets is open, so the union of such an open set for every $x$ in $A$ is an open set, this union will yield exactly $A$. $A$ is the complement of $B$, so since $A$ is open, $B$ is closed.
    \item Let $h: X\rightarrow Y$ be the function $h(x)=\min(f(x),g(x))$. Show that $h$ is continuous.\hspace{5in}\n
    \indent Consider $A=\{x\;|\; f(x)\leq g(x)\}$ and $B=\{x\;|\; g(x)\leq f(x)\}$. $A$ and $B$ are closed by the previous part. Clearly, $x\in A\cap B\rightarrow f(x)=g(x)$, by antisymmetry of an order relation. By the \hyperref[thm:PastingLemma]{pasting lemma}, the minimum of these two functions is a continuous function. \n
  \end{enumerate}
  \item Let $\{A_\alpha\}$ be a collection of subsets of $X$; let $X=\bigcup_\alpha A_\alpha$. Let $f: X\rightarrow Y$; suppose that $f|A_\alpha$ is continuous for each $\alpha$. 
  \begin{enumerate}
    \item Show that if the collection $\{A_\alpha\}$ is finite and each set $A_\alpha$ is closed, then $f$ is continuous.\hspace{5in}\n
    \indent The \hyperref[thm:LocalFormulationContinuity]{local formulation of continuity} shows that this would be true if each $A$ were open. Furthermore, by \hyperref[dfn:continuous3]{part 3 of definition of continuity}, just like for open sets, if $f$ is continuous, then for every closed set $B$ of $Y$, the set $f^{-1}(B)$ is closed in $X$. Consider such a closed set $B$ of $Y$. $f^{-1}(B)\cap A_\alpha = (f|A_\alpha)^{-1}(B)$: both expressions are exactly those points $x$ in $A_\alpha$ such that $f(a)\in B$. Since $f|A_\alpha$ is continuous, $(f|A_\alpha)^{-1}(B)$ is closed. $f^{-1}(B) = \bigcup_\alpha(f|A_\alpha)^{-1}(B)$. The finite union of closed sets is closed, so $f^{-1}(B)$ is closed, so $f$ is continuous.
    \item Find an example where the collection $\{A_\alpha\}$ is countable and each $A_\alpha$ is closed, but $f$ is not continuous.\hspace{5in}\n
    \indent Consider 
    $$A_{n>0}=\left\{\frac{1}{n}\right\}\text{, and } f(x)=\begin{cases} x & \quad x\neq 1 \\ 0 & \quad x=1\end{cases}$$
    Where $f: \bigcup_\alpha A_\alpha \rightarrow \mathbb{R}$. Each $A_\alpha$ is closed in the topology inherited as a subspace of $\mathbb{R}$, and each $f|A_\alpha$ is clearly continuous, as both the identity function and the constant function are continuous. Consider $V=[0,\sfrac{1}{2}]$, a closed set in $\mathbb{R}$. Then, $f^{-1}(B)$ is not closed, since it does not contain $0$. Therefore, $f$ is not continuous.
    \item An indexed family of sets $\{A_\alpha\}$ is said to be $\emph{locally finite}$ if each point $x$ of $X$ has a neighborhood that intersects $A_\alpha$ for only finitely many values of $\alpha$. Show that if the family $\{A_\alpha\}$ is locally finite and each $A_\alpha$ is closed, then $f$ is continuous.\hspace{5in}\n
    %\indent The proof proceeds exactly as the proof of part a. The requirement of local finiteness exists because of the last step - the \emph{finite} union of closed sets is closed. To get $f^{-1}(B)$ there is no need to take the union over all $\alpha$, but rather only finitely many $\alpha$. \n
    %In detail: Consider a closed set $V$ of $Y$. $f^{-1}(V)\cap A_\alpha = (f|A_\alpha)^{-1}(V)$. Each $f|A_\alpha$ is continuous, therefore each $(f|A_\alpha)^{-1}(V)$ is closed. Consider any point $x$ such that $f(x)\in V$. Each such $x$ has a neighborhood $U$ that intersects $A_\alpha$ for finitely many values of alpha, call the intersected family of sets $\{B_\beta\}$.
    \indent Consider a point $x$ in $X$, which by hypothesis has a neighborhood $U_x$ which intersects a finite number of $A_\alpha$s, collect all such $A_\alpha$s in an indexed family of sets $B_\beta$. Each $f|B_\beta$ is continuous and each $B_\beta$ is closed, so $f|\bigcup_\beta B_\beta$ is continuous by repeated application of \hyperref[thm:PastingLemma]{the pasting lemma}. $U_x \subset \bigcup_\beta B_\beta$, so restricting the domain to $U_x$ gives a continuous function $f|U_x$. Such a function exists for each $x$, and $X$ is the union of such open sets $U_x$, so by the \hyperref[thm:LocalFormulationContinuity]{local formulation of continuity} $f$ is continuous.
  \end{enumerate}
  \item Let $f: A \rightarrow B$ and $g: C \rightarrow D$ be continuous functions. Let us define a map $f\times g: A\times C \rightarrow B\times D$ by the equation $$(f\times g)(a\times c) = f(a) \times g(c).$$ Show that $f\times g$ is continuous.\hspace{5in}\n
  \indent Define $X = A\times C$. Then define $f_1: X\rightarrow B$ as $f_1(a \times c) = f(a)$ and $g_1: X\rightarrow D$ as $g_1(a \times c) = g(c)$.
  $f_1$ and $g_1$ are continuous, so by \hyperref[thm:MapsProducts18.4]{theorem 18.4} $f\times g$ is continuous.
  \item Let $F:X\times Y \rightarrow Z$. We say that $F$ is \emph{continuous in each variable separately} if for each $y_0$ in $Y$, the map $h: X\rightarrow Z$ defined by $h(x)=F(x\times y_0)$ is continuous, and for each $x_0$ in $X$, the map $k:Y\rightarrow Z$ defined by $k(y) = F(x_0 \times y)$ is continuous. Show that if $F$ is continuous, then $F$ is continuous in each variable separately. \hspace{5in} \n
  \indent $F$ is continuous, therefore for each open subset $V$ of $Z$ the set $F^{-1}(V)$ is open. For any $y_0\in Y$, define the set $H_0 = \{X\times y_0\}$. The restricted function $F|H_0: H_0 \rightarrow Z$ is continuous, and equivalent to the function $h$.
  \item Let $F: \mathbb{R} \times \mathbb{R} \rightarrow \mathbb{R}$ be defined by the equation
    $$F(x\times y) = \begin{cases}
      xy/(x^2+y^2) & \quad x\times y \neq 0\times 0 \\
      0            & \quad x\times y = 0\times 0
    \end{cases}$$
    \begin{enumerate}
      \item Show that $F$ is continuous in each variable separately. \hspace{5in} \n
      \indent Setting $y$ to an arbitrary constant $c$ gets the rational function $h(x) = c*x/(x^2+c^2)$, and since the denominator is always positive this function is continuous. If $c=0$ then the function is simply $0$ at all points, so it is continuous. 
      \item Compute the function $g: \mathbb{R} \rightarrow \mathbb{R}$ defined by $g(x) = F(x\times x)$\hspace{5in}\n
      $$g(x) = \begin{cases} 1/2 & \quad x\neq 0 \\ 0 & \quad x=0 \end{cases}$$
      \item Show that $F$ is not continuous.\hspace{5in}\n
      \indent The previous part shows a discontinuity when $x=y$ and both approach 0 together, the function jumps from $1/2$ to $0$.
    \end{enumerate}
    \item Let $A\subset X$; let $f: A\rightarrow Y$ be continuous; let $Y$ be Hausdorff. Show that if $f$ may be extended to a continuous function $g: \bar{A}\rightarrow Y$, then $g$ is uniquely determined by $f$. \hspace{5in}\n
    \indent Suppose that $g$ is not uniquely determined by $f$, there is another such extension $h$. Consider a point $x$ that is in $\bar{A}$ but not $A$. Suppose $V$ is a neighborhood of $g(x)$; since $g(x)$ is continuous, $U=g^{-1}(V)$ is an open set containing $x$. Let $V_2$ be a neighborhood of $h(x)$ disjoint from $V$. Then, $U_2=h^{-1}(V_2)$ is a neighborhood of $x$. $x$ is a limit point of $A$, thus there exists points $y\in U$ and $y_2\in U_2$ that must both be in $A$, and $g(y)=h(y)=f(y), g(y_2)=h(y_2)=f(y_2)$. %$V$ contains infinitely many points of $g(\bar{A})$, and $W$ contains infinitely many points of $h(\bar{A})$, since \hyperref[thm:LimitPointT1]{all neighborhoods of a limit point contain infinitely many points of the set}.
    % Any finite point set in $Y$ is closed, in particular $g(x)$ is closed, so the set $V\cap (Y-g(x))$ is open, as is $V_2 \cap (Y-h(x))$.
    We now consider two cases - where $U \cap U_2 = \{x\}$ and where there is at least one other point of intersection.
    \begin{itemize}
      \item [Case 1:] There exist disjoint neighborhoods $V$ and $V_2$ that yield a $U$ and $U_2$ such that $U \cap U_2 = \{x\}$. Thus $\{x\}$ is an open set which contains no elements of $A$, which contradicts the fact that $x$ is a limit point.
      \item [Case 2:] $U \cap U_2$ contains some $y\in A$. $g(y)=h(y)=f(y)$. Thus, $f(y)$ was in both $V_1$ and $V_2$, a contradiction, as those were disjoint.
    \end{itemize}
    Thus, $g(x) = h(x)$ for all $x$.
    $\qed$
\end{enumerate}
\end{document}
