\documentclass[12pt,letterpaper]{article}
\usepackage[pdftex]{graphicx}
\usepackage{alltt}
\usepackage[margin=1in]{geometry}
\usepackage{amsmath, amsthm, amssymb}
\usepackage{verbatim}
\newcommand{\degree}{\ensuremath{^\circ}}
\newcommand{\n}{\break}
\let\oldemptyset\emptyset
\let\emptyset\varnothing
\newcommand{\Wlog}{without loss of generality}
\newcommand{\WLOG}{Without loss of generality}

\begin{document}
\raggedright
\begin{enumerate}
  \setcounter{enumi}{2}
  \item Let $X$ be the set $\{0,1\}$. Show that there is a bijective correspondense between $\mathcal{P}(\mathbb{Z}_+)$ and the cartesian product $X^\omega$ \n
  \indent Let $x=(x_1, x_2, x_3, ...)\in X$ where each $x_i\in \{0,1\}$. \n
  Now define the map
  $$f : X^\omega \rightarrow \mathcal{P}(\mathbb{Z}_+)$$
  to be
  $$f(x) = \{i\, |\, x_i=1\}$$
  $f$ is injective - consider $x_a, x_b$ such that $f(x_a)=f(x_b)$.
  Then for each $i$, $x_{ai}=x_{bi}$, so $x_a=x_b$ \n
  Furthermore, $f$ is surjective. For each $n\in\mathbb{Z}_+$, each subset of $\mathbb{Z}_+$ either contains $n$ or does not. $X$ contains every sequence where $x_n$ is 1 ($n$ is in the subset) and all those where $x_n$ is 0 ($n$ is not in the subset).
  \addtocounter{enumi}{1}
  \begin{enumerate}
    \item (also part b) Countablely infinite, there's a bijective mapping to $\mathbb{Z}_+^n$.
    The set $B_n$ can be bijectively mapped to the set of subsets of $\mathbb{Z}_+$ of size $n$.
    \addtocounter{enumii}{1}
    \item Uncountable, equivalent to $\mathcal{P}(\mathbb{Z}_+)$.
    \item (also part e) Uncountable, assigning a 0 or 1 to each element in $\mathbb{Z}_+$ was shown to be equivalent to $X^\omega$ in 3), which is uncountably infinite.
    \addtocounter{enumii}{1}
    \item Countable, there is one such function for every choice of whether $n$ is 0 or 1 for all $n$ up to $N$, thus there are $2^N$ such functions.
    \item Countable infinite, it's simply the number of subsets of $\mathbb{Z}_+$ of size N.
    \item A countable union of the above, thus it is countably infinite.
  \end{enumerate}
\end{enumerate}
\end{document}
