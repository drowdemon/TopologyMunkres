\documentclass[12pt,letterpaper]{article}
\usepackage[pdftex]{graphicx}
\usepackage{alltt}
\usepackage[margin=1in]{geometry}
\usepackage{amsmath, amsthm, amssymb}
\usepackage{verbatim}
\usepackage{ragged2e}
\usepackage{enumitem}
\usepackage{xfrac}
\setlist{parsep=0pt,listparindent=\parindent}
\setlength{\RaggedRightParindent}{\parindent}
\newcommand{\degree}{\ensuremath{^\circ}}
\newcommand{\n}{\break}
\let\oldemptyset\emptyset
\let\emptyset\varnothing
\newcommand{\Wlog}{without loss of generality}
\newcommand{\WLOG}{Without loss of generality}
\usepackage{accents}
\let\thinbar\bar
\newcommand\thickbar[1]{\accentset{\rule{.4em}{.8pt}}{#1}}
\let\bar\thickbar
\usepackage{standalone}
\usepackage{hyperref}
\newcommand{\R}{\ensuremath{\mathbb{R}}}
\usepackage{mathtools}
\DeclarePairedDelimiter{\ceil}{\lceil}{\rceil}
\DeclarePairedDelimiter{\floor}{\lfloor}{\rfloor}
\DeclarePairedDelimiter\abs{\lvert}{\rvert}
\DeclarePairedDelimiter\norm{\lVert}{\rVert}
%%%%%%%%%%%%%%%%%%%%%%%%%%%%%%%%%%%%%%%%%%%%%%%%%%%%%
%TOPOLOGY DOCUMENTS ONLY%
\newcommand{\T}{\ensuremath{\mathcal{T}}}
%%%%%%%%%%%%%%%%%%%%%%%%%%%%%%%%%%%%%%%%%%%%%%%%%%%%%

\begin{document}
\RaggedRight
\begin{enumerate}
  \item Let $X$ be a metric space with metric $d$, and let $A$ be a subset of $X$. Show that $d|A\times A$ is a metric for the subspace topology on $A$.\\
  Let $U$ be a basis element of $A$ in the subspace topology that contains $x$, $U$ is of the form $U=B_d(x,\epsilon) \cap A$. There exists some $\delta$ such that $B_{d|A\times A}(y,\delta)\subset U$, where $y\in U$, for instance, $\delta = \epsilon - d(x,y)$. Therefore, the topology created by the restricted metric is finer than the subspace topology.\\
  In the other direction, given an $x\in A$ and $U=B_{d|A\times A}(x,\epsilon)$, $U$ is equivalent to $B_d(x, \epsilon)\cap A$, so the subspace topology is finer than the metric topology. Therefore, the two topologies are equal.
  \item Let $X$ and $Y$ be metric spaces with metrics $d_x$ and $d_Y$. Let $f:X\rightarrow Y$ have the property that for every pair of points $x_1,x_2\in X$, $d_Y(f(x_1),f(x_2))=d_X(x_1,x_2)$. Show that $f$ is an embedding. $f$ is called an \emph{isometric embedding} of $X$ in $Y$. \\
  First we show that $f$ is injective. Suppose the opposite, there exist distinct $x_1,x_2$ such that $f(x_1)=f(x_2)$. But then, by the properties of metrics, $d_Y(f(x_1),f(x_2)) = 0$, whereas $d_X(x_1, x_2)\neq 0$. Therefore $f$ is injective. \\
  Now we show that $f$ is continuous.  Suppose the opposite, there exists an open basis element $V\subset Y$, $V=B_{d_Y}(y,\epsilon)$, such that $f^{-1}(V)$ is not open in $X$, where $y\in Y$. Let $x=f^{-1}(y)$, $x$ is a single element since $f$ is injective. For every point $y'$ in $V$,  $d_Y(y,y')=d_X(x,f^{-1}(y'))<\epsilon$, and so $B_{d_X}(x,\epsilon)$ is open in $X$. Therefore, $f$ is continuous. \\
  Let $Z=f(X)$ and let $f': X\rightarrow Z$ be obtained by restrincting the range of $f$ to $Z$. $f'$ is a bijection. We show that $f$ is an embedding, that is, that $f'$ is a homeorphism, or equivalently, that $f'^{-1}$ is continuous. Assume the opposite, there exists some basis element $U\in X$, $U=B_{d_X}(x,\epsilon)$ such that $f(U)$ is not open in $Y$. The argument that showed that $f$ is continuous also shows that this claim is true, therefore $f'^{-1}$ is continuous, $f'$ is homeomorphic, and $f$ is an embedding of $X$ in $Y$.
  \item Let $X_n$ be a metric space with metric $d_n$, for $n\in \mathbb{Z}_+$.
  \begin{enumerate}
    \item Show that $p(x,y) = \max\{d_1(x_1,y_1),\dots,d_n(x_n,y_n)\}$ is a metric for the product space $X_1\times\dots\times X_n$. \\
    Let $B=U_1\times\dots\times U_n$ be a basis element of the given product space, where each $U_i=B_{d_i}(x_i,\epsilon_i)$ is a basis element of $X_i$, with $x\in X_i$. Then, the basis element $B_p((x_1,\dots,x_n), min_i(\epsilon_i))$ lies within $B$, therefore the metric for the product space is finer than the product topology on that space. \\
    In the other direction, let $x=(x_1,\dots,x_n)$, and $V=B_p(x, \epsilon)$ be a basis element of the metric topology. Then, $B_{d_1}(x_1,\epsilon)\times\dots\times B_{d_n}(x_n, \epsilon)$ lies within $V$. Therefore the product topology is finer than the metric topology, and so the two topologies are equivalent. 
    \item Let $d_i=\min\{d_i,1\}$. Show that $D(x,y)=\sup\{\bar{d}_i(x_i,y_i)/i\}$ is a metric for the product space $\prod(X_i)$ \\
    The first two properties of a metric are satisfied trivially. The triangle inequality is proved as follows. $\bar{d}(x_i,z_i)/i\leq \bar{d}(x_i,y_i)/i + \bar{d}(y_i,z_i)/i \leq D(x,y)+D(y,z)$. Therefore, $D(x,z) = \sup\{ \bar{d}(x_i,z_i)/i\} \leq D(x,y)+D(y,z)$. \\
    Now we show that $D$ gives the product topology $\prod(X_i)$. Let $B=B_D(x,\epsilon)$ be a basis element of the metric topology. Let $N$ be large enough that $1/N < \epsilon$, then, the basis element of the product topology $B_{d_1}(x_1,\epsilon)\times\dots\times B_{d_N}(x_N,\epsilon)\times X_{N+1}\times\cdots$ lies within $B$. For the first $N$ elements, each one is no more than $\max(\epsilon,1)$ away from $x$, and for the rest, $\sup\{\bar{d}(x,y)\}=1$, so $\bar{d}(x,y)/i < 1/N < \epsilon$, since $i>N$. Therefore, overall the metric topology is finer than the product topology. \\
    In the other direction, Let $U=\prod U_i$ be a basis element of the product topology, where $U_i=B_{d_i}(x_i,\epsilon_i)$ is a basis element of $X_i$ and $x_i\in X_i$ for $i=\alpha_1,\dots,\alpha_n$ and $U_i=X_i$ for all other $i\in \mathbb{Z}_+$. Then let $\epsilon = min(\epsilon/i)$ for $i=\alpha_i,\dots,\alpha_n$. $x\in B_D(x,\epsilon)\subset U$. Thus, the two topologies are equivalent.
  \end{enumerate}
  \item Show that \hyperref[dfn:lowerLimitTopology]{$\R_\ell$} and the \hyperref[dfn:orderedSquare]{ordered square} satisfy the first countability axiom. (This result does not, of course, imply that they are metrizable.) \\
  For each element $x\in\R_\ell$ of the lower limit topology, consider the sequence $U_i=[x,x+1/i)$, where each $U_i$ is a basis element of $\R_\ell$. The countable collection of such $U_i$ for all $i\in\mathbb{Z}_+$ forms a countable basis for $x$; any neighborhood of $x$ contains a $U_i$ for at least one $i$. Therefore $\R_\ell$ satisfies the first countability axiom. \\
  For each element $(x\times y)\in I_o^2$, the ordered square, let $U_i=(x \times y - \frac{y}{2i}, x\times y + \frac{1-y}{2i})$. $y-\frac{y}{2i}>0$ and $y+\frac{1-y}{2i} < 1$, and by the same arguments as above, $I_o^2$ satisfies the first countability axiom.
  \item{} [prove] Theorem. Let $x_n\rightarrow x$ and $y_n \rightarrow y$ in the space $\mathbb{R}$. Then, $x_n+y_n \rightarrow x+y,\; x_n-y_n\rightarrow x-y,\; x_ny_n\rightarrow xy$ and provided each $y_n\neq 0$ and $y\neq 0$, $x_n/y_n\rightarrow x/y$. [Hint: apply lemma 21.4, recall from the exercises of section 19 that if $x_n\rightarrow x$ and $y_n\rightarrow y$, then $x_n\times y_n \rightarrow x\times y.$]
  \item Define $f_n : [0,1]\rightarrow \R$ by the equation $f_n(x)=x^n$. Show that the sequence $(f_n(x))$ converges for each $x\in [0,1]$, but that the sequence $(f_n)$ does not converge uniformly. \\
  Calculus shows that for each such $x$, the sequence $(f_n(x))$ converges to 0, except $x=1$ where it converges to 1. Now, suppose that the sequence does converge uniformly; then there is a function $f:X\rightarrow Y$ such that given $\epsilon>0$ there exists an integer $N$ such that $d(f_n(x),f(x))<\epsilon$ for all $n>N$ and for all $x\in X$. Therefore, for all $n>N$, $d(x^n,f(x))<\epsilon$. For all $0\leq x<1$, as $n$ increases $x^n$ decreases to $0$, thus $f(x)$ must be less than $\epsilon$ for the condition to hold. However, for $x=1$ $f(x)$ would have to be 1, so the condition fails for any $\epsilon < 1/2$, so $(f_n)$ does not converge uniformly.
  \item Let $X$ be a set, and let $f_n:X\rightarrow \R$ be a sequence of functions. Let $\bar{p}$ be the uniform metric on the space $\R^X$. [Note: $\R^X$ is the set of all functions mapping $X$ to $\R$.] Show that the sequence $(f_n)$ converges uniformly to the function $f: X\rightarrow \R$ if and only if the sequence $(f_n)$ converges to $f$ as elements of the metric space $(R^X,\bar{p})$.
  \item Let $X$ be a topological space and let $Y$ be a metric space. Let $f_n: X\rightarrow Y$ be a sequence of continuous functions. Let $x_n$ be a sequence of points of $X$ converging to $x$. Show that if the sequence $(f_n)$ converges uniformly to $f$, then $(f_n(x_n))$ converges to $f(x)$. \\
  %We are given the fact that $d(f_n(x_i), f(x_i)) < \epsilon$ for all $n > N_i$, for each $i$.
  %Let $V_\epsilon=B(f(x), \epsilon)$ be a basis element of $Y$. Each $U_\epsilon = f^{-1}(V_\epsilon)$ is a neighborhood of $x$ in $X$, since each element of $(f_n)$ is continuous, so $f$ is continuous. Let $M_\epsilon$ be the value such that for all $n>M_\epsilon$, $x_n\in U$, and so for all $n>M_\epsilon$, $d(f(x), f(x_n)) < \epsilon$.\\
  %Suppose $(f_n(x_n))$ does not converge to $f(x)$, that is, there exists a neighborhood $B=B(f(x),\epsilon)$ such that there is no $M$ for which $f_n(x_n)\in B$ for each $n>M$. \\
  %For each $i$, there is an $N_i$ such that $U=d(f_n(x_i),f(x_i))<\epsilon$ for all $n>N_i$. Furthermore, there is an $M$ such that each $x_{i>M}\in U$, or equivalently, $d(f(x_i),f(x)) < \epsilon$. Therefore, combining these, $d(f_n(x_i), f(x))<2\epsilon$ when $n>N_i$ and $i>M$. \\
  Let $U=f^{-1}(B_d(f(x), \epsilon/2))$. $U$ is open since $f$ is continuous, so there is some $N$ for which $x_{n>N}\in U$, and so $d(f(x_{n>N}),f(x))<\epsilon/2$ Since $(f_n)$ converges uniformly, there is an $M$ such that $d(f_n(y),f(y)) < \epsilon/2$, for all $y\in X$ and all $n>M$. Consider those $y$ that are in $U$, and particularly those that happen to be an $x_n$. Then, $d(f_n(x_n),f(x_n))<\epsilon/2$. Using the triangle inequality, $d(f_n(x_n),f(x_n)) + d(f(x_n), f(x)) < d(f_n(x_n), f(x) < \epsilon$, for all $n > \max(M,N)$. Thus, for any neighborhood $B = B_d(f(x), \epsilon)$, for all $n>\max(M,N)$ every element of the sequence $(f_n(x_n))\in B$; so by definition $(f_n(x_n))$ converges to $f(x)$.
  \item Let $f_n: \R \rightarrow \R$ be the function $$f_n(x) = \frac{1}{n^3(x-\sfrac{1}{n})^2+1}$$ Let $f: \R \rightarrow \R$ be the zero function.
  \begin{enumerate}
    \item Show that $f_n(x)\rightarrow f(x)=0$ for each $x\in R$.\\
    If $x>0$, then for some $N$, $(x-1/n)$ will be greater than $0$ for all $n>N$, so $f_n(x)>0$ for all such $n$. Furthermore, for all such $n$, $1/n > x-1/n$, if $x<1$, so $1/(n^3/n^2+1) = 1/n > f_n(x)$. $(1/n)_{n>N}$ converges to $0$, so $f_n(x)$ converges to $0$. If $x\geq 1$, it simly increases the denominator, which decreases the value of the function further, so the logic holds. \\
    If $x\leq 0$, then $(x-1/n)^2>0$ anyway, the same logic shows that it converges to $0$.
    \item Show that $f_n$ does not converge uniformly to $f$. (This shows that the converse of theorem 21.6 does not hold; the limit function $f$ may be continuous even though the convergence is not uniform.) \\
    Suppose  the sequence does converge uniformly to $f$; for any $\epsilon$ there is an $N$ such that for all $x\in\R$, $f_n(x)<\epsilon$. However, as the figure shows, when $x=1/n$, $f_n(x)=1$ for any $n$. Thus if $\epsilon<1$, there can be no such $N$, so $f_n$ does not converge uniformly.
  \end{enumerate}
  \item Using the \hyperref[dfn:continuous3]{closed set formulation of continuity}, (Theorem 18.1), show that the following are closed subsets of $\R^2$: $A = \{ x\times y\; |\; xy=1\}$; $S^1 = \{ x\times y\;|\; x^2+y^2=1\}$; and $B^2=\{x\times y\;|\;x^2+y^2\leq 1$. The set $B^2$ is the closed unit ball in $\R^2$. \\
  % Define $f(x) = 1/x$, where $f:\R - \{0\} \rightarrow \R - {0}$ then $A$ is equivalent to the set $x\times f(x)$ for all $x\in \R - \{0\}$, so define $h(x) = i(x) \times f(x)$ where $i$ is the identity function. $i$ and $f$ are continuous, so $f$ is continuous, and so $h: \R-\{0\}\rightarrow A$ is continuous.
  Let $\pi_1: A \rightarrow \R - {0}$ be defined $\pi_1(x\times y) = x$. This function is continuous, and it is bijective, it's inverse is the continuous function $f(x) = x\times 1/x$. $\pi_1^{-1}(\R-\{0\}) = A$, $\R - \{0\}$ is closed, since $\pi_1$ is a homeomorphism it preserves topological properties, so $A$ is closed. \\
  \item Skipping this too, for now
  \item Prove continuity of the algebraic operations on $\R$ as follows: use the metric $d(a,b)=\abs{a-b}$ on $\R$ and the square metric $p((x\times y),(x_0\times y_0)) = \max(\abs{x-x_0}, \abs{y-y_0})$ on $\R^2$.
  \begin{enumerate}
    \item Show that addition is continuous. [Hint: Given $\epsilon$, let $\delta = \epsilon/2$ and note that $d(x+y, x_0+y_0) \leq \abs{x-x_0} + \abs{y-y_0}.$]\\
    Let addition be encapsulated in the function $f(x,y) = x+y$, which maps $\R^2$ to $\R$. \\
    In $\R\times\R$, a point $x\times y$ represents two numbers which are to be added up to $x+y$. Consider an $\epsilon$-ball around this point $x\times y$, $B=B_p(x\times y, \epsilon/2)$. Let $x_0\times y_0$ be a point in $B$. Then, consider the distance $d(f(x\times y), f(x_0\times y_0)) = d(x+y, x_0+y_0) = \abs{x+y-x_0-y_0} \leq \abs{x-x_0}+\abs{y-y_0}$ (Proof of the inequality proceeds by case study of the components $x-x_0$ and $y-y_0$. If both are positive or both are negative the LHS and RHS are equal, if the have opposite signs then the RHS increases while the LHS decreases.) We have that $\max(\abs{x-x_0},\abs{y-y_0})<\epsilon/2$, therefore $d(x+y, x_0+y_0)<2*\max(\abs{x-x_0},\abs{y-y_0})<\epsilon$. \\
    Let $z\in \R$, and let $U=B_d(z,\epsilon)$, and $V=f^{-1}(U)$.  Then for each point $x\times y$ in $f^{-1}(z)$, $B$ is an open element of $\R\times\R$ that contains $x\times y$. Each point in $B$ represents a point whose sum is no more than $\epsilon$ away $f(x\times y)=z$, and therefore is within $U$. The union of all such sets $B$ is exactly $U$, and the arbitrary union of basis elements is open, so for every open $U\subset \R$, the preimage is open, so $f$ is continuous.
    \item Show that multiplication is continuous. [Hint: Given $(x_0,y_0)$ and $0<\epsilon<1$, let $3\delta = \epsilon/(|x_0|+|y_0|+1)$ and note that $d(xy,x_0y_0)\leq \abs{x_0}\abs{y-y_o} + \abs{y_0}\abs{x-x_0} + \abs{x-x_0}\abs{y-y_0}$]. \\
    Let multiplication be encapsulated in the function $f(x,y) = xy$, which maps $\R^2$ to $\R$. Let $z\in \R$, $U=B_d(z,\epsilon)$, and $V=f^{-1}(U)$. Then for each point $x\times y$ in $f^{-1}(z)$, we wish to construct a neighborhood of that point which contains every point whose image under $f$ (product of its coordinates) is less than $\epsilon$ away from $z$. \\
    To do this, consider $B=B_p(x\times y, \delta)$, and $x_0\times y_0\in B$. $d(f(x\times y), f(x_0\times y_0)) = \abs{xy-x_0y_0}  \leq \abs{x_0}\abs{y-y_0} + \abs{y_0}\abs{x-x_0} + \abs{x-x_0}\abs{y-y_0}$. Also, $\max(\abs{x-x_0},\abs{y-y_0})<\delta$. Therefore, $d(f(x\times y),f(x_0\times y_0)) < (\abs{x_0}+\abs{y_0})\delta+\delta^2$
    \item Show that the operation of taking reciprocals is a continuous map from $\R-\{0\}$ to $\R$. [Hint: Show the inverse image of the interval $(a,b)$ is open. Consider five cases, according as $a$ and $b$ are positive, negative, or zero.]
    \item Show that the subtraction and quotient operations are continuous.
  \end{enumerate}
\end{enumerate}
\end{document}