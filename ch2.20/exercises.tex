\documentclass[12pt,letterpaper]{article}
\usepackage[pdftex]{graphicx}
\usepackage{alltt}
\usepackage[margin=1in]{geometry}
\usepackage{amsmath, amsthm, amssymb}
\usepackage{verbatim}
\usepackage{ragged2e}
\usepackage{enumitem}
\usepackage{xfrac}
\setlist{parsep=0pt,listparindent=\parindent}
\setlength{\RaggedRightParindent}{\parindent}
\newcommand{\degree}{\ensuremath{^\circ}}
\newcommand{\n}{\break}
\let\oldemptyset\emptyset
\let\emptyset\varnothing
\newcommand{\Wlog}{without loss of generality}
\newcommand{\WLOG}{Without loss of generality}
\usepackage{accents}
\let\thinbar\bar
\newcommand\thickbar[1]{\accentset{\rule{.4em}{.8pt}}{#1}}
\let\bar\thickbar
\usepackage{standalone}
\usepackage{hyperref}
\newcommand{\R}{\ensuremath{\mathbb{R}}}
\usepackage{mathtools}
\DeclarePairedDelimiter{\ceil}{\lceil}{\rceil}
\DeclarePairedDelimiter{\floor}{\lfloor}{\rfloor}
\DeclarePairedDelimiter\abs{\lvert}{\rvert}
\DeclarePairedDelimiter\norm{\lVert}{\rVert}
%%%%%%%%%%%%%%%%%%%%%%%%%%%%%%%%%%%%%%%%%%%%%%%%%%%%%
%TOPOLOGY DOCUMENTS ONLY%
\newcommand{\T}{\ensuremath{\mathcal{T}}}
%%%%%%%%%%%%%%%%%%%%%%%%%%%%%%%%%%%%%%%%%%%%%%%%%%%%%

\begin{document}
\RaggedRight
\begin{enumerate}
  \item \begin{enumerate}
    \item In $\R^n$, define $d'(x,y)=|x_1-y_1|+\dots +|x_n-y_n|$. Show that $d'$ is a metric that induces the usual topology of $\R^n$. Describe the basis elements under $d'$ when $n=2$.\hspace{5in}\n
    \indent It is trivial to see that the first two conditions of a metric are satisfied by $d'$. The third can be seen as follows. We assert that $|x_1-y_1|+|y_1-z_1|+\dots + |x_n-y_n|+|y_n-z_n| \geq |x_1-z_1|+\dots + |x_n-z_n|$. For each $i$ between $1$ and $n$, $|x_i-y_i| + |y_i-z_i| \geq |x_i-z_i|$, which is true - it is the triangle inequality on $\R$. Thus it is true for the sum.\hspace{5in}\n
    \indent The basis elements in $\R^2$ are squares with vertices at $(0,\epsilon), (\epsilon, 0), (0,-\epsilon), (-\epsilon,0)$. \hspace{5in}\n
    \indent Consider an element $B=(a_1,b_1)\times\dots (a_n,b_n)$ of the basis for the standard topology on $\R^n$, and let $x$ be an element of $B$. For each $i$ between $1$ and $n$, let $\epsilon_i=\min(|x_i-b_i|, |x_i-a_i|)$. Then, let $\epsilon = \min_i(\epsilon_i)$. The $\epsilon$-ball $b_{d'}(x,\epsilon)$ contains $x$ and lies within $B$. Thus, the topology induced by $d'$ is finer than the usual topology. \hspace{5in}\n
    \indent In the other direction, consider an $\epsilon$-ball, $B=B_{d'}(x,\epsilon)$, for some $x\in \R^n$ and $\epsilon \in \R$. A basis element $(x_1-\epsilon/4, x_1+\epsilon/4)\times\dots \times (x_n-\epsilon/4, x_n+\epsilon/4)$ will contain $x$, and will lie within $B$. Therefore the usual topology is finer than the topology induced by the metric, and together with the previous part this shows that they are the same topology.
    \item More generally, given $p\geq 1$, define $$d'(x,y)=\left(\sum_{i=1}^n|x_i-y_i|^p\right)^{1/p}$$ for $x,y\in \R^n$. Assume that $d'$ is a metric. Show that it induces the usual topology on $\R^n$.\hspace{5in}\n
    \indent Consider an element $B=(a_1,b_1)\times\dots (a_n,b_n)$ of the basis for the standard topology on $\R^n$, and let $x$ be an element of $B$. For each $i$ between $1$ and $n$, let $\epsilon_i=\min(|x_i-b_i|,|x_i-a_i|)$, and let $\epsilon = \left(\displaystyle\sum_{i=1}^{n}\epsilon_i^p\right)^{1/p}$. Then, $B_{d'}(x,\epsilon)$ lies within $B$ and contains $x$, thus the induced metric topology is finer than the usual topology. \hspace{5in}\n
    \indent In the other direction, the proof proceeds exactly as in the previous problem, because the $\epsilon$-ball for larger $p$ contains the ones for smaller $p$.
  \end{enumerate}
  \item Show that $\R \times \R$ in the dictionary order topology is metrizable.\hspace{5in}\n
  \indent Let $d$ be the square metric on $\R^2$, and let $\bar{d}$ be the standard bounded metric on $d$. Consider the metric $d'((x_1\times y_1), (x_2\times y_2))= \begin{cases} 1 \quad & x_1\neq x_2 \\ \bar{d}((x_1\times y_1),(x_2\times y_2)) \quad & x_1=x_2\end{cases}$. For $\epsilon<1$, an $\epsilon$ ball with this metric will be a vertical line, exactly the basis of the dictionary order metric.
  \item Let $X$ be a metric space with metric $d$. \begin{enumerate}
    \item Show that $d:X\times X \rightarrow \R$ is continuous. \hspace{5in}\n
    \indent Suppose $d$ is not continuous; there is an open basis element $V=(y_1,y_2)$ of $\R$ such that $U=d^{-1}(V)$ is not open. $U$ is the set of points that are a distance between $y_1$ and $y_2$ away from one another. This set could also be expressed as the union of all $B_x$ such that $B_x = B_d(x,y_2) - B_d(x,y_1)$, for any $x\in X$. For each $p\in B_x$, there is a $\delta = \min(d(x,p)-y_1, y_2-d(x,p))$, and $B_d(p,\delta)$ lies in $B_x$, therefore $B_x$ is open by the \hyperref[dfn:metricTopology]{definition of the metric topology}. Since each $B_x$ is open, their union is open, a contradiction, since by hypothesis $U$ is closed. 
    \item Let $X'$ denote a space having the same underlying set as $X$. Show that if \\\noindent$d: X'\times X'\rightarrow \R$ is continuous, then the topology of $X'$ is finer than the topology of $X$.\\
    Let $A$ be the underlying set of both $X$ and $X'$. Suppose $d$ is continuous relative to $X'$. Then for 
  \end{enumerate}
  The result can be summarized as follows: the topology induced by $d$ on $X$ is the coarsest topology relative to which $d$ is continuous.
  \item Consider the product, uniform, and box topologies on $\R^\omega$.
  \begin{enumerate}
    \item In which topologies are the following functions from $\R$ to $\R^\omega$ continuous?
    \begin{itemize}
      \item $f(t) = (t,2t,3t,\dots)$\\
      % Consider an element of the basis of the product topology, $B=V_1\times\dots\times V_n\times\R\times\R\cdots$.
      In summary, $f$ is continuous only in the product topology. \\
      Product topology: Let $f_\alpha : \R \rightarrow \R$ be defined as $f_\alpha(t) = \alpha*t$, then $f$ can also be defined as $f(t) = (f_\alpha(t))_{\alpha\in \mathbb{N}}$. By \hyperref[thm:MapsProducts19.6]{Theorem 19.6}, $f$ is continuous if and only if each $f_\alpha$ is continuous, which they clearly are. \\
      Uniform topology: Let $B$ be a basis element of the uniform topology, $B=B_{\bar{p}}(p,\epsilon)$. Suppose $x\in f^{-1}(B)$ and $y=f(x)$. Then, if $f^{-1}(B)$ is open, there exists some positive real $\delta$ such that $A=(x-\delta, x+\delta)\subset f^{-1}(B)$. Then, $\pi_n(f(A)) = (y_n-n\delta, y_n+n\delta) \subset \pi_n(B)$ Since $n\delta$ increases as $n$ increases, there will be some $n$ for which $n\delta > 2\epsilon = \text{diam } B$, a contradiction, since this means $\bar{d}(y, y+n\delta) > \text{diam } B$, so $(y+n\delta)$ is outside the $\epsilon$-ball. \\
      Box topology: Consider the basis element: $B = (-1, 1) \times (-1/2, 1/2) \times (-1/4, 1/4) \times (-1/8, 1/8) \cdots$ Suppose $f^{-1}(B)$ were open. $0\in f^{-1}(B)$, so for $f^{-1}(B)$ to be open it would have to contain some interval $(-\delta, \delta)$ for some positive real $\delta$. Then $f((-\delta, \delta)) \in B$, so applying $\pi_n$ to each side, $f_n((-\delta,\delta)) = (-n\delta, n\delta) \subset (\sfrac{-1}{2^n}, \sfrac{1}{2^n})$ for all $n$, a contradiction.
      \item $g(t) = (t, t, t, \dots)$ \\
      By the same argument as for $f$, $g$ is continuous in the product topology, and is not continuous in the box topology.
      The proof proceeds similarly as for $f$. If $f^{-1}(B)$ where not open, then for some element $x\in g^{-1}(B)$ there would exist no positive real $\delta$ such that the image of $A = (x-\delta, x+\delta)$ under $g$ would be contained within $B$. Let $y=g(x)$. Then, $\pi_n(g(A)) = (y_n - \delta, y_n+\delta)$. $\delta$ has an upper bound, but no lower bound, so it is possible to choose a $\delta$ that arbitrarily small, including one such that $\bar{d}(y_n, y_n-\delta)=\bar{d}(y_n, y_n+\delta)=\delta < \epsilon-\bar{d}(p, y_n)$. For such a $\delta$, $g(A)\in B$, a contradiction. Thus, $g$ is continuous in the uniform topology.
      \item $h(t) = (t, t/2, t/3, \dots)$ \\
      By the same argument as for $g$, $h$ is continuous in the product and uniform topologies, and is not continuous in the box topology. 
    \end{itemize}
    \item In which topologies do the following sequences converge? \\
    If a sequence converges in a topology $\T$ which is coarser than topology $\T'$, then it clearly converges in $\T'$.
    \begin{itemize}
      \item $w_1 = (1,1,1,\dots), w_2=(0,2,2,2,\dots), w_3=(0,0,3,3,3,\dots)$ \\
      This sequence converges only in the product topology. Let $B=B_D(x, \epsilon)$. Suppose $x=(0,0,0,\dots)$. The distance between $x$ and an element of $w$ is $d_n = D(x,w_n) = 1/n$, so every $\epsilon$-ball around $0$ will eventually contain all $w$. More simply, for every open set of the product topology, there will eventually come a point where all coordinates are in the set.
      \item $x_1 = (1,1,1,\dots), x_2=(0,1/2,1/2,1/2,\dots), x_3=(0,0,1/3,1/3,1/3,\dots)$\\
      % This sequence converges in the box, product, and uniform topologies to $(0,0,0,\dots)$.
      This sequence does not converge in the box topology: consider the open set $U=(-1,1)\times (-1/2,1/2)\times (-1/4,1/4)\times (-1/8,1/8)\cdots$. There is no such $N$ such that every $x_{n>N}\in U$. However, in the uniform topology, it does converge - for an $\epsilon$-ball around $(0,0,0,\dots)$, define $N$ such that $1/N<\epsilon$. Then, every $w_{n>N}$ will lie within the given $\epsilon$-ball.
      \item $y_1 = (1,0,0,0,\dots), y_2=(1/2,1/2,0,0,0,\dots), y_3=(1/3,1/3,1/3,0,0,\dots)$. \\
      The argument proceeds identically to the one for $x$. Thus, $y$ converges in the uniform and product topologies only.
      \item $z_1 = (1,1,0,0,\dots), z_2=(1/2,1/2,0,0,0,\dots), z_3=(1/3,1/3,0,0,0,\dots)$. \\
      This converges in each topology to $(0,0,0,\dots)$. Every open set $U$ in the box topology must contain an interval around $0$ for each coordinate, and there will be a $z$ within that interval.
    \end{itemize}
  \end{enumerate}
  \item Let $\R^\infty$ be the subset of $\R^\omega$ consisting of all sequences that are eventually zero. What is the closure of $\R^\infty$ in $\R^\omega$ in the uniform topology? \\
  We claim that $\bar{\R^\infty}$ is the set of all sequences which converge to $0$.  Suppose $x=(x_1,x_2,\dots)$ is such a point, and let $B=B_{\bar{p}}(x,\epsilon)$ be a basis element of the uniform topology containing $x$. There exists some $n$ such that $\bar{d}(x_{i>n},0)<\epsilon$, and so $x_0,\dots,x_n,0,0,\dots$ is a point in $B$ and in $\R^{\infty}$, which means that $x$ is a limit point of $\R^\infty$.\\
  If $x$ is not a sequence which converges to $0$, then there is an $\epsilon$ such that $\bar{d}(x_i,0)>\epsilon$ for infinitely many $x_i$, and so $x$ is not a limit point of $\R^\infty$. Thus the set of all sequences converging to 0 is the closure of $\R^\infty$ in the uniform topology.
  \item Let $\bar{p}$ be the uniform metric on $\R^\omega$. Given $x = (x_1, x_2, \dots) \in \R^\omega$, and $0<\epsilon < 1$, let $U(x,\epsilon) = (x_1 - \epsilon, x_1+\epsilon) \times \dots \times (x_n-\epsilon, x_n+\epsilon)\times \cdots$
  \begin{enumerate}
    \item Show that $U(x,\epsilon)$ is not equal to the $\epsilon$-ball $B_{\bar{p}}(x,\epsilon)$. \\
    The point $y=(x_1+\epsilon/2 \times x_2+2\epsilon/3 \times x_3+3\epsilon/4\times\cdots$ is in $U(x,\epsilon)$ since each coordinate is less than $x_n+\epsilon$, but it is not in $B_{\bar{p}}(x,\epsilon)$, since the supremum $\sup(\{\bar{d}(x_\alpha, y_\alpha) | \alpha \in \mathbb{N}\})$ is equal to $\epsilon$.
    \item Show that $U(x,\epsilon)$, is not even open in the uniform topology. \\
    There is no $\epsilon$-ball containing $y$ that is in $U(x,\epsilon)$, so $U(x,\epsilon)$ is not a basis element and is not the union of basis elements.
    \item Show that $B_{\bar{p}}(x,\epsilon) = \displaystyle\bigcup_{\delta<\epsilon}U(x,\delta)$. \\
    This definition contains only those elements up to $\epsilon$ away from $x$, because the union will never contain the problematic element $U(x,\epsilon)$, so no element in the union can be a sequence approaching a coordinate $\epsilon$ away from $x$.
  \end{enumerate}
  \item Given sequences $a=(a_1,a_2,\dots)$ and $b=(b_1,b_2,\dots)$ in $\R^{omega}$, define the map $h: \R^{\omega} \rightarrow \R^{\omega}$, by the equation $h((x_1,x_2,\dots)) = (a_1x_1+b_1,a_2x_2+b_2,\dots)$. Give $\R^{\omega}$ the uniform topology. Under what conditions on the numbers $a_i$ and $b_i$ is $h$ continuous? A homeomorphism? \\
  For continuity, the sequence $a$ must be bounded above by some real $N$, and bounded below by $-N$. Consider an element $B=B_{\bar{p}}(p,e)$. If there exists a $B$ such that $h^{-1}(B)$ was not open, then for some element $x\in h^{-1}(B)$ there would exist no positive real $\delta<|\epsilon|$ such that the image of $A=(x-\delta, x+\delta)$ under $h$ would be contained within $B$. (Intuitively, this means that there there is an $x$ that is at the ``edge'' of $B$) Let $y=h(x)$. Then, the set $\pi_n(h(A)) = (y_n - a_n\delta, y_n+a_n\delta)$ must be contained within $\pi_n(B)$. Equivalently, $a_n\delta < \epsilon - \bar{d}(p,y_n)$. If $a_n$ were to increase without bound, then eventually there would be some $a_n$ for which the previous inequality were false. However, if $a_n$ is bounded above by $N$, then one simple chooses a $\delta<\epsilon$ that is also less than $(\epsilon-\bar{p}(p,y))/N$. (If $N=0$, i.e. $a_n=0$ for all $n$, then $h$ always yields $b$, and the preimage of $b$ is simply $\R^\omega$, which is open.) For such a $\delta$, $A$ is open, thus there can be no $B$ such that $h^{-1}(B)$ is not open, so $h$ is continuous.\\
  Bijection is a property of the function, it is independent of the topologies of the function's domain and range, thus the proof of bijection in exercise 8 of section 19 is sufficient. Note that $a_i\neq 0$ is a further requirement in this case. It remains to show that $h^{-1}$ is continuous. $h^{-1}$ is equivalent to $h$ with the sequences $a_n' = 1/a_n$ and $b_n' = b_n/a_n$. Thus the same condition is needed to make $h^{-1}$ continuous: $a'$ must be bounded above and below. Thus, $1/a_n$ must be bounded above and below. Therefore, the limit of $a_n$ must not be $0$. In fact, the sequence cannot approach $0$ in any way, for example $\begin{cases} a_n=1/n \quad & n \text{ is odd} \\ a_n = -1 \quad & n \text{ is even}\end{cases}$ is also unacceptable. This is equivalent to requiring a lower and upper bound on $|a_n|$.
  \item Let $X$ be the subset of $\R^\omega$ consisting of all sequences $x$ such that $\sum x_i^2$ converges. Then the formula $$d(x,y) = \left(\sum_{i=1}^\infty (x_i-y_i)^2\right)^{1/2}$$ defines a metric on $X$ (See exercise 10). On $X$ we have the three topologies it ineherits from the box, uniform, and product topologies on $R^\omega$. We also have the topology given by the metric $d$, which we call the $\ell^2$-topology, read ``little ell two.''
  \begin{enumerate}
    \item Show that on $X$ we have the inclusions: box topology $\supset$ $\ell^2$-topology $\supset$ uniform topology. \\
    For each $x$, show that there is basis element of the little ell two topology containing $x$ that is in turn contained within an element of the uniform topology \\
    Consider an $x\in U$, where $U$ is a basis element of the product topology on $X$ inherited from $\R^\omega$, of the form $(U_1\times\dots\times U_n\times \R\times R\cdots)\cap X$, where $U_1,\dots,U_n$ are open sets in $\R$.
    In a convergent series, a finite number of the elements may be arbitrary - the sum of a finite number of elements is always finite. Therefore, the first $n$ coordinates of $U$ are unchanged by the intersection with $X$, they are just open sets in $\R$. In each one, $x_i$ is contained within an interval $(a_i, b_i)$. Let $\epsilon_i = \min((x_i-a_i)^2, (b_i-x_i)^2)$. Let $\epsilon=(\min_i(\epsilon_i))^{1/2}$. Then $B_d(x,\epsilon)\in U$. Therefore, $\ell^2$-topology $\supset$ product topology. Well damn it I did the wrong problem.\\
    Consider an $x\in U$, where $U$ is a basis element of the uniform topology on $X$ inherited from $\R^w$, of the form $B_{\bar{p}}(p,\epsilon)\cap X$. Let $\epsilon' = \min(\bar{p}(p,x), \epsilon - \bar{p}(p,x))$. Then $V=B_{\bar{p}}(x,\epsilon')\cap X \subset U$. Furthermore, $A=B_d(x, \epsilon') \in V$, since in this case, no single coordinate $y_i$ of an arbitrary point $y\in A$ can be $\epsilon_i$ away from $x$, even if all the other coordinates were equal to the corresponding coordinate of $x$. \hyperref[thm:basisFiner]{Therefore}, the $\ell^2$-topology is finer than the uniform topology. \\
    Now, consider an $x\in U$, where $U=B_d(x,\epsilon)$, a basis element of the $\ell^2$-topology. Consider the geometric series: $\sum_{k=1}^\infty(r^k) = r/(1-r)$. Suppose we set this series equal to $\epsilon$. Algebra gets us the result: $r=\epsilon/(1+\epsilon)$, i.e. $\sum_{k=1}^\infty(\epsilon/(1+\epsilon))^k = \epsilon$ Now consider the infinite sequence $y=((\epsilon/(1+\epsilon))^{1/2}, (\epsilon/(1+\epsilon))^{2/2},\dots (\epsilon/(1+\epsilon))^{n/2},\dots)$. Clearly, the sum of each element squared is $\epsilon$, and so $d((0,0,0,\dots),y)=\sqrt{\epsilon}$. Therefore, the set $V = (x_1-(\epsilon/(1+\epsilon))^{1/2}, (x_1+\epsilon/(1+\epsilon))^{1/2})\times \dots \times(x_n-(\epsilon/(1+\epsilon))^{n/2}, (x_n+\epsilon/(1+\epsilon))^{n/2})\times \cdots$. This is clearly a subset of $U$, because each coordinate is at most an element of a sequence that is $\sqrt{\epsilon}$ away from $x$. $V$ is also clearly an element of the box topology. Hence, by the same lemma as used in the previous paragraph, the box topology is finer than the $\ell^2$-topology.
    \item The set $\R^\infty$ of all sequences that are eventually zero is contained in $X$. Show that the four topologies that $\R^\infty$ inherits as a subspace of $X$ are all distinct. \\
    Consider a basis element $U$ of the topology on $R^\infty$ inherited as a subspace of the product topology, an let $x\in U$. $U=\prod U_i \cap \R^\infty$, where finitely many $U_i$ are not $R$. For the remaining $U_i$, let $(a_i,b_i)$ be a subset of $U_i$, and let $\epsilon = \min_i(\min(\bar{d}(a_i,x), \bar{d}(b_i,x)))$. Then $B_{\bar{p}}(x,\epsilon) \cap R^\infty \subset U$. Therefore the subspace of the uniform topology is finer than the subspace of the product topology. \\
    Now we show that the uniform topology is distinct from the product topology. An open basis element of the product topology on $\R^\infty$ cannot lie within an open basis element of the uniform topology. Consider an aribitrary $\epsilon$-ball, $U=B_{\bar{p}}(x,\epsilon) \cap \R^\infty$, and any basis element $V$ of the product topology on $\R^\infty$ that is a neighborhood of $x$. For infinitely many $n$, $\pi_n(V)$ is a set that contains elements farther than $\epsilon$ from $x_n$, because any one of those coordinates may be non-zero. Thus, the topology on $\R^\infty$ inherited from the uniform topology is \emph{strictly} finer than the one inherited from the product topology. \\
    The proof that the inherited $\ell^2$-topology is finer than the inherited uniform topology is identical to the one in part $A$, the change from $X$ to $\R^\infty$ does not affect the proof. Furthermore, it is strictly finer: in the uniform topology, a basis element $B_{\bar{p}}(x,\delta) \cap R^\infty$ contains a point with any arbitrarily large finite amount of coordinates that are up to $\delta$ away from $x$ using the $\bar{d}$ metric. A point with $\ceil{(\epsilon+1)/\delta}$ such coordinates will be at least $\epsilon$ away from $x$ under the $d$ metric, but still only $\delta$ away under the $\bar{p}$ metric. Therefore there is no basis element in the uniform topology containing $x$ and lying within the basis element of the $\ell^2$-topology, proving that the $\ell^2$-topology is strictly finer.\\
    The box topology is finer than the $\ell^2$-topology - each basis element of the $\ell^2$-topology clearly is an element of the box topology; each coordinate is an open set of $\R$. It is strictly finer: consider an open set $U=((-1,1)\times(-1/2,1/2)\times(-1/4,1/4)\times(-1/8,1/8)\times\cdots)\cap \R^\infty$. Let $x=0\times0\times0\cdots \in U$, and suppose there where a basis element of the $\ell^2$-topology, $V=B_d(x,\epsilon)\cap \R^\infty$ that lies within $U$. But, for any $\epsilon$, there will be a finite $n$ such that the supremum of $\pi_n(U)$ is smaller than $\epsilon$, thus $V$ does not lie inside of $U$.
    \item The set $H=\displaystyle\prod_{n\in\mathbb{Z}_+} [0,\sfrac{1}{n}]$ is contained in $X$; it is called the Hilbert cube. Compare the four topologies that $H$ inherits as a subspace of $X$. \\
    The inherited product topology lies within the inherited uniform topology, the proof is the same as the first paragraph of part b. However, on the Hilbert cube, the product topology also contains the uniform topology; they are the same topology. Let $B=B_{\bar{p}}(x,\epsilon)\cap H$ be a basis element of the uniform topology. Define $n$ such that $1/n<\epsilon$, for example, $n=\ceil{2/(\epsilon)}$. Let $U_{1\leq i \leq n} = (x-\epsilon/2, x+\epsilon/2)$, and let $U_{i>n} = \R$. Finally, let $U=(\prod U_i) \cap H$ be a basis element of the topology inherited by the Hilbert cube as a subspace of the product topology. This $U$ lies within $B$ and contains $x$, because for all $i\leq n$, there is no point farther than $\epsilon$ from $x_i$, and for all $i$ greater than $n$, $\pi_i(H)$ is a set that contains no point farther than $\epsilon$ away from $x_i$. \\
    Now we consider the inherited $\ell^2$-topology. It is finer than the uniform topology, the proof again proceeds the same way as the proof in part a. However, the same technique as was used in the previous paragraph shows that the uniform topology is also finer than the $\ell^2$-topology, they are the same topology in the Hilbert Cube.\\
    The same argument as used in part b shows that the box topology is still strictly finer than the $\ell^2$ topology on the Hilbert cube.
  \end{enumerate}
  \item Show that the euclidean metric $d$ on $\R^n$ is a metric, as follows: if $x,y\in \R^n$ and $c\in \R$, define $x+y=(x_1+y_1,\dots,x_n+y_n)$; $cx = (cx_1, \dots, cx_n)$; and $x\cdot y = x_1y_1 + \dots + x_ny_n$.\begin{enumerate}
    \item Show that $x\cdot (y+z) = (x\cdot y) + (x\cdot z)$
    $$x_1(y_1+z_1)+\dots + x_n(y_n+z_n) = x_1y_1+\dots+x_ny_n + x_1z_1+\dots+x_nz_n$$
    \item Show that $\abs{x\cdot y} \leq \norm{x}\norm{y}$. Hint: If $x,y \neq 0$, let $a=1/\norm{x}$ and $b=1/\norm{y}$, and use the fact that $\norm{ax\pm by} \geq 0$. \\
    Squaring both sides, we have $(x_1y_1 + \dots + x_ny_n)^2 \leq (x_1^2+\dots+x_2^2)(y_1^2+\dots+y_2^2)$. Fuck vectors. \\
    \item Show that $\norm{x+y} \leq \norm{x} + \norm{y}$. Hint: compute $(x+y)\cdot(x+y)$ and apply (b). \\
    See the last sentence of above.
    \item Verify that $d$ is a metric.
  \end{enumerate}
  \item Let $X$ denote the subset of $R^\omega$ consisting of all sequences $(x_1,x_2,\dots)$ such that $\sum x_i^2$ converges. You may assume the standard facts about infinite series, listed in exercise 11 of the following section. \begin{enumerate}
    \item Show that if $x,y\in X$, then $\sum \abs{x_iy_i}$ converges. Hint: Use (b) of exercise 9 to show that the partial sums are bounded.
    \item Let $x\in \R$. Show that if $x,y\in X$, then so are $x+y$ and $cx$. \\
    \item Show that $d(x,y) = \left(\displaystyle\sum_{i=1}^\infty(x_i-y_i)^2\right)^{1/2}$ is a well-defined metric on $X$.
  \end{enumerate}
  \item [*11.] Show that if $d$ is a metric for $X$, then $d'(x,y) = d(x,y)/(1+d(x,y))$ is a bounded metric that gives the topology of $X$. Hint: if $f(x) = x/(1+x)$ for $x>0$, use the mean-value theorem to show that $f(a+b)-f(b)\leq f(a)$. \\
  The first two conditions for a metric clearly hold. For the third, we must show that $$\frac{d(x,y)}{1+d(x,y)} + \frac{d(y,z)}{1+d(y,z)} \leq \frac{d(x,z)}{1+d(x,z)}$$.
  %\begin{equation} \begin{split}
  %    d(x,y)(1 + d(y,z) + d(x,z) + d(y,z)d(x,z)) + \\
  %    d(y,z)(1 + d(x,y) + d(x,z) + d(x,y)d(x,z)) \leq \\
  %    d(x,z)(1 + d(x,y) + d(y,z) + d(x,y)d(y,z))
  %  \end{split} \end{equation}
  %$$\frac{1+d(x,y)+d(y,z)+d(x,y)d(y,z)}{d(x,y)+d(y,z)+2d(y,z)d(x,y)} \geq \frac{1+d(x,z)}{d(x,z)}$$
  %$$\frac{1+d(x,z)+d(x,y)d(y,z)}{d(x,z)+2d(x,y)d(y,z)} \geq \frac{1+d(x,z)}{d(x,z)}$$
  %Let $d(x,z)=u$ and $d(x,y)d(y,z)=c$ Then we have $$\frac{1+u+c}{u+2c} \geq \frac{1+u}{u}$$
  %$$u(1+u)+cu \geq u(1+u)+2c+2cu$$
  %$$cu \geq 2c+2cu$$
  %Oops.
\end{enumerate}
\end{document}