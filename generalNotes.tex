\documentclass[12pt,letterpaper]{article}
\usepackage[pdftex]{graphicx}
\usepackage{alltt}
\usepackage[margin=1in]{geometry}
\usepackage{amsmath, amsthm, amssymb}
\usepackage{verbatim}
\usepackage{ragged2e}
\usepackage{enumitem}
\usepackage{xfrac}
\setlist{parsep=0pt,listparindent=\parindent}
\setlength{\RaggedRightParindent}{\parindent}
\newcommand{\degree}{\ensuremath{^\circ}}
\newcommand{\n}{\break}
\let\oldemptyset\emptyset
\let\emptyset\varnothing
\newcommand{\Wlog}{without loss of generality}
\newcommand{\WLOG}{Without loss of generality}
\usepackage{accents}
\let\thinbar\bar
\newcommand\thickbar[1]{\accentset{\rule{.4em}{.8pt}}{#1}}
\let\bar\thickbar
\usepackage{standalone}
\usepackage{hyperref}
\newcommand{\R}{\ensuremath{\mathbb{R}}}
\usepackage{mathtools}
\DeclarePairedDelimiter{\ceil}{\lceil}{\rceil}
\DeclarePairedDelimiter{\floor}{\lfloor}{\rfloor}
\DeclarePairedDelimiter\abs{\lvert}{\rvert}
\DeclarePairedDelimiter\norm{\lVert}{\rVert}
%%%%%%%%%%%%%%%%%%%%%%%%%%%%%%%%%%%%%%%%%%%%%%%%%%%%%
%TOPOLOGY DOCUMENTS ONLY%
\newcommand{\T}{\ensuremath{\mathcal{T}}}
%%%%%%%%%%%%%%%%%%%%%%%%%%%%%%%%%%%%%%%%%%%%%%%%%%%%%

\begin{document}
\RaggedRight
\textbf{Table of Contents}
\begin{itemize}
  \item[] \hyperref[sec:Definitions]{Definitions and Theorems}
  \begin{itemize}
    \item[] \hyperref[sec:EarlyChapters]{Early Chapters}
    \item[] \hyperref[sec:continuity]{Continuity}
    \item[] \hyperref[dfn:cartesianProducts]{Cartesian Products}
    \item[] \hyperref[dfn:metric]{Metrics}
    \item[] \hyperref[dfn:quotientMapTopology]{Quotient Maps, Spaces, and Topology}
    \item[] \hyperref[sec:connectedness]{Connected Spaces}
  \end{itemize}
  \item[]  Exercises
  \begin{itemize}
    \item[] \hyperref[sec:chapter1.7]{Chapter 1 Section 7}
    \item[] \hyperref[sec:chapter2.13]{Chapter 2 Section 13}
    \item[] \hyperref[sec:chapter2.16]{Chapter 2 Section 16}
    \item[] \hyperref[sec:chapter2.17]{Chapter 2 Section 17}
    \item[] \hyperref[sec:chapter2.18]{Chapter 2 Section 18}
    \item[] \hyperref[sec:chapter2.19]{Chapter 2 Section 19}
    \item[] \hyperref[sec:chapter2.20]{Chapter 2 Section 20}
    \item[] \hyperref[sec:chapter2.21]{Chapter 2 Section 21}
    \item[] \hyperref[sec:chapter2.22]{Chapter 2 Section 22}
    \item[] \hyperref[sec:chapter3.23]{Chapter 3 section 23}
  \end{itemize}
\end{itemize}
\noindent \textbf{Definitions:} \label{sec:Definitions}
\begin{enumerate}
  \item \label{sec:EarlyChapters} Basics, before continuity. \begin{itemize}
    \item \label{dfn:functionRestriction} $f|A$: The function $f$ restricted to the domain $A$.
    \item \label{thm:AxiomChoice} The axiom of choice: given a collection $\mathcal{A}$ of disjoint nonempty sets, there exists a set $C$ consisting of exactly one element from each element of $\mathcal{A}$; that is, a set $C$ such that $C$ is contained within the union of elements of $\mathcal{A}$, and for each $A\in\mathcal{A}$, the set $C\cap A$ contains a single element.
    \item \label{dfn:finiteComplementTopology} The finite complement topology $\T_f$: Let $X$ be a set, then $\T_f$ is the collection of all subsets $U$ of $X$ such that $X-U$ is either finite or all of $X$
    \item \label{dfn:finer} Finer: $\T_1$ is finer than $\T_2$ implies that $\T_2\subseteq \T_1$, in other words, $\T_1$ has all the open sets of $\T_2$ and more.
    %previous is at most section 12, now at least section 13
    \item \label{dfn:basis} A basis for a topology on $X$ is a collection $\mathcal{B}$ of subsets of $X$ (called basis elements) such that for each $x\in X$ there is at least one basis element containing $x$, and if $x$ belongs to the intersection of two basis elements $B_1$ and $B_2$, then there is a basis element $B_3$ containing $x$ such that $B_3 \subset B_1 \cap B_2$. Then the topology generated by the basis is: a subset $U$ of $X$ is open if for each $x\in U$ there is a basis element $B$ such that $x\in B$ and $B\subset U$.
    \begin{itemize}
      \item \label{thm:basisUnion} Equivalently, given $X$ and a basis $\mathcal{B}$, the topology generated by the basis is the collection of all unions of elements of $\mathcal{B}$.
      \item \label{thm:basisFiner} Also, $\T'$ is finer that $\T$ if and only if for each $x\in X$ and each basis element $B\in\mathcal{B}$ containing $x$, there is a basis element $B'\in \mathcal{B}'$ such that $x\in B'\subset B$.
    \end{itemize}
    \item \label{dfn:subbasis} Subbasis: a subbasis $\mathcal{S}$ for a topology on $X$ is a collection of subsets of $X$ whose union equals $X$. The topology generated by the subbasis is defined to be the collection of all unions of finite intersections of elements of $\mathcal{S}$.
    \item \label{dfn:lowerLimitTopology} The lower limit topology on $\R$, denoted $\R_\ell$ is defined as the topology with the basis that is the collection of all half open intervals of the form $[a,b) = \{ x\; |\; a\leq x < b \}$
    \item \label{dfn:KTopology} The $K$-Topology on $\R$ is defined as follows. First, let $K$ denote the set of all numbers $1/n$, for $n\in\mathbb{Z}_+$. The $K$-topology is generated by the basis that is defined by the collection of all open intervals $(a,b)$, along with all sets of the form $(a,b)-K$, with $a,b\in\R$.
    % previous is section 13, after is at least section 14
    \item \label{dfn:orderTopology} Order Topology: If $S$ is a totally ordered set, then let $\chi$ be the basis made of open rays $S$.
    \item \label{dfn:dictionaryOrder} Dictionary order for $\R\times\R$ is defined by the basis that is the collection of all open intervals $(a\times b, c\times d)$ where $a<c$, or both $a=c$ and $b<d$.
    \item \label{dfn:orderedSquare} The ordered square, $I_o^2$ is defined as $[0,1]\times [0,1]$ in the dictionary order topology.
    \item \label{dfn:projection} Projection: for example, $\pi_1: X\times Y \rightarrow X$ is defined by $\pi_1(x,y)=x$, and $\pi_2(x,y)=y$.
    \item \label{dfn:subspace} Subspace Topology: $Y$ is the \emph{subspace} of $X$ if $Y$ is a subset of $X$, and $X$ is a topological space with topology $\T$, and the \emph{subspace topology} on $Y$ is $\T_Y=\{Y\cap U | U\in \T$.
  \end{itemize}
  \item \label{sec:closedSetsLimitPointsHausdorff} Closed Sets, Limit Points, Hausdorff Spaces - Chapter 2, Section 17 \begin{itemize}
    \item \label{dfn:intersects} Intersects: $A$ intersects $B$ if $A \cap B \neq \emptyset$
    \item \label{dfn:neighborhood} Neighborhood: If $U$ is an open set containing $x$, then $U$ is a neighborhood of $x$.
    \item \label{dfn:interior} Interior of $A$ ($\text{Int }A$): The union of all open sets contained in $A$. The interior of an open set is itself.
    \item \label{dfn:closure} Closure of $A$ ($\bar{A}$): The intersection of all the closed sets containing $A$. Equivalently, $A$ together with all of its limit points. The closure of a closed set is itself.
    \item \label{dfn:boundary} Boundary: $\text{Bd } A = \bar{A}\cap(\overline{X-A})$. Also, $\bar{A} = \text{Int } A \cup \text{Bd } A$. The interior and the boundary of $A$ are disjoint, and the boundary is empty if and only if $A$ is open and closed. $U$ is open if and only if the boundary is $\bar{U}-U$.
    \item \label{dfn:limitPoint} Limit Point: $x$ is a limit point of $A$ if and only if every neighborhood $U$ of $x$ contains some point $y\in A$ distinct from $x$.
    \item \label{thm:BelongsClosure17.5} Theorem 17.5:  $x\in \bar{A}$ if and only if every open set $U$ or basis element $B$ containing $x$ intersects $A$.
    \item \label{dfn:converge} Convergence: A sequence $x_1, x_2, ...$ converges to $x$ if for every neighborhood $U$ of $x$ there exists some $m_U\in\mathbb{N}$ such that $\forall n>m_U\; x_n\in U$.
    \item \label{dfn:Hausdorff} Hausdorff Space: If for each pair of distinct points $x_1,x_2$ of a topologocal space $X$ there exist \emph{disjoint} neighborhoods $U_1$ and $U_2$ of $x_1$ and $x_2$ respectively, then the space is Hausdorff.
    \begin{itemize}
      \item \label{thm:FinitePointHausdorff} Every finite point set in a Hausdorff space is closed. (This is weaker than Hausdorff, on it's own, it is the \textbf{$T_1$ axiom.}
      \item \label{thm:LimitPointT1} $X$ satisfies the $T_1$ axiom, $A$ is a subset. Then, the point $x$ is a limit point of $A$ if and only if every neighborhood of $x$ contains infinitely many points of $A$.
      \item \label{thm:inheritingHausdorff} Every simply ordered set is Hausdorff in the order topology, the product of two Hausdorff spaces is Hausdorff, and a subspace of a Hausdorff space is Hausdorff.
    \end{itemize}
    %End of section 17
  \end{itemize}
  \item \label{sec:continuity} Continuity. Chapter 2, section 18
  \begin{itemize}
    \item \label{dfn:continuous} Continuous: A function $f: X\rightarrow Y$ is said to be continuous if for each open subset $V$ of $Y$, the set $f^{-1}(V)$ is an open subset of $X$. It is sufficient to show that the inverse image of every basis element, or even of every subbasis element is open.
    Equivalently:
    \begin{enumerate} %Theorem 18.1
      \item \label{dfn:continuous2} For every subset $A$ of $X$, $f(\bar{A}) \subset \overline{f(A)}$
      \item \label{dfn:continuous3} For every closed set $B$ of $Y$, the set $f^{-1}(B)$ is closed in $X$.
      \item \label{dfn:continuous4} For each $x\in X$ and each neighborhood $V$ of $f(x)$, there is a neighborhood $U$ of $x$ such that $f(U) \subset V$
    \end{enumerate}
    Note: if the third condition holds for one point, then $f$ is continuous at that point.
    \item \label{dfn:homeomorphism} Homeomorphism: A bijective function $f$ is a homeomorphism is both $f$ and $f^{-1}$ are continuous.
    Equivalently, $f(U)$ is open if and only if $U$ is open.
    \item \label{dfn:embedding} Embedding: Let $f: X\rightarrow Y$ be a continuous injective map, and set $Z=f(X)$ considered as a subspace of $Y$; then the function $f': X\rightarrow Z$ obtained by restricting the range of $f$ is bijective. If $f'$ is a homeomorphism, then we say $f:X\rightarrow Y$ is an embedding of $X$ in $Y$.
    \item \label{thm:PastingLemma} Pasting Lemma: let $X=A\cup B$ where $A$ and $B$ are closed int $X$. Let $f: A\rightarrow Y$ and $g: B\rightarrow Y$ be continuous. If $f(x)=g(x)$ fo every $x\in A\cap B$ then $f$ and $g$ combine to give a continuous function $h: X\rightarrow Y$ defined by setting $h(x) = f(x)$ if $x\in A$ and $h(x) = g(x)$ if $x\in B$.
    \item \label{thm:LocalFormulationContinuity} The local formulation of continuity: If $X$ can be written as the union of upen sets $U_\alpha$ such that $f|U_\alpha$ is continuous for each $\alpha$, then the map $f: X\rightarrow Y$ is continuous.
    \item \label{thm:ChangingRangeDomainContinuity} Restricting the domain, or restricting or expanding the range of a continuous function will yield a continuous function.
    \item \label{thm:MapsProducts18.4} Theorem 18.4 - Maps into products. Let $f : A\rightarrow X\times Y$ be given by $f(a)=(f_1(a),f_2(a))$. Then $f$ is continuous if and only if $f_1$ and $f_2$ (the coordinate functions of $f$) are continuous.
  \end{itemize}
  \item \label{dfn:cartesianProducts} Cartesian Products, chapter 2 section 19
  \begin{itemize}
    \item $\mathcal{A}$ is a nonempty collection of sets. Indexing function for $\mathcal{A}$ is a surjective function $f$ from the index set, $J$, to $\mathcal{A}$. Given $\alpha \in J$, the set $f(\alpha)$ will be denoted $A_\alpha$, the indexed family will be $\{A_\alpha\}_{\alpha\in J}$ or $\{A_\alpha\}$.
    \item Arbitrary unions/intersections: $\displaystyle\bigcup_{\alpha\in J} A_\alpha = \{x |\; \exists \alpha \in J \text{ such that } x\in A_\alpha \}$. $\displaystyle\bigcap_{\alpha\in J} A_\alpha = \{x |\; \forall \alpha \in J,\; x\in A_\alpha \}$.
    \item $m$-Tuples: an $m$-tuple is the function $x: (1,\dots,m)\rightarrow X$. $x(i)$ is typically denoted $x_i$ and called the $i$th coordinate of $x$, and $x$ is denoted $(x_1\dots x_m)$.
    \item Cartesian product. $\{A_1\dots A_m\}$ is a family of sets, indexed by the set $\{1\dots m\}$. $X = A_1 \cup\dots \cup A_m$. The cartesian product $\displaystyle\prod_{i=1}^{m}A_i$ is the set of all $m$-tuples $(x_1,\dots ,x_m)$ of elements of $X$ such that $\forall i\; x_i \in A_i$
    \item $J$-tuple: given an index set $J$ a $J$-tuple is a function $x: J\rightarrow X$. For any $\alpha\in J$, we denote $x(\alpha)$ by $x_\alpha$, we call it the $\alpha$-th coordinate of $x$, and denote the function $x$ by the symbol $(x_\alpha)_{\alpha\in J}$. The set of all $J$-tuples of elements of $X$ is denoted $X^J$.
    \item Cartesian product. $\{A_\alpha\}_{\alpha\in J}$ is an indexed family of sets, $X=\bigcup_{\alpha\in J}A_\alpha$. The cartesian product, $\prod_{\alpha\in J} A_\alpha$ is defined as the set of all $J$-tuples of elements of $X$ such that $x_\alpha \in A_\alpha$ for each $\alpha\in J$. Also denoted $\prod A_\alpha$, and its general element as $(x_\alpha)$ if the index set is understood.
    \item Box Topology: let $(X_\alpha)_{\alpha\in J}$ be an indexed family of topological spaces. Then give the product space $\displaystyle\prod_{\alpha\in J}X_\alpha$ the basis that is the collection of all sets of the form $\displaystyle\prod_{\alpha\in J}U_\alpha$, where $U_\alpha$ is open in $X_\alpha$ for all $\alpha\in J$.
    \item Product topology: Let $\pi_\beta : \displaystyle\prod_{\alpha\in J}X_\alpha \rightarrow X_\beta$ be defined $\pi_\beta((x_\alpha)_{\alpha\in J}) = x_\beta$; it maps each element of the product space to its $\beta$th coordinate, it is the projection mapping associated with index $\beta$. \\
    Let $\mathcal{S}_\beta$ denote the collection $\mathcal{S}_\beta = \{\pi_\beta^{-1}(U_\beta) | U_\beta \text{ is open in } X_\beta\}$. Then let $\mathcal{S}$ denote the union $\mathcal{S}=\displaystyle\bigcup_{\beta\in J}\mathcal{S}_\beta$. The topology generated by subbasis $\mathcal{S}$ is called the product topology, and $\prod_{\alpha\in J} X_\alpha$ is a product space.\\
    From this, we get the basis of the product topology: all sets of the form $\prod U_\alpha$, where $U_\alpha$ is an open set of $X_\alpha$, and is equal to $X_\alpha$ for all but finitely many values of $\alpha$.
    \item \label{thm:MapsProducts19.6} Theorem 19.6: Let $f: A \rightarrow \prod_{\alpha\in J}X_\alpha$ be defined $f(a) = (f_\alpha(a))_{\alpha \in J}$, where $f_\alpha : A \rightarrow X_\alpha$ for each $\alpha$. Let $\prod X_\alpha$ have the product topology. Then $f$ is continuous if and only if each $f_\alpha$ is continuous.
  \end{itemize}
  \item \label{dfn:metric} Metrics, chapter 2 sections 20,21
  \begin{itemize}
    \item A metric is a function $d: X \times X \rightarrow \mathbb{R}$ such that: $d(x,y)>0$ for $x\neq y$, $d(x,x)=0$; $d(x,y)=d(y,x)$; and $d(x,y) + d(y,z)\geq d(x,z)$.
    \item The $\epsilon$-ball centered at $x$ is $B_d(x, \epsilon) = \{ y | d(x,y) < \epsilon \}$.
    \item \label{dfn:metricTopology} The metric topology induced by a metric $d$ on a set $X$ is defined by the basis consisting of all $\epsilon$-balls $B_d(x,\epsilon)$ for all $x\in X$ and $\epsilon > 0$. Alternatively, a set $U$ is open in the metric topology induced by $d$ if and only if for each $y\in U$, there is a $\delta > 0$ such that $B_d(y,\delta) \in U$.
    \item A space is metrizable of there exists a metric on it that induces its topology. A metric space is a metrizable space together with the metric that gives it its topology.
    \item Bounded: A subset $A$ of $X$ is bounded if there exists an $M$ such that for all points $x,y$ $d(x,y)<M$. If $A$ is bounded and nonempty, the diameter of $A$ is defined as $\sup( \{d(x,y) | x,y\in X\})$
    \item Standard Bounded Metric $\bar{d}$: Let $X$ be a metric space with metric $d$. Define $\bar{d}: X\times X \rightarrow \mathbb{R}$ by the equation $\bar{d}(x,y) = \min(d(x,y),1)$. $\bar{d}$ induces the same topology as $d$.
    \item Euclidean and Square Metrics on $\mathbb{R}^n$: The euclidean metric $d(x,y) = ||x-y|| = ((x_1 - y_1)^2 + \dots + (x_n - y_n)^2)^{1/2}$. The square metric $p(x,y) = \max(|x_1 - y_1|, \dots , |x_n - y_n|)$.
    \item Uniform Metric $\bar{p}$ on $\mathbb{R}^J$, inducing the uniform topology: $\bar{p}(x,y) = \sup(\{\bar{d}(x_\alpha, y_\alpha) | \alpha \in J\})$. The uniform topology is finer than the product topology and coarser than the box topology, and different if $J$ is infinite.
    \item \label{dfn:countableBasis} Countable Basis at Point $x$: one exists if there is a countable collection $\{U_n\}_{n\in\mathbb{Z}_+}$ of neighborhoods of $x$ such that any neighborhood $U$ of $x$ contains at least one of the sets $U_n$. A space that has a countable basis at each of its points satisfies the \emph{first countability axiom}. All metrizable spaces satisfy this axiom.
    \item \label{dfn:convergesUniformly} Converging Uniformly: Let $(f_n): X\rightarrow Y$ be a sequence of functions where $Y$ is a metric space with metric $d$. The sequence converges uniformly to the function $f:X\rightarrow Y$, if given any $\epsilon>0$ there exists an integer $N$ such that $d(f_n(x), f(x))<\epsilon$ for all $n>N$ and for all $x\in X$.  If $(f_n)$ converges uniformly to $f$ and each $f_n$ is continuous, then $f$ is continuous.
    \item \label{thm:SequenceContinuity21.3} Theorem 21.3 and Lemma 21.2 (The Sequence Lemma) - If there is a sequence of points of $A$ converging to $x$, then $x\in \bar{A}$, the converse holds if $X$ has a countable basis at point $x$, or, more strongly, if $X$ is metrizable. If a function $f$ is continuous, then for every convergent sequence $x_n\rightarrow x$ in $X$, the sequence $f(x_n)$ converges to $f(x)$, again the converse holds if $X$ has a countable basis at $x$. 
  \end{itemize}
  \item \label{dfn:quotientMapTopology} Quotient Topology, chapter 2 section 22
  \begin{itemize}
    \item \label{dfn:saturated} Saturated: A subset $C$ of $X$ is saturated with respect to a surjective map $p: X\rightarrow Y$ if $C$ contains every set $p^{-1}(\{y\})$ that it intersects. 
    \item \label{dfn:quotientMap} Quotient Map: let $p: X\rightarrow Y$ be a surjective map. $p$ is a quotient map when a subset $U$ of $Y$ is open if and only if $p^{-1}(U)$ is open in $X$, which is stronger than continuity. Equivalently, replace 'open' by 'closed'. Alternatively, $p$ is a quotient map if $p$ is continuous and $p$ maps saturated open sets of $X$ to open sets of $Y$, or saturated closed sets of $X$ to closed sets of $Y$.
    \item \label{dfn:OpenClosedMap} An open map $f:X\rightarrow Y$ is one that maps every open set of $X$ to an open set of $Y$, and a closed map sends every closed set of $X$ to a closed set of $Y$. If $f$ is a surjective open or closed map, $f$ is a quotient map, but some quotient maps are neither open nor closed.
    \item \label{dfn:quotientTopology} Quotient Topology: If $X$ is a space and $A$ is a set and if $p: X\rightarrow A$ is a surjective map, then there exists exactly one topology $\T$ on $A$ relative to which $p$ is a quotient map; $\T$ is the quotient topology induced by $p$.
    \item \label{dfn:quotientSpace} Quotient Space: Let $X$ be a topological space and let $X^*$ be a partition of $X$ into disjoint subsets whose union is $X$. Let $p: X\rightarrow X^*$ be the surjective map that carries each element of $X$ to the element of $X^*$ that contains it. In the quotient topology induced by $p$, the space $X^*$ is called the quotient space of $X$. Often called the identification space or decomposition space of $X$. The typical open set of $X^*$ is a collection of equivalence classes whose union is open in $X$.
    \item \label{thm:restrictionQuotientMap} Theorem 22.1 Restricting Quotient Maps: Let $p: X\rightarrow Y$ be a quotient map; let $A$ be a subspace of $X$ that is saturated with respect to $p$; let $q: A\rightarrow p(A)$ be the map obtained by restricting $p$. If $A$ is either open or closed in $X$, then $q$ is a quotient map. If $p$ is either an open or closed map, then $q$ is a quotient map.
    \item \label{thm:quotientMapGeneralProperties} The restriction of a quotient map need not be a quotient map. The composition of two quotient maps is a quotient map. The cartesian product of two quotient maps need not be a quotient map, unless both maps are open, or the spaces are locally compact (defined later). If $X$ is Hausdorff, the quotient space $X^*$ need not be Hausdorff.
    \item \label{thm:continuityQuotientMap} Theorem 22.2 Continuity of quotient maps: Let $p: X\rightarrow Y$ be a quotient map. Let $Z$ be a space and let $g\rightarrow X\rightarrow Z$ be a map that is constant on each set $p^{-1}(\{y\})$, for $y\in Y$. Then $g$ induces a map $f:Y\rightarrow Z$ such that $f\circ p = g$. The induced map $f$ is continuous if and only if $g$ is continuous; $f$ is a quotient map if and only if $g$ is a quotient map.
    \item \label{thm:HausdorffHomeomorphismQuotientMap} Let $g: X\rightarrow Z$ be a surjective continuous map. Let $X^*$ be the following collectino of subsets of $X$: $X^*=\{g^{-1}(\{z\}) | z\in Z\}$. Give $X^*$ the quotient topology. If $Z$ is Hausdorff, so is $X^*$. The map $g$ induces a bijective continuous map $f: X^*\rightarrow Z$, which is a homeomorphism if and only if $g$ is a quotient map.
  \end{itemize}
  \item \label{sec:connectedness} Connected Spaces, chapter 3 section 23
  \begin{itemize}
    \item \label{dfn:separation} Separation: Let $X$ is a topological space. A separation of $X$ is a pair $U,V$ of disjoint nonempty open subsets of $X$ whose union is $X$.
    \item \label{dfn:connected} Connected: A space $X$ is connected if there exists no separation of $X$, or equivalently if and only if the subsets of $X$ that are both open and closed in $X$ are the empty set and itself. Any space homeomorphic to a connected space is connected.
    \item \label{dfn:subspaceSeparation} Separation of a subspace - Lemma 23.1: If $Y$ is a subspace of $X$, a separation of $Y$ is a pair of disjoint nonempty sets $A$ and $B$ whose union is $Y$, neither of which contains a limit point of the other.
    \item \label{thm:subspaceOfConnected} Subspace of a Connected Space - Lemma 23.2: If $C$ and $D$ form a separation of $X$ and $Y$ is a connected subspace of $X$, then $Y$ lies entirely in $C$ or $D$.
    \item \label{thm:unionConnected} Union of connected spaces, Theorem 23.3: The union of a collection of connected subspaces of $X$ that have a point in common is connected.
    \item \label{thm:closureConnected} Closure of a connected space: Let $A$ be a connected subspace of $X$. If $A\subset B\subset \bar{A}$, then $B$ is also connected. Equivalently, if $B$ the union of $A$ and some or all of its limit points, then $B$ is connected.
    \item \label{thm:continuousConnected} Theorem 23.5: The image of a connected space under a continuous map is connected.
    \item \label{thm:finiteCartesianConnected} Theorem 23.6: A finite cartesian product of connected spaces is connected. The arbitrary product is connected in the product topology.
  \end{itemize}
\end{enumerate}
\textbf{Chapter 1.7} \label{sec:chapter1.7}
\documentclass[12pt,letterpaper]{article}
\usepackage[pdftex]{graphicx}
\usepackage{alltt}
\usepackage[margin=1in]{geometry}
\usepackage{amsmath, amsthm, amssymb}
\usepackage{verbatim}
\newcommand{\degree}{\ensuremath{^\circ}}
\newcommand{\n}{\break}
\let\oldemptyset\emptyset
\let\emptyset\varnothing
\newcommand{\Wlog}{without loss of generality}
\newcommand{\WLOG}{Without loss of generality}

\begin{document}
\raggedright
3. Let $X$ be the set $\{0,1\}$. Show that there is a bijective correspondense between $\mathcal{P}(\mathbb{Z}_+)$ and the cartesian product $X^\omega$ \n
Let $x=(x_1, x_2, x_3, ...)\in X$ where each $x_i\in \{0,1\}$. \n
Now define the map
$$f : X^\omega \rightarrow \mathcal{P}(\mathbb{Z}_+)$$
to be
$$f(x) = \{i\, |\, x_i=1\}$$
$f$ is injective - consider $x_a, x_b$ such that $f(x_a)=f(x_b)$.
Then for each $i$, $x_{ai}=x_{bi}$, so $x_a=x_b$ \n
Furthermore, $f$ is surjective. For each $n\in\mathbb{Z}_+$, each subset of $\mathbb{Z}_+$ either contains $n$ or does not. $X$ contains every sequence where $x_n$ is 1 ($n$ is in the subset) and all those where $x_n$ is 0 ($n$ is not in the subset).
\n
\n
5a,b) Countablely infinite, there's a bijective mapping to $\mathbb{Z}_+^n$.
The set $B_n$ can be bijectively mapped to the set of subsets of $\mathbb{Z}_+$ of size $n$.\n
c) Uncountable, equivalent to $\mathcal{P}(\mathbb{Z}_+)$.\n
d,e) Uncountable, assigning a 0 or 1 to each element in $\mathbb{Z}_+$ was shown to be equivalent to $X^\omega$ in 3), which is uncountably infinite.\n
f) Countable, there is one such function for every choice of whether $n$ is 0 or 1 for all $n$ up to $N$, thus there are $2^N$ such functions.\n
g) Countable infinite, it's simply the number of subsets of $\mathbb{Z}_+$ of size N.\n
h) A countable union of the above, thus it is countably infinite.\n




\end{document}

%\textbf{Chapter 1.10}
%\documentclass[12pt,letterpaper]{article}
\usepackage[pdftex]{graphicx}
\usepackage{alltt}
\usepackage[margin=1in]{geometry}
\usepackage{amsmath, amsthm, amssymb}
\usepackage{verbatim}
\usepackage{ragged2e}
\usepackage{enumitem}
\usepackage{xfrac}
\setlist{parsep=0pt,listparindent=\parindent}
\setlength{\RaggedRightParindent}{\parindent}
\newcommand{\degree}{\ensuremath{^\circ}}
\newcommand{\n}{\break}
\let\oldemptyset\emptyset
\let\emptyset\varnothing
\newcommand{\Wlog}{without loss of generality}
\newcommand{\WLOG}{Without loss of generality}
\usepackage{accents}
\let\thinbar\bar
\newcommand\thickbar[1]{\accentset{\rule{.4em}{.8pt}}{#1}}
\let\bar\thickbar
\usepackage{standalone}
\usepackage{hyperref}
\newcommand{\R}{\ensuremath{\mathbb{R}}}
\usepackage{mathtools}
\DeclarePairedDelimiter{\ceil}{\lceil}{\rceil}
\DeclarePairedDelimiter{\floor}{\lfloor}{\rfloor}
\DeclarePairedDelimiter\abs{\lvert}{\rvert}
\DeclarePairedDelimiter\norm{\lVert}{\rVert}
%%%%%%%%%%%%%%%%%%%%%%%%%%%%%%%%%%%%%%%%%%%%%%%%%%%%%
%TOPOLOGY DOCUMENTS ONLY%
\newcommand{\T}{\ensuremath{\mathcal{T}}}
%%%%%%%%%%%%%%%%%%%%%%%%%%%%%%%%%%%%%%%%%%%%%%%%%%%%%

\begin{document}
\RaggedRight
\begin{enumerate}
  \item Prove theorem 19.2, which is as follows: Suppose the topology on each space $X_\alpha$ is given by a basis $\mathcal{B}_\alpha$. The collection of all sets of the form $\prod_{\alpha\in J}B_\alpha$ where $B_\alpha \in \mathcal{B}_\alpha$ for each $\alpha$, will serve as a basis for the box toplogy on $\prod_{\alpha\in J} X_\alpha$. The collection of all sets of the same form, where $B_\alpha \in \mathcal{B}_\alpha$ for finitely many indices $\alpha$ and $B_\alpha=X_\alpha$ for all remaining indices, will serve as a basis for the product topology $\prod_{\alpha\in J}X_\alpha$.\hspace{5in}\n
  \indent Let the collection that is to be proven to be a basis for the topology in question over the product space $X=\prod_{\alpha\in J}X_\alpha$ be denoted $\mathcal{C}$. Let $x=(x_\alpha)$ be any point of $\prod X_\alpha$. We show that $\mathcal{C}$ is a basis. \n
  \indent For each $x_\alpha$ there exists a basis element $B_\alpha$ containing it (or $X_\alpha$ obviously contains it), since each $\mathcal{B}_\alpha$ is a basis; call each such element $C_\alpha$. Then $\prod_\alpha C_\alpha$ gets an element of $\mathcal{C}$ which contains $x$. Hence there is at least a basis element containing any $x\in X$.
  Next, suppose $x$ belongs to the intersection of two basis elements, $C_1$ and $C_2$. This is equivalent to $\prod_\alpha C_{1_\alpha} \cap C_{2_\alpha}$. Since each $x_\alpha$ must belong to $C_{1_\alpha} \cap C_{2_\alpha}$, there must be a corresponding basis element $C_{3_\alpha}\in\mathcal{B}_\alpha$ such that $x\in C_{3_\alpha}\subset C_{1_\alpha} \cap C_{2_\alpha}$ since each $\mathcal{B}_\alpha$ is a basis. Then $\prod_\alpha C_{3_\alpha}$ is an element of $\mathcal{C}$. Hence, $\mathcal{C}$ is a basis.\n
  \indent Now we show that the topology generated by $\mathcal{C}$ is the correct topology.
  We show that $\mathcal{C}$ is equivalent to the basis that generally forms the box topology, the collection of all sets of the form $\prod_\alpha U_\alpha$, where $U_\alpha$ is open in $X_\alpha$ for all $\alpha\in J$. Consider an element of the defining basis, $U=\prod_\alpha U_\alpha$. Since each $\mathcal{B}_\alpha$ is a basis, each $U_\alpha$ can be formed by some union of elements of $\mathcal{B}_\alpha$. Hence $U$ can be formed by unions of elements of $\mathcal{C}$. The same argument applies when looking at the basis generated by the defining subbasis of the product topology.
  \item Prove theorem 19.3, which is as follows: Let $A_\alpha$ be a subspace of $X_\alpha$ for each $\alpha \in J$. Then $A=\prod A_\alpha$ is a subsapce of $X=\prod X_\alpha$ if both products are given either the box topology or the product topology.\hspace{5in}\n
  \indent Let $\T_s$ be the topology $A$ inherits from $X$ in the product topology. Hence, $\T_s$ is generated by $\mathcal{B}_s = \{ B \cap A | B\in \mathcal{B} \}$, where $\mathcal{B}$ is the basis of the $X$ with the product topology. Let $\T_p$ be the product topology on $A$, with the basis $\mathcal{B}_p=\prod U_\alpha$, where $U_\alpha$ is open in $A_\alpha$ and equal to $A_\alpha$ except for finitely many values of $\alpha$. \n
  \indent Consider an element $B$ of $\mathcal{B}_s$. $B=\prod B_\alpha \cap A_\alpha$. Since $B_\alpha \cap A_\alpha$ is open in $A_\alpha$, and where $B_\alpha$ is equal to $X_\alpha$, $B_\alpha \cap A_\alpha = A_\alpha$, $B\in \mathcal{B}_p$.\hspace{5in}\n
  \indent Now consider an element $B$ of $\mathcal{B}_p$. Any open element $U_\alpha$ of $A_\alpha$ must be equal to some $V_\alpha \cap A_\alpha$, where $V_\alpha$ is an open set of $A_\alpha$, because $A_\alpha$ is a subspace of $X_\alpha$. Thus, $B=\prod V_\alpha \cap A_\alpha$. For all but a finite number of $\alpha$s, $V_\alpha$ may be $X_\alpha$, since the corresponding $U_\alpha$ is $A_\alpha$ for all but a finite number of $\alpha$s. Hence, $B\in \mathcal{B_s}$, since the corresponding $B_\alpha$ of a $B\in \mathcal{B}$ may be any open set of $X_\alpha$. The proof follows similarly for the box topology.
  \item Prove Theorem 19.4, which is: If each space $X_\alpha$ is a Hausdorff space, then $X=\prod X_\alpha$ is a Hausdorff space in both the box and the product topologies. \hspace{5in} \n
  \indent For any distinct points $x', x'' \in X$ there exist disjoint neighborhoods $U_\alpha', U_\alpha''$ for $x', x''$ respectively, for each $\alpha \in J$ where $x_\alpha' \neq x_\alpha''$, in the other cases, let $U_\alpha'=U_\alpha''$ be any neighborhood of $x_\alpha'$. Thus there exist disjoint neighborhoods $U' = \prod U_\alpha',\; U''=\prod U_\alpha''$, thus $X$ is hausdorff.
  \addtocounter{enumi}{1}
  \item One of the implications stated in Theorem 19.6 holds for the box topology. Which one?\n
  \indent If $f$ is continuous, then $f_\alpha$ is continuous for each $\alpha$.
  \item Let $x_1, x_2, \dots$ be a sequence of points of the product space $X=\prod X_\alpha$. Show that this sequence converges to the point $x$ if and only if the sequence $\pi_\alpha(x_1), \pi_\alpha(x_2), \dots$ converges to $\pi_\alpha(x)$ for each $\alpha$. Is this fact true if one uses the box topology instead of the product topology? \hspace{5in} \n
  \indent Suppose the sequence converges to $x$. Then every neighborhood $U$ of $x$ contains every $x_n$ for all $n$ larger than some $m_U$. Hence, every neighborhood $\pi_\alpha(U)$ of $\pi_\alpha(x)$ contains every $\pi_\alpha(x_n)$ for all $n$ larger than some $m_U$, for each $\alpha$. Furthermore, every neighborhood $V$ of $\pi_\alpha(x)$ has a corresponding neighborhood of $x$: $\pi^{-1}_\alpha(x)$. Thus, since neighborhood $\pi_\alpha(U)$ of $\pi_\alpha(x)$ is every neighorhood $\pi_\alpha(x)$. Therefore, every sequence $\pi_\alpha(x_1), \pi_\alpha(x_2),\dots$ converges to $\pi_\alpha(x)$.\n
  \indent Now suppose every sequence $\pi_\alpha(x_1), \pi_\alpha(x_2)\dots$ converges to $\pi_\alpha(x)$. Any open neighborhood $U_\alpha$ of $\pi_\alpha(x)$ in $X_\alpha$ contains all $\pi_\alpha(x_n)$ greater than some $m_{U_\alpha}$. Use finitely many $U_\alpha$s to construct an open neighborhood of $x$; label these by $i$. $U=\prod U_\alpha$, where $U_\alpha=U_i$ for finitely many values of $\alpha$, and $U_\alpha=X_\alpha$ for the others. Then, for all values $n$ over $\max_\alpha(m_{U_\alpha})$, $U$ will contain $x_n$, thus the sequence converges to $x$.\n
  \indent In the box topology, the second result does not hold, since if the value $U_{m_\alpha}$ increases unboundedly, the final sequence does not converge. For example, in the space $\mathbb{R}^\omega$, consider $x=(0,0,\dots)$, its neighborhood $((-1,1),\allowbreak (-1/2,1/2),\allowbreak (-1/4,1/4),\allowbreak (-1/8,1/8),\allowbreak (-1/16,1/16)\dots)$, and the sequence $x_n = (1/n,1/n,1/n,\dots)$.
  \item Let $\mathbb{R}^\infty$ be the subset of $\mathbb{R}^\omega$ consisting of all sequences that are ``eventually zero,'' that is, all sequences $(x_1, x_2,\dots)$ such that $x_i\neq 0$ for only finitely many values of $i$. What is the closure of $\mathbb{R}^\infty$ in $\mathbb{R}^\omega$ in the box and product topologies?\n
  \indent GO BACK TO THIS, NOT SURE HOW TO APPROACH IT.
  \item Given sequences $(a_1, a_2, \dots)$ and $(b_1, b_2,\dots)$ of real numbers $a_i > 0$ for all $i$, define $h: \mathbb{R}^\omega \rightarrow \mathbb{R}^\omega$ by the equation $h((x_1, x_2, \dots)) = (a_1x_1+b_1, a_2x_2+b_2, \dots)$. Show that if $\mathbb{R}^\omega$ is given the product topology, $h$ is a homeomorphism of $\mathbb{R}^\omega$ with itself. What happens if $\mathbb{R}^\omega$ is given the box topology?\hspace{5in}\n
  \indent First, to prove that $h$ is a homeomorphism, we prove that it is bijective. To do that, we first show that it is injective. Suppose the opposite, there exist distinct $x$ and $y$ such that $h(x)=h(y)$. Then $a_1x_1+b_1 = a_1y_1 + b_1$, so $x=y$, a contradiction. Now we show that $h$ is surjective, again by contradiction. Suppose there is a point $y$ such that there exists no $x$ such that $h(x)=y$. However, the point $x=((y_1-b_1)/a_1, (y_2-b_2)/a_2, \dots)$ is such a point. Thus, $h$ is bijective. \hspace{5in}\n
  \indent Now we show that $h$ is continuous. Each function $h_1(x) = a_1x_1+b_1, h_2(x)=a_2x_2+b_2, \dots$ is clearly continuous, thus, by \hyperref[thm:MapsProducts19.6]{theorem 19.6}, $h$ is continous. Similarly, $h^{-1}$ is continuous, thus $h$ is homeomorphic.\hspace{5in}\n
  \indent In the box topology, theorem 19.6 does not hold, $h$ would not be continuous, and so would not be a homeomorphism. For example, if $a=1$ and $b=0$, we get exactly the function Munkres uses to show that this theorem does not hold for the box topology.
  \item Show that the \hyperref[thm:AxiomChoice]{axiom of choice} is equivalent to the statement that for any indexed family $\{A_\alpha\}_{\alpha\in J}$ of nonempty sets, with $J\neq 0$, the cartesian product $\prod_{\alpha\in J} A_\alpha$ is not empty.\n
  \indent Intuitively, each element of the cartesian product requires the choice of an arbitrary element from each set $A_\alpha$.
  \item Let $A$ be a set; let $(X_\alpha)_{\alpha\in J}$ be an indexed family of spaces; and let $(f_\alpha)_{\alpha\in J}$ be an indexed family of functions $f_\alpha: A \rightarrow X_\alpha$.
  \begin{enumerate}
    \item Show that there is a unique coarsest topology $\T$ on $A$ relative to which each of the functions $f_\alpha$ is continuous.\hspace{5in}\n
    \indent Suppose there were two such coarsest topologies, $\T$ and $\T'$. Each would contain at least one open set, $U$ and $U'$ respectively which the other did not contain. There then exist two $\alpha$, $i$ and $j$, and two open sets $V$ and $V'$ of $X_i$ and $X_j$ such that $f_i^{-1}(V)=U$ and $f_j^{-1}(V')=U'$. However, for $f_j$ to be continuous, $T'$ must contain $U$ and vice versa, hence there is a single unique coarsest topology. NOTE: This shows that the subbasis for such a topology must be unique, not the topology itself. \n
    \indent Consider the intersection of $\T$ and $\T'$. Each $f_\alpha$ is continuous in this intersection, since it is continuous in each topology, but the intersection is necessarily coarser. Thus there is only one possible coarsest topology. 
    \item Let $\mathcal{S}_\beta=\{f_\beta^{-1}(U_\beta) | U_\beta \text{ is open in } X_\beta\}$, and let $\mathcal{S}=\bigcup \mathcal{S}_\beta$. Show that $\mathcal{S}$ is a subbasis for $\T$.\hspace{5in}\n
    \indent For each function to be continuous, it is necessary that each $f_\alpha^{-1}(U_\alpha)$ be open, where $U_\alpha$ is an open set of $X_\alpha$. This subbasis allows for exactly that, without including any other sets. The inverse of each open set must be part of a subbasis to be sure that it forms a topology, they cannot simply be the topology directly.
    \item Show that a map $g: Y\rightarrow A$ is continuous relative to $\T$ if and only if each map $f_\alpha \circ g$ is continuous.\hspace{5in}\n
    \indent If $g$ is continuous, then each $f_\alpha \circ g$ is continuous, since the composition of continuous functions is continuous. Conversely, suppose each $f_\alpha \circ g$ is continuous, but $g$ is not continous. %Then there exists some open set $V\in A$ such that $g^{-1}(V)$ is not open in $Y$. Denote $f_\alpha(V)$ as $U_\alpha$. Then, each $g^{-1}(f^{-1}_\alpha(U_\alpha))$ is open in $Y$.
    % $V$ is some union of some finite intersection sets in $\mathcal{S}$.
    However, consider each subbasis element $U\in\mathcal{S}$. For each one, there exists some $\alpha$ such that there exists an open set $V\in X_\alpha$ with $f_\alpha^{-1}(V)=U$. $g^{-1}(f^{-1}_\alpha(V))$ is open by hypothesis, so $g^{-1}(U)$ is open. It is sufficient to show that the inverse image of each subbasis element is open to show that a function is continuous, thus $g$ is continuous.
    \item Let $f: A\rightarrow \prod X_\alpha$ be defined by the equation $f(a) = (f_\alpha(a))_{\alpha\in J}$; let $Z$ denote the subspace $f(A)$ of the product space $\prod X_\alpha$. Show that the image under $f$ of each element of $\T$ is an open set of $Z$. \hspace{5in}\n
    \indent %In the subspace $f(A)$, $f$ is obviously surjective. Identically, each $\pi_\alpha\circ f = f_\alpha$ maps to the space $\pi_\alpha(f(A)) = f_\alpha(A)$, so each $f_\alpha$ is surjective.
    Let $U$ belong to the subbasis $\mathcal{S}$ of $\T$, $U$ is the preimage of some open set $V\in X_\beta$ for some $\beta$. $f_\beta(U) = f_\beta(f_\beta^{-1}(V)) = V \cap f_\beta(A)$, thus it is an open set in the subspace $f_\beta(A)$.\n
    % Now, consider the set $B=\prod_{\alpha\in J} V_\alpha$, where $V_\alpha = f_\alpha(A)$ for all $\alpha \neq \beta$, and $U_\alpha = f_\alpha(U)$ when $\alpha = \beta$.
    %Let $B=f(U)$, and let $x$ be an element of $B$. %$B$ must be open in $\prod X_\alpha$, because
    The set $\prod V_\alpha$ where $V_\alpha = X_\alpha$ for all $\alpha \neq \beta$ and $V_\alpha=V$ when $\alpha=\beta$ is open in $\prod X_\alpha$. The intersection $B = \prod V_\alpha\cap f(A)$ is exactly $f(U)$. Proof proceeds: consider an $x\in U$. $f_\beta(x) \in V$, and for any other $\alpha$, $f_\alpha(x)$ is obviously in $X_\alpha$, so $f(x)\in B$. In the other direction, consider an $x\in B$. $\pi_\beta(x) \in f_\beta(U)$. Suppose there were a $y$ such that $f_\beta(y)=\pi_\beta(x)$, but $f(y)\not\in f(U)$. However, $y\in f^{-1}_\beta(\pi_\beta(x))$, equivalently, $y\in f^{-1}_\beta(f_\beta(U))$, so $y\in U$. Thus, $\pi_\beta(x)$ being an element of $f_\beta(U)$ is a sufficient condition for $x\in U$. %However, $y$ must be in $f^{-1}_\beta(V)$, and therefore in $U$, since $U=f^{-1}_\beta(V)$, and $\pi_\beta(x) \in f^{-1}(U)$, so $f(\pi_\beta(x)) \in U$ \n
    $B$ is clearly an open set of $Z$, so $f(U)$ is an open element of $Z$, where $U$ is an arbitrary subbasis element of $\T$. To extend this to arbitrary basis elements, simply note that these are finite intersections of subbasis elements, so if $U$ were a basis element, then $U$ would be intersection of the preimage of some finite sets $V_1, V_2,\dots$ of $X_{\beta_1},X_{\beta_2},\dots$ for some finite number of subscripts $\beta$. Since a finite number of the open sets comprising a basis element of the product topology can be any open set, the proof proceeds similarly. The function of a union is the union of the function of each set, thus we are done.
  \end{enumerate}
\end{enumerate}
\end{document}
\textbf{Chapter 2.13} \label{sec:chapter2.13}
\documentclass[12pt,letterpaper]{article}
\usepackage[pdftex]{graphicx}
\usepackage{alltt}
\usepackage[margin=1in]{geometry}
\usepackage{amsmath, amsthm, amssymb}
\usepackage{verbatim}
\usepackage{ragged2e}
\usepackage{enumitem}
\setlist{parsep=0pt,listparindent=\parindent}
\setlength{\RaggedRightParindent}{\parindent}
\newcommand{\degree}{\ensuremath{^\circ}}
\newcommand{\n}{\break}
\let\oldemptyset\emptyset
\let\emptyset\varnothing
\newcommand{\Wlog}{without loss of generality}
\newcommand{\WLOG}{Without loss of generality}

%%%%%%%%%%%%%%%%%%%%%%%%%%%%%%%%%%%%%%%%%%%%%%%%%%%%%
%THIS DOCUMENT ONLY%
\newcommand{\T}{\ensuremath{\mathcal{T}}}
%%%%%%%%%%%%%%%%%%%%%%%%%%%%%%%%%%%%%%%%%%%%%%%%%%%%%

\begin{document}
\RaggedRight
\begin{enumerate}
\item Let $X$ be a topological space; let $A$ be a subset of $X$.
Suppose that for each $x\in A$ there is an open set $U$ containing $x$ such that $U \subset A$.
Show that $A$ is open in $X$. \n
\indent Define $A'$ as the union of all such open sets $U$ . Since each $x\in A$ is in one such set $U$, $A\subset A'$. Since each such $U \subset A$, $A' \subset A$. Thus $A'=A$.
The union of open sets is a open set, thus $A$ is an open set.
\addtocounter{enumi}{2}
\item 
  \begin{enumerate}
  \item If $\{\mathcal{T}_a\}$ is a family of topologies on $X$, show that $\bigcap \mathcal{T}_a$ is a topology on $X$.
    Is $\bigcup \mathcal{T}_a$ a topology on $X$?\n
    \indent Define $\mathcal{T}_b=\bigcup \mathcal{T}_a$. Consider sets $A_1,A_2\in \mathcal{T}_b$.
    By definition, these exist in every topology $\mathcal{T}_a$; since these are topologies, $A_1\cup A_2$ exists in each one, and thus in the intersection.
    The argument is identical for finite intersections of sets $A_i\in \mathcal{T}_b$ \n
    \indent $\bigcup \mathcal{T}_a$ is not a topology on $X$. Consider $X$ = $\{a,b,c\}$, $\mathcal{T}_1=\{\emptyset,\{1\},X\}$, $\mathcal{T}_2=\{\emptyset,\{2\},X\}$. The union of these sets is not a topology.
  \item Let $\{\mathcal{T}_a\}$ is a family of topologies on $X$. Show that there is a unique smallest topology on $X$ containing all the collections $\mathcal{T}_a$, and a unique largest topology contained in all $\mathcal{T}_a$.\n
    The union as the subbasis and the intersection of all the sets respectively.
  \addtocounter{enumii}{1}
  \end{enumerate}
  \addtocounter{enumi}{1}
\item Show that the topologies of $\mathbb{R}_l$ and $\mathbb{R}_K$ are not comparable. \n
  \indent Let $\mathcal{T}$ and $\mathcal{T}'$ be the topologies of $\mathbb{R}_l$ and $\mathbb{R}_K$ respectively.
  Given a basis element $[x,b)$ for $\T$, there is no open interval in $\T '$ that contains $x$ and lies in $[x,b)$.
  On the other hand, give a basis element $B = (-1, 1) - K$, there is no interval in $\T$ that contains 0.
\item Consider the following topologies on $\mathbb{R}$
  \begin{enumerate}
  \item[] $\T_1$ = the standard topology
  \item[] $\T_2$ = the topology of $\mathbb{R}_K$
  \item[] $\T_3$ = the finite complement topology
  \item[] $\T_4$ = the upper limit topology, having all sets $(a,b\,]$ as a basis
  \item[] $\T_5$ = the topology having all sets $(-\infty, a) = \{x\, |\, x<a\}$ as a basis
  \end{enumerate}
  Determine, for each of these topologies, which of the others it contains.\n
  \indent Previously, $\T_1 \subset \T_2;\; \T_1 \subset \T_4$, as the proof for the upper limit topology is the same as the one for the lower limit topology.\n
  Comparing $\T_1$ and $\T_3$: the open interval $(0,1)$ is a basis element of $\T_1$, thus it is a member of $\T_1$, however it is not a member of the finite complement topology, as $\mathbb{R}-(0,1)$ is infinite.
  On the other hand, an open set $U$ of $\T_3$ exists such that $\mathbb{R} - U$ is finite, i.e. $\mathbb{R} - U = \{a_1, a_2, a_3, ... , a_n\}$, with $a_i < a_{i+1}$ The open sets between any $a_i$ and $a_{i+1}$ are in $\T_1$. Thus, $\T_1\supset\T_3$.\hspace{5in}\n 
  Finally, comparing $\T_1$ and $\T_5$: given a basis element $(a, b)$ for $\T_1$, there is clearly no basis element in $\T_5$ that lies in $(a,b)$. On the other hand, given a basis element $(-\infty, c)$ for $\T_5$ and a point $x\in(-\infty, c)$, there clearly exists an open set that cointains $x$ and lies in $(-\infty, c)$. Thus, $\T_1\supset \T_5$.\hspace{5in} \n
  \indent Comparing, $\T_2$ and $\T_4$: given a basis element $(a, b]$ for $\T_4$, there is no open set in $\T_2$ that lies in $(a,b]$. However, given an open interval there is clearly a set in the basis $\T_4$ that lies within it, and given $x\in(a,b)-K = U \subset \T_2$, there exists a set $A\subset U$ which contains $x$ and is in the basis of $\T_4$. If $x>1$ or $x<0$, this is obvious. Otherwise, set $t$ equal to the greatest value $1/n$ such that $t<x$. If $t>a$ use the interval define $A=(t,x]$, otherwise $A=(a,x]$. There are no elements of $K$ in $A$. Thus, $\T_2 \subset \T_4$\n 
  \indent What remains is to compare $\T_3$ and $\T_5$. None of the elements of $\T_5$ are such that their complement with respect to the reals is finite, these two topologies are uncomparable.\n
  Overall, $\T_3,\T_5\subset\T_1\subset\T_2\subset\T_4$; $T_3$ and $T_5$ are uncomparable.
\item
  \begin{enumerate}
  \item Apply Lemma 13.2 to show that the countable collection $$\mathcal{B}=\{(a,b)\,|\, a<b, a,b \text{ are rational}\}$$ is a basis that generates the standard topology on $\mathbb{R}$.\n
    \indent For each $x$ in each open set $U$ of the topology on $\mathbb{R}$, there exist rational $a,b$ such that $x\in (a,b)\in U$. One can always choose a rational value between any two reals, i.e. between $a$ and $x$ and between $x$ and $b$. Thus $\mathcal{B}$ is a basis for $\mathbb{R}$.
  \item Show that the collection $$\mathcal{C}=\{[a,b)\,|\, a<b, a,b \text{ are rational}\}$$ is a basis that generates a topology different from the lower limit topology on $\mathbb{R}$. \n
    \indent Given the open set $[a,b)$, where $a$ is an irrational value, there exists no interval in $\mathcal{C}$ that lies in $[a,b)$ and contains $a$. Thus $\mathcal{C}$ does not generate the lower limit topology on $\mathbb{R}$.
  \end{enumerate}
\end{enumerate}

\end{document}

\textbf{Chapter 2.16} \label{sec:chapter2.16}
\documentclass[12pt,letterpaper]{article}
\usepackage[pdftex]{graphicx}
\usepackage{alltt}
\usepackage[margin=1in]{geometry}
\usepackage{amsmath, amsthm, amssymb}
\usepackage{verbatim}
\usepackage{ragged2e}
\usepackage{enumitem}
\usepackage{xfrac}
\setlist{parsep=0pt,listparindent=\parindent}
\setlength{\RaggedRightParindent}{\parindent}
\newcommand{\degree}{\ensuremath{^\circ}}
\newcommand{\n}{\break}
\let\oldemptyset\emptyset
\let\emptyset\varnothing
\newcommand{\Wlog}{without loss of generality}
\newcommand{\WLOG}{Without loss of generality}

%%%%%%%%%%%%%%%%%%%%%%%%%%%%%%%%%%%%%%%%%%%%%%%%%%%%%
%THIS DOCUMENT ONLY%
\newcommand{\T}{\ensuremath{\mathcal{T}}}
%%%%%%%%%%%%%%%%%%%%%%%%%%%%%%%%%%%%%%%%%%%%%%%%%%%%%

\begin{document}
\RaggedRight
\begin{enumerate}
  \setcounter{enumi}{7}
  \item If $L$ is a straight line in the plane, describe the topology $L$ inherits as a subspace of $\mathbb{R}_l \times \mathbb{R}$ and as a subspace of $\mathbb{R}_l\times\mathbb{R}_l$ \hspace{5in}.\n
  \indent Let $L$ be the set of all points $x\times \alpha x+\beta$, for some real $\alpha, \beta$. Then, the interval $[a,b]\times [a\alpha+\beta, b\alpha+\beta] = [a\times\alpha a+\beta, b\times\alpha b + \beta]$ is an interval of $L$. Any open interval of $L$ is a subset of a closed interval of this form. \n
  $\mathbb{R}_l\times\mathbb{R}$ has a basis consisting of all intervals $[x_1,x_2) \times (y_1,y_2) = \{x\times y \;|\; x_1\leq x < x_2 \wedge y_1<y<y_2\}$. Now consider a set of the subspace topology
  $$U = [x_1, x_2) \times (y_1, y_2) \cap [a,b)\times(a\alpha+\beta, b\alpha+\beta)$$ The intervals of $L$ are written like this because $L$ must be a subset of $\mathbb{R}_l\times\mathbb{R}$.
  Now there are several cases.
  \begin{enumerate}
    \item[case 1] $\alpha>0$. Then, each written interval on the line is a basis for the subspace topology, forming the topology $\mathbb{R}_l$.
    \item[case 2] $\alpha=0$. By taking $y_1<\beta<y_2$, one easily all sets of the line, forming $\mathbb{R}_l$.
    \item[case 3] $\alpha<0$. Then one must actually consider the line as $[a,b)\times(b\alpha+\beta, a\alpha+\beta)$, again forming $\mathbb{R}_l$
    \item[case 4] The line is vertical. One includes the sole $x$ value as one included the $y$ value in a horizontal line, generating $\mathbb{R}$.
  \end{enumerate}
  Using $\mathbb{R}_l\times\mathbb{R}_l$, the procedure is the same, except that just as horizontal lines had the topology $\mathbb{R}_l$, vertical lines also do here.
  \item Show that the dictionary order topology $\T_o$ on the set $\mathbb{R}\times\mathbb{R}$ is the same as the product topology $\T_p$ on $\mathbb{R}_d\times\mathbb{R}$, where $\mathbb{R}_d$ denotes $\mathbb{R}$ in the discrete topology. Compare this topology with the standard topology on $\mathbb{R}^2$.\n
  \indent Consider a basis interval of $\T_o$: $(a\times b, c\times d)$, and a point $x\times y$ that lies in this interval. There exists a basis element $[x,x] \times (b,d)$ of $\T_p$ that lies in the basis element of $\T_o$. Likewise when $a=c$.
  On the other hand, given a basis element $[a,c]\times (b,d)$ of $\T_p$ and a point $x \times y$ in this interval, there is a basis element of $\T_o$ containing this point that lies in the given basis interval. The nontrivial case here is the point $a \times y$. This lies in the basis set $(a\times b, a\times d)$ of the order topology (a vertical line with $x=a$).
  Since these two topologies are each finer than the other, they are equal. \n
  These topologies are finer than the standard topology, since given a basis element $[a,c]\times (b,d)$ of $\T_p$, there does not exist a basis element of the standard topology that contains the point $a\times y$.
\end{enumerate}

\end{document}

\textbf{Chapter 2.17} \label{sec:chapter2.17}
\documentclass[12pt,letterpaper]{article}
\usepackage[pdftex]{graphicx}
\usepackage{alltt}
\usepackage[margin=1in]{geometry}
\usepackage{amsmath, amsthm, amssymb}
\usepackage{verbatim}
\usepackage{ragged2e}
\usepackage{enumitem}
\usepackage{xfrac}
\setlist{parsep=0pt,listparindent=\parindent}
\setlength{\RaggedRightParindent}{\parindent}
\newcommand{\degree}{\ensuremath{^\circ}}
\newcommand{\n}{\break}
\let\oldemptyset\emptyset
\let\emptyset\varnothing
\newcommand{\Wlog}{without loss of generality}
\newcommand{\WLOG}{Without loss of generality}
\usepackage{accents}
\let\thinbar\bar
\newcommand\thickbar[1]{\accentset{\rule{.4em}{.8pt}}{#1}}
\let\bar\thickbar

%%%%%%%%%%%%%%%%%%%%%%%%%%%%%%%%%%%%%%%%%%%%%%%%%%%%%
%THIS DOCUMENT ONLY%
\newcommand{\T}{\ensuremath{\mathcal{T}}}
%%%%%%%%%%%%%%%%%%%%%%%%%%%%%%%%%%%%%%%%%%%%%%%%%%%%%

\begin{document}
\RaggedRight
\begin{enumerate}
  \item Let $\mathcal{C}$ be a collection of subsets of the set $X$. Suppose that $\emptyset$ and $X$ are in $\mathcal{C}$, and that finite unions and arbitrary intersections of elements of $\mathcal{C}$ are in $\mathcal{C}$. Show that the collection $$\T = \{X-C\; |\; C\in\mathcal{C}\}$$ is a topology on $X$.\n
  \indent See Theorem 17.1, proof proceeds similarly, just looking at the opposite side of the equations.
  \item Show that if $A$ is closed in $Y$ and $Y$ is closed in $X$, then $A$ is closed in $X$.\n
  \indent Since $A$ is closed in $Y$, by theorem 17.2, there exists a set $D$ closed in $X$ such that $A=Y\cap D$.
  $Y$ and $D$ are both closed in $X$, and by Theorem 17.1 intersections of closed sets are closed, so $Y\cap D$ is closed, hence $A$ is closed in $X$.
  \item Show that if $A$ is closed in $X$ and $B$ is closed in $Y$, then $A\times B$ is closed in $X\times Y$.\n
  \indent Consider the set $\pi_1^{-1}(X-A)$. This is equivalent to the open set $(X-A)\times Y$, which is open since both $X-A$ and $Y$ are open. Similarly, $\pi_2^{-1}(Y-B)$ is open. The union of these is an open set, since the union of any two open sets is open.
  $$((X-A)\times Y)\cup (X\times (Y-B))$$
  which is precisely the complement of $A\times B$.
  \item Show that if $U$ is open in $X$ and $A$ is closed in $X$, then $U-A$ is open in $X$, and $A-U$ is closed in $X$.\n
  \indent $U,A\subset X$, therefore $U-A = U\cap(X-A)$. $U$ and $X-A$ are both open in $X$, so their intersection is open, so $U-A$ is open.\hspace{5in}\n
  Similarly, $A-U=A\cap (X-U)$. $X-U$ and $A$ are both closed sets, thus their intersection is closed, so $A-U$ is closed. \hspace{5in}
  \item Let $X$ be an ordered set in the order topology. Show that $\overline{(a,b)} \subset [a,b]$. Under what conditions does equality hold? \n
  \indent If $X$ has a minimum or a maximum, denote the minimum by $m$ and the maximum by $n$. Then $X-(a,b) = [m,a) \cup (b, n]$. Otherwise, $X-(a,b)=(-\infty,a]\cup[b,\infty)$. Generally, neither of these sets are open, as an open set in the order topology is an open interval, except at the endpoints. Thus the closure includes $a$ and $b$. The only case where it does not is when $a$ has an immediate successor and $b$ has an immediate predecessor. Then the set $[m,a]$ can be written as $[m,a+1)$, likewise with $b$. The union of these is open, thus in this case $(a,b)$ is it's own closure. Therefore, $\overline{(a,b)}=[a,b]$ if $a$ has no immediate successor and $b$ has no immediate predecessor.\n
  \item Let $A, B$, and $A_a$ denote subsets of a space $X$. Prove the following: 
  \begin{enumerate}
    \item If $A\subset B$, then $\bar{A}\subset\bar{B}$. 
    \item $\overline{A\cup B} = \bar{A}\cup \bar{B}$.
    \item $\overline{\bigcup A_a} \supset \bigcup \bar{A}_a$; give an example where equality fails.
  \end{enumerate}
  Solutions:
  \begin{enumerate}
    \item Proceeding by contradiction, $\bar{A}$ contains at least one element $x$ which is not an element of $\bar{B}$, implying by theorem 17.5 that every open set containing $x$ interesects $A$. Since $A\subset B$, then every open set containing $x$ intersects $B$, implying that $x\in B$, a contradiction.
    \item Consider $x\subset \bar{A}\cup\bar{B}$. Every open set that contains $x$ intersects $A$ or $B$, equivalently it intersects either $A\cup B$. Thus, it is in $\overline{A\cup B}$. Therefore, $\bar{A}\cup\bar{B} \subset \overline{A\cup B}$.\n
    \indent Consider the other direction. Let $C=A\cup B$. Then $\bar{C}\subset\bar{A}\cup\bar{B}$, so $C\cup C' \subset A \cup B \cup A' \cup B'$, implying that $C' \subset A' \cup B'$ is equivalent to what is to be proven. By contradiction, suppose $x$ is an element of $C'$ but neither $A'$ nor $B'$. Presume $x$ is in neither $A$ nor $B$, as that is a trivial case. Then there exists a neighborhood $U$ of $x$ in $X$ that intersects $A$ but not $B$, and a neighborhood $V$ of $X$ that intersects $B$ but not $A$. Consider the intersection of these neighboorhoods, $U\cap V$. This is an open set containing $X$ an no elements of either $A$ or $B$, thus it does not intersect $A\cup B$, so $x$ is not in $\bar{C}$, so $x$ is not in $C'$, a contradiction. Thus $\overline{A\cup B}\subset \bar{A}\cup\bar{B}$, implying that $\overline{A\cup B}=\bar{A}\cup\bar{B}$.
    \item If $x$ is in $\bigcup \bar{A}_a$, then every open set that contains $x$ intersects some element $A_a$, thus it intersects $\bigcup A_a$, thus it is in $\overline{\bigcup A_a}$, thus $\overline{\bigcup A_a} \supset\bigcup\bar{A}_a$. For a case where equality fails, consider the case when $\bigcup A_a$ is a union of infinitely many sets. Then there can exist an $x$ such that all of its neighborhoods intersect $\bigcup A_a$, but not all of its neighborhoods intersect any particular $A_a$. For example, $A_n=\{1/n\}$: $\overline{\bigcup A_a}$ contains 0.
  \end{enumerate}
  \item Criticize the following ``proof'' that $\overline{\bigcup A_a} \subset \bigcup \bar{A}_a$: if $\{A_a\}$ is a collection of sets in $X$ and if $x\in\overline{\bigcup A_a}$, then every neighborhood $U$ of $x$ intersects $\bigcup A_a$. Thus $U$ must intersect some $A_a$, so that $x$ must belong to the closure of some $A_a$. Therefore, $x\in\bigcup\bar{A_a}$.\n
  \indent $U$ must intersect \emph{some} $A_a$, but not every $U$ intersects the same $A_a$. 
  \item Let $A, B$, and $A_a$ denote subsets of a space $X$. Determine whether the following equations hold; if an equality fails, determine whether one of the inclusions $\supset$ or $\subset$ holds.
  \begin{enumerate}
    \item $\overline{A\cap B} = \bar{A}\cap\bar{B}$.\hspace{5in}\n
    \indent Consider an element $x$ of $\overline{A \cap B}$. Every neighborhood of $x$ intersects $A\cap B$, thus intersecting both $A$ and $B$, thus $x$ is in both $\bar{A}$ and $\bar{B}$. Therefore, $\overline{A\cap B}\subset \bar{A}\cap\bar{B}$. In the other direction, consider $x$ in $\bar{A}\cap\bar{B}$. Specifically, consider an $x$ that is not in $A$ or $B$, but is in both $\bar{A}$ and $\bar{B}$. Such an element might not be in $\overline{A\cap B}$. For example, consider $A=\mathbb{R}_+$ and $B=\mathbb{R}_-$. Both closures contain $0$, but the intersection of these sets is empty. Thus equality does not hold, only $\overline{A\cap B}\subset \bar{A}\cap{B}$.
    \item $\overline{\bigcap A_a} = \bigcap\bar{A}_a$.\hspace{5in}\n
    \indent The same argument as above applies. $\overline{\bigcap A_a}\subset\bigcap\bar{A}_a$.
    \item $\overline{A-B}=\bar{A}-\bar{B}.$ \hspace{5in}\n
    \indent Consider an $x$ that lies in $\bar{A}$, $\bar{B}$, and $\overline{A-B}$. This $x$ won't exist in $\bar{A}-\bar{B}$. Once again, the example of $A=\mathbb{R}_+$, $B=\mathbb{R}_-$ illustrates this. The difference of the closures is just $\mathbb{R}_+$, while the closure of the difference is $\bar{\mathbb{R}}_+=\mathbb{R}_+ + \{0\}$. Thus at best $\overline{A-B}\supset\bar{A}-\bar{B}$. To prove this direction, consider an $x$ in $\bar{A}-\bar{B}$. All neighborhoods of $x$ intersect $A$, but there exists some neighborhood $U$ that does not intersect $B$. Suppose there is a neighborhood $V$ that does not intersect $A-B$. Then $U\cap V$ must intersect $A$, since it's a neighborhood of $x$. Since it does not intersect $A-B$, the only way it could intersect $A$ is if it intersected $B$, i.e. if it intersected $A\cap B$. But $U$ does not intersect $B$, a contradiction. Thus we conclude that $\overline{A-B}\supset\bar{A}-\bar{B}$.
  \end{enumerate}
  \item Let $A\subset X$ and $B\subset Y$. Show that in the space $X\times Y$, $\overline{A\times B}=\bar{A}\times\bar{B}$.\n
  \indent Consider an $x_0\times y_0\in \bar{A} \times \bar{B}$. Every neighborhood of $x_0$ must intersect $A$, and every neighborhood of $y_0$ must intersect $B$. Identically, for an element $x_1\times y_1\in\overline{A\times B}$, every neighborhood of $x_1$ intersects $A$ and every neighborhood of $y_1$ intersects $B$. 
  \item Show that every order topology is Hausdorff.\n
  \indent Consider the pair of distinct points $a,b$ in a topological space $X$. \WLOG, assume $a<b$. Then either there exists a point $c$ such that $a<c<b$, or $b$ is the immediate successor of $a$. In the first case, the neighborhoods $(-\infty, c)$ and $(c,\infty)$ are disjoint neighborhoods of $a$ and $b$ respectively, so the space is Hausdorff. In the second case, $(-\infty, b)$ and $(a,\infty)$ are the disjoint neighborhoods. 
  \item Show that the product of two Hausdorff spaces $X$ and $Y$ is Hausdorff.\n
  \indent For every $x_1,x_2$ in $X$ there exist disjoint neighborhoods $U_1$ and $U_2$. Similarly, for every $y_1,y_2$ in $Y$ there exist disjoint neighborhoods $V_1$ and $V_2$. Thus in the product, for every $x_1\times y_1$ and $x_2\times y_2$ in $X\times Y$, there exist disjoint neighborhoods $U_1\times V_1$ and $U_2\times V_2$.
  \item Show that the subspace $A$ of a Hausdorff space $X$ is Hausdorff.\n
  \indent For every $x_1,x_2$ in $X$ there exist disjoint neighborhoods $U,V$. For such elements, the neighborhoods $U\cap A, V\cap A$ will be disjoint in the subspace topology.
  \item Show that $X$ is Hausdorff if and only if the diagonal $\Delta = \{x\times x\; |\; x\in X\}$ is closed in $X\times X$.\hspace{5in}\n
  \indent If $X$ is Hausdorff, for every pair of distinct points $x_1,x_2\in X$, they have disjoint neighborhoods $U,V$. Clearly $x_1 \times x_2$ is in the open set $U\times V$. Furthermore, $U\cap V=\emptyset$ implies that $U\times V \cap \Delta=\emptyset$, since if the latter intersection was not empty, that would imply there was an element $c\times c$ in $U\times V$, i.e. $c\in U$ and $c\in V$, which contradicts the previous assertion. Since every element of $X\times X - \Delta$ lies in an open set, $X\times X - \Delta$ is open, so $\Delta$ is closed.\n
  \indent In the other direction proceeds similarly: if the diagonal is closed, then it's complement is open, i.e. every pair of distinct points $x_1,x_2$ lies in an open set $U\times V$. $x_1\in U$ and $x_2\in V$. If $U\cap V\neq \emptyset$, then there exists an element $c$ such that $c\in U$ and $c\in V$, so $c\times c\in U\times V$, but $c\times c\in \Delta$, a contradiction as $U$ and $V$ are subsets of the complement of $\Delta$. Thus $U,V$ are disjoint, so $X\times X$ is Hausdorff.
  \item In the finite complement topology on $\mathbb{R}$ to what point or points does the sequence $x_n=1/n$ converge?\hspace{5in}\n
  \indent The finite complement topology on $\mathbb{R}$ is defined such that all finite sets are closed. Consider a finite set $A\in\mathbb{R}$. For any $N$, the set of $x_{n>N}$ is infinite, so there must be a smallest $x_n$ that $A$ contains (if it contains any of the $x_n$s), and $A$ will not contain any of the $x_n$ below that point. Therefore the complement of $A$ will be an open set that contains all $x_{n>N}$. This describes every open set, so every this sequence converges to every real.
  \item Show that the $T_1$ axiom is equivalent to the condition that for each pair of points of $X$, each has a neighborhood not containing the other. ($T_1$ axiom: all finite point sets are closed). \n
  \indent Suppose that $X$ is a space satisfying the $T_1$ axiom. Then, for any pair of points $a,b$, the sets $\{a\}$ and $\{b\}$ are closed. The complement of $\{a\}$ is a neighborhood of $a$ that does not contain $b$, and the complement of $\{b\}$ is a neighborhood of $b$ that does not contain $a$.\n
  \indent In the other direction, suppose that each pair of points $a,b$ has a neighborhood, $U,V$ respectively, that does not contain the other point. The corresponding closed sets are $\mathbb{R}-U\subset \mathbb{R}-\{a\}$, with the same format for $V$. Now, for every $c$ in that closed set except $c=b$, take the closed set $C$ that is the complement of the open set that does not contain $b$. The intersection of each such $C$ with $U$ will yield a closed set $\{b\}$. Thus every one point set is closed, and since finite unions are allowed, every finite set is closed.
  \item Consider the following five topologies on $\mathbb{R}$: $\T_1$ - the standard topology, $\T_2$ - the topology of $\mathbb{R}_K$, $\T_3$ - the finite complement topology, $\T_4$ - the upper limit topology, and $\T_5$ - the topology having all sets $(-\infty,a)$ as a basis. For each one, 
  \begin{enumerate}
    \item Determine the closure of the set $K=\{1/n\;|\;n\in\mathbb{Z}_+\}$
    \item Is it Hausdorff? Does it satisfy the $T_1$ axiom?
  \end{enumerate}
  \begin{enumerate}[label=\roman*)]
    \item $\T_1$ - standard topology
    \begin{enumerate}[label=(\alph*)]
      \item $\bar{K}=K\cup \{0\}$. Every neighborhood of 0 intersects $K$ at a point other than 0, and the next part shows that there is only one limit point of the sequence $x_n=1/n$.
      \item The standard topology is Hausdorff. For any $a$ and $b$, $a<b$ \Wlog . Then, the intervals $(a-1, (a+b)/2)$ and $((a+b)/2, b+1)$ are disjoint open neighborhoods of $a,b$ respectively.
    \end{enumerate}
    \item $\T_2$ - $\mathbb{R}_K$
    \begin{enumerate}[label=(\alph*)]
      \item $K$ is its own closure. 0 is not a limit point, since the set $(-1, 1) - K$ is a neighborhood of 0 that does not intersect $K$.
      \item This is finer than the standard topology, $\T_1$ is Hausdorff, so this is too.
    \end{enumerate}
    \item $\T_3$ - the finite complement topology
    \begin{enumerate}[label=(\alph*)]
      \item all reals, by exercise 14.
      \item $T_1$, in text. Also, by previous part obviously not Hausdorff, and $T_1$ by definition.
    \end{enumerate}
    \item $\T_4$ - the upper limit topology
    \begin{enumerate}[label=(\alph*)]
      \item The set is its own closure. 0 is not a limit point because the interval $(-1,0]$ is one of its neighborhoods.
      \item Finer than $\T_1$, which is Hausdorff, so this is Hausdorff.
    \end{enumerate}
    \item $\T_5$ - the topology having all sets $(-\infty, a)$ as a basis
    \begin{enumerate}[label=(\alph*)]
      \item $\bar{K}=[0,\infty)$, since for all points $x$ in that interval, all of their open neighborhoods must include an element larger than $x$, and all elements smaller than $x$. 
      \item Given points $a,b$ suppose, \Wlog , that $a<b$. All neighborhoods of $b$ include everything less than $b$, including $a$. This space does not follow the $\T_1$ axiom, because the set $\{a\}$ is not closed for any $a$. 
    \end{enumerate}
  \end{enumerate}
  \item Consider the lower limit topology on $\mathbb{R}$ and the topology generated by the basis $\mathcal{C}=\{[a,b)\;|\; a<b,\; a,b \text{ are rational}\}$. Determine the closures of the intervals $A=(0,\sqrt{2})$ and $B=(\sqrt{2},3)$ in these topologies.
  \begin{enumerate}
    \item Lower limit topology
    \begin{enumerate}
      \item $\bar{A} = A\cup \{0\}$. The neighborhood $\left[\sqrt{2},2\right)$ of $\sqrt{2}$ does not lie in $A$.
      \item $\bar{B} = B\cup \{\sqrt{2}\}$.
    \end{enumerate}
    \item The topology generated by $\mathcal{C}$.
    \begin{enumerate}
      \item $\bar{A} = A\cup \{0,\sqrt{2}\}$, since $\left[\sqrt{2}, b\right)$ is not an open set.
      \item $\bar{B} = B\cup \{\sqrt{2}\}$.
    \end{enumerate}
  \end{enumerate}
  \item Determine the closures of the following subsets of the ordered square $I^2_o = [0,1]\times [0,1]$
  \begin{enumerate}
    \item $A=\{(1/n)\times 0 \;|\; n\in\mathbb{Z}_+\}$\hspace{5in}\n
    \indent $A\cup \{0\times 0\}$
    \item $B=\{(1-1/n)\times \sfrac{1}{2}\;|\;n\in\mathbb{Z}_+\}$\hspace{5in}\n
    \indent $B\cup \{1\times \sfrac{1}{2}\}$
    \item $C=\{x\times 0 \;|\; 0<x<1\}$\hspace{5in}\n
    \indent $[0,1]\times 0$
    \item $D=\{x\times \sfrac{1}{2} \;|\; 0<x<1\}$\hspace{5in}\n
    \indent $[0,1]\times \sfrac{1}{2}$
    \item $E=\{\sfrac{1}{2} \times y \;|\; 0<y<1\}$\hspace{5in}\n
    \indent $[\sfrac{1}{2}\times 0, \sfrac{1}{2}\times 1]$
  \end{enumerate}
  \item If $A\subset X$ we define the boundary of $A$ by the equation $\text{Bd } A = \bar{A} \cap \overline{(X-A)}$
  \begin{enumerate}
    \item Show that Int $A$ and Bd $A$ are disjoint, and $\bar{A}=\text{Int } A\ \cup\ \text{Bd } A$.\hspace{5in}\n
    \indent Suppose there exists an element $x$ that is in both Int $A$ and Bd $A$. Then $x$ lies in an open set $U$ of $X$ that is a subset of $A$.
    $x$ must also lie within $\overline{(X-A)}$. $x$ is an element of $A$, so it cannot be in $X-A$, so it must be a limit point of $X-A$, i.e. every neighborhood of $x$ intersects $X-A$ at a point other than itself. However, $U$ is a subset of $A$, and thus is a neighborhood of $x$ that does not intersect $X-A$, a contradition. Thus the two are disjoint.\n
    \indent Bd $A = \bar{A} \cap \overline{(X-A)}$, therefore $\bar{A} = (X - \overline{(X-A)})\cup \text{Bd } A$. $X - \overline{(X-A)} = (X - (X - A)) \cap (X - (X-A)') = A \cap (X-(X-A)') = (A\cap X) - (A\cap (X-A)') = A - (A\cap(X-A)') = A - (X-A)'$\hspace{3in}\n
    This is equivalent to Int $A$. Obviously no points outside $A$ are in Int $A$. Now, suppose that there exists an $x$ in Int $A$ that is also a limit point of $X-A$, i.e. every neighborhood of $x$ intersects $X-A$ at a point other than $x$, and $x$ lies in an open set of $A$. The open set of $A$ that $x$ must lie in to be in its interior does not intersect $X-A$. Thus, Int $A = A - (X-A)'$ and $\bar{A} = \text{Int } A \cup \text{Bd } A$
    \item Show that Bd $A = \emptyset \leftrightarrow A$ is both open and closed.\hspace{5in}\n
    \indent Bd $A = \bar{A} \cap \overline{(X-A)} = (A \cup A') \cap ((X-A)\cup (X-A)') = (A\cap (X-A)') \cup (A'\cap (X-A)) \cup (A' \cap (X-A)')$ For any $U$, $U$ and $U'$ are disjoint, so the previous expression is equal to $(X-A)' \cup A' \cup (A'\cap (X-A)') = (X-A)' \cup A' = \text{Bd } A$.\n
    If $(X-A)'\cup A'=\emptyset$, then $A'=\emptyset$, implying that the set is closed. Furthermore, this implies that $\bar{A}=A$. In the previous part, we found $\bar{A}=\text{Int } A \cup \text{Bd } A$, so Int $A=A$, implying that $A$ is open. Thus $A$ is open and closed.\n
    \indent In the other direction, suppose $A$ is both open and closed. Then Int $A = A = \bar{A}$. Since $\bar{A} = $ Int $A\ \cup\ $Bd $A$, and since Int $A$ and Bd $A$ are disjoint, Bd $A$ must be empty.
    \item Show that $U$ is open $\leftrightarrow$ Bd $U = \bar{U} - U$.\hspace{5in}\n
    \indent $\bar{U} - U = U'$ and Bd $U = U' \cup (X-U)'$. So the hypothesis is equivalent to $U$ is open $\leftrightarrow (X-U)'\subset U'$. These two sets are disjoint (suppose there exists an $x$ in both. $x$ is not in $X-U$ and $x$ is not in $U$, a contradiction.) So the hypothesis is $U$ is open $\leftrightarrow (X-U)'=\emptyset$.\hspace{5in}\n
    If $U$ is open, then $(X-U)$ is closed, so it is its own closure, so $(X-A)'=\emptyset$.
    In the other direction, if $(X-U)'=\emptyset$, then $(X-U)$ is it's own closure, so it is closed, so $U$ is open.
    \item If $U$ is open, is it true that $U = \text{Int } \bar{U}$?\hspace{5in}\n
    \indent No. Consider $\mathbb{R} - \{0\}$. It's an open set in the standard topology, but the interior of the closure is the interior of $\mathbb{R}$ is just $\mathbb{R}\neq \mathbb{R}-\{0\}$.
  \end{enumerate}
  \item Find the boundary and the interior of each of the following subsets of $\mathbb{R}^2$
  \begin{enumerate}
    \item $A=\{x\times 0\}$\hspace{5in}\n
    \indent Bd $A$ = $A$. Int $A = \emptyset$
    \item $B=\{x\times y\;|\; x>0 \wedge y\neq 0\}$\hspace{5in}\n
    \indent Bd $B = \{x\times 0 \cup 0\times y \cup 0\times 0\}$. Int $B = B$.
    \item $C=A\cup B$\hspace{5in}\n
    \indent Bd $C = \{x\times 0\;|\; x<0 \cup 0\times y \}$. Int $C = \{x\times y\;|\; x>0\}$.
    \item $D=\{x\times y\;|\;x\in\mathbb{Q}\}$\hspace{5in}\n
    \indent Bd $D = \{x\times y\}$. Int $D = \emptyset$
    \item $E=\{x\times y\;|\;0<x^2-y^2\leq 1\}$\hspace{5in}\n
    \indent Bd $E = \{x\times y\;|\; x^2=y^2 \vee x^2=1+y^2$. Int $E=\{x\times y\;|\; 0<x^2-y^2<1\}$.
    \item $F=\{x\times y\;|\;x\neq 0 \wedge y\leq \frac{1}{x}\}$\hspace{5in}\n
    \indent Bd $F=\{x\times y\;|\; x=0 \vee y=\frac{1}{x}\}$. Int $F=\{x\times y\;|\; x\neq 0 \wedge y<\frac{1}{x}\}$.
  \end{enumerate}
  \item (Kuratowski) Consider the collection of all subsets $A$ of the topological space $X$. The operations of closure and complementation are functions from this collection to itself.
  \begin{enumerate}
    \item Show that starting from a given set $A$, one can form no more than 14 distinct sets by applying these operations successively.\n
    \indent Taking either operation twice on the same set obviously yields the original set.\n
    \indent The sets $A$, $X-A$, $\bar{A}$, $\overline{X-A}$ and $X-\bar{A}$ can obviously be distinct. For example the with the set $(0,1]$, these five resultant sets are distinct. Furthermore, applying these operations to one of the first three sets does not give a distinct sets, nor does the closure of the fourth, nor does the complement of the fifth.
    So, currently we need to examine $X-\overline{X-A}$ and $\overline{X-\bar{A}}$. The former, with the same example, yields a new set. The latter is identical to $\overline{X-A}$ in this case. \n
    So far, the sets we have, in order, with the running example are $(0,1]; (-\infty,0]\cup (1,\infty); [0,1]; (-\infty,0]\cup [1,\infty); (-\infty,0)\cup(1,\infty); (0,1)$.
    We now try $\overline{X-\overline{X-A}}$. This is identical to $\bar{A}$ in this case.
    %(2,3)\cup (3,4) is interesting, since the complement has a finite closed set {3}. Maybe that'll come to something. 
    \item Find a subset $A$ of $\mathbb{R}$ (in its usual topology) for which the maximum of 14 is reached.
  \end{enumerate}
\end{enumerate}
\end{document}

\textbf{Chapter 2.18} \label{sec:chapter2.18}
\documentclass[12pt,letterpaper]{article}
\usepackage[pdftex]{graphicx}
\usepackage{alltt}
\usepackage[margin=1in]{geometry}
\usepackage{amsmath, amsthm, amssymb}
\usepackage{verbatim}
\usepackage{ragged2e}
\usepackage{enumitem}
\usepackage{xfrac}
\setlist{parsep=0pt,listparindent=\parindent}
\setlength{\RaggedRightParindent}{\parindent}
\newcommand{\degree}{\ensuremath{^\circ}}
\newcommand{\n}{\break}
\let\oldemptyset\emptyset
\let\emptyset\varnothing
\newcommand{\Wlog}{without loss of generality}
\newcommand{\WLOG}{Without loss of generality}
\usepackage{accents}
\let\thinbar\bar
\newcommand\thickbar[1]{\accentset{\rule{.4em}{.8pt}}{#1}}
\let\bar\thickbar

%%%%%%%%%%%%%%%%%%%%%%%%%%%%%%%%%%%%%%%%%%%%%%%%%%%%%
%THIS DOCUMENT ONLY%
\newcommand{\T}{\ensuremath{\mathcal{T}}}
%%%%%%%%%%%%%%%%%%%%%%%%%%%%%%%%%%%%%%%%%%%%%%%%%%%%%

\begin{document}
\RaggedRight
\begin{enumerate}


  
  \item Show that the subspace $(a,b)$ of $\mathbb{R}$ is homeomorphic with $(0,1)$ and the subspace $[a,b]$ of $\mathbb{R}$ is homeomorphic with $[0,1]$.\n
  \indent Consider the function from $(a,b)$ to $(0,1)$: $f(x)=\frac{x-a}{b-a}$. The inverse is $f^{-1}(x)=x*(b-a)+a$. That this is the inverse and that $f(f^{-1}(x)) = x$ and $f^{-1}(f(x))=x$ is easily checked, so $f$ is bijective. For any open set $U$ of $(a,b)$, $f(U)$ will be open, the converse also holds, so it is homeomorphic. The same function will work for the second example.
  \item Find a function $f: \mathbb{R}\rightarrow \mathbb{R}$ that is continuous at precisely one point.\hspace{5in}\n
  \indent Consider the function $f(x) = \begin{cases} x & \text{if } x\in\mathbb{Q} \\ 0 & \text{if } x\not\in\mathbb{Q}\end{cases}\quad$ Then, $f$ is continuous only at the point 0. \hspace{3in}\n
  Consider a neighborhood $V$ of $f(0)$ of the form $(a,b)$, where $a<0$ and $b>0$. $f^{-1}(V)$ is all of the rationals in the set $(a,b)$ together with all of the irrationals, which clearly contains the set $(a,b)$. So $f$ is continuous at 0. Now consider $f$ at some other point, $x_0$, and a neighborhood $V$ of $f(x_0)$. Suppose that $x_0$ is rational, and that $V$ does not contain 0 - this is possible since any interval $(a,b)$ containing $x_0\neq 0$ can be made smaller until it does not contain 0. Then, $f^{-1}(V)$ does not contain any irrationals, but $V$ does, so $f$ is not continuous at $x_0$. If $x_0$ is irrational, then $f(x_0)=0$, so $V$ is a neighborhood of $0$. Then, consider a $V$ that contains only elements closer to 0 than $x$. Then $f^{-1}(V)$ will not contain any rationals close to $x_0$, so it will not contain any neighborhood of $x_0$, so $f$ is not continuous at $x_0$.
  \item
  \begin{enumerate}
    \item Suppose that $f: \mathbb{R} \rightarrow \mathbb{R}$ is ``continuous from the right,'' that is, $$\lim_{x\rightarrow a^+} f(x)=f(a),$$ for each $a\in\mathbb{R}$. Show that $f$ is continuous when considered as a function from $\mathbb{R}_\ell$ to $\mathbb{R}$.\hspace{5in}\n
    \item Can you conjecture what functions $f:\mathbb{R}\rightarrow \mathbb{R}$ are continuous when considered as maps from $\mathbb{R}$ to $\mathbb{R}_l$? As maps from $\mathbb{R}_\ell$ to $\mathbb{R}_\ell?$ We shall return to this question in chapter 3.
  \end{enumerate}
  \item Let $Y$ be an ordered set in the order topology. Let $f,g: X\rightarrow Y$ be continuous.
  \begin{enumerate}
    \item Show that the set $B=\{x\;|\; f(x)\leq g(x)\}$ is closed in $X$.\hspace{5in}\n
    \indent Define $A=X-B$ For any $x\in A$, for each of its neighborhoods $V_f=f(x)$, $U_f=f^{-1}(V_f)$ is open, and for each of its neighborhoods $V_g=g(x)$, $U_g=g^{-1}(V_g)$ is open. It is sufficient to consider only neighborhoods that are basis elements. Then, since $x\in A$, there exist neighborhoods $V_f$ and $V_g$ such that the smallest element in $V_f$ is greater than the largest element in $V_g$. $U_f$ is an open set containing $x$, and furthermore, since the smallest element in $V_f$ is greater than the largest element in $V_g$, every element in $U_f$ is contained in $A$. The union of open sets is open, so the union of such an open set for every $x$ in $A$ is an open set, this union will yield exactly $A$.
    \item Let $h: X\rightarrow Y$ be the function $h(x)=\min(f(x),g(x))$. Show that $h$ is continuous.\hspace{5in}\n
    \indent Consider $A=\{x\;|\; f(x)\leq g(x)\}$ and $B=\{x\;|\; g(x)\leq f(x)\}$. $A$ and $B$ are closed by the previous part. Clearly, $x\in A\cap B\rightarrow f(x)=g(x)$, by antisymmetry of an order relation. By the pasting lemma, the minimum of these two functions is closed.
  \end{enumerate}
  \item Let $\{A_\alpha \}$ be a collection of subsets of $X$; let $X=\bigcup_\alpha A_\alpha$. Let $f: X\rightarrow Y$; suppose that $f|A_\alpha$ is continuous for each $\alpha$.
  \begin{enumerate}
    \item Show that if the collection $\{A_\alpha\}$ is finite and each set $A_\alpha$ is closed, then $f$ is continuous.\hspace{5in}\n
    \item Find an example where the collection $\{A_\alpha\}$ is countable and each $A_\alpha$ is closed, but $f$ is not continuous. \hspace{5in}\n
    \item An indexed family of sets $\{A_\alpha\}$ is said to be \emph{locally finite} if each point $x$ of $X$ has a neighborhood that intersects $A_\alpha$ for only finitely many values of $\alpha$. Show that if the family $\{A_\alpha\}$ is locally finite and each $A_\alpha$ is closed, then $f$ is continuous.\hspace{5in}\n
  \end{enumerate}
\end{enumerate}
\end{document}

\textbf{Chapter 2.19} \label{sec:chapter2.19}
\documentclass[12pt,letterpaper]{article}
\usepackage[pdftex]{graphicx}
\usepackage{alltt}
\usepackage[margin=1in]{geometry}
\usepackage{amsmath, amsthm, amssymb}
\usepackage{verbatim}
\usepackage{ragged2e}
\usepackage{enumitem}
\usepackage{xfrac}
\setlist{parsep=0pt,listparindent=\parindent}
\setlength{\RaggedRightParindent}{\parindent}
\newcommand{\degree}{\ensuremath{^\circ}}
\newcommand{\n}{\break}
\let\oldemptyset\emptyset
\let\emptyset\varnothing
\newcommand{\Wlog}{without loss of generality}
\newcommand{\WLOG}{Without loss of generality}
\usepackage{accents}
\let\thinbar\bar
\newcommand\thickbar[1]{\accentset{\rule{.4em}{.8pt}}{#1}}
\let\bar\thickbar
\usepackage{standalone}
\usepackage{hyperref}
\newcommand{\R}{\ensuremath{\mathbb{R}}}
\usepackage{mathtools}
\DeclarePairedDelimiter{\ceil}{\lceil}{\rceil}
\DeclarePairedDelimiter{\floor}{\lfloor}{\rfloor}
\DeclarePairedDelimiter\abs{\lvert}{\rvert}
\DeclarePairedDelimiter\norm{\lVert}{\rVert}
%%%%%%%%%%%%%%%%%%%%%%%%%%%%%%%%%%%%%%%%%%%%%%%%%%%%%
%TOPOLOGY DOCUMENTS ONLY%
\newcommand{\T}{\ensuremath{\mathcal{T}}}
%%%%%%%%%%%%%%%%%%%%%%%%%%%%%%%%%%%%%%%%%%%%%%%%%%%%%

\begin{document}
\RaggedRight
\begin{enumerate}
  \item Prove theorem 19.2, which is as follows: Suppose the topology on each space $X_\alpha$ is given by a basis $\mathcal{B}_\alpha$. The collection of all sets of the form $\prod_{\alpha\in J}B_\alpha$ where $B_\alpha \in \mathcal{B}_\alpha$ for each $\alpha$, will serve as a basis for the box toplogy on $\prod_{\alpha\in J} X_\alpha$. The collection of all sets of the same form, where $B_\alpha \in \mathcal{B}_\alpha$ for finitely many indices $\alpha$ and $B_\alpha=X_\alpha$ for all remaining indices, will serve as a basis for the product topology $\prod_{\alpha\in J}X_\alpha$.\hspace{5in}\n
  \indent Let the collection that is to be proven to be a basis for the topology in question over the product space $X=\prod_{\alpha\in J}X_\alpha$ be denoted $\mathcal{C}$. Let $x=(x_\alpha)$ be any point of $\prod X_\alpha$. We show that $\mathcal{C}$ is a basis. \n
  \indent For each $x_\alpha$ there exists a basis element $B_\alpha$ containing it (or $X_\alpha$ obviously contains it), since each $\mathcal{B}_\alpha$ is a basis; call each such element $C_\alpha$. Then $\prod_\alpha C_\alpha$ gets an element of $\mathcal{C}$ which contains $x$. Hence there is at least a basis element containing any $x\in X$.
  Next, suppose $x$ belongs to the intersection of two basis elements, $C_1$ and $C_2$. This is equivalent to $\prod_\alpha C_{1_\alpha} \cap C_{2_\alpha}$. Since each $x_\alpha$ must belong to $C_{1_\alpha} \cap C_{2_\alpha}$, there must be a corresponding basis element $C_{3_\alpha}\in\mathcal{B}_\alpha$ such that $x\in C_{3_\alpha}\subset C_{1_\alpha} \cap C_{2_\alpha}$ since each $\mathcal{B}_\alpha$ is a basis. Then $\prod_\alpha C_{3_\alpha}$ is an element of $\mathcal{C}$. Hence, $\mathcal{C}$ is a basis.\n
  \indent Now we show that the topology generated by $\mathcal{C}$ is the correct topology.
  We show that $\mathcal{C}$ is equivalent to the basis that generally forms the box topology, the collection of all sets of the form $\prod_\alpha U_\alpha$, where $U_\alpha$ is open in $X_\alpha$ for all $\alpha\in J$. Consider an element of the defining basis, $U=\prod_\alpha U_\alpha$. Since each $\mathcal{B}_\alpha$ is a basis, each $U_\alpha$ can be formed by some union of elements of $\mathcal{B}_\alpha$. Hence $U$ can be formed by unions of elements of $\mathcal{C}$. The same argument applies when looking at the basis generated by the defining subbasis of the product topology.
  \item Prove theorem 19.3, which is as follows: Let $A_\alpha$ be a subspace of $X_\alpha$ for each $\alpha \in J$. Then $A=\prod A_\alpha$ is a subsapce of $X=\prod X_\alpha$ if both products are given either the box topology or the product topology.\hspace{5in}\n
  \indent Let $\T_s$ be the topology $A$ inherits from $X$ in the product topology. Hence, $\T_s$ is generated by $\mathcal{B}_s = \{ B \cap A | B\in \mathcal{B} \}$, where $\mathcal{B}$ is the basis of the $X$ with the product topology. Let $\T_p$ be the product topology on $A$, with the basis $\mathcal{B}_p=\prod U_\alpha$, where $U_\alpha$ is open in $A_\alpha$ and equal to $A_\alpha$ except for finitely many values of $\alpha$. \n
  \indent Consider an element $B$ of $\mathcal{B}_s$. $B=\prod B_\alpha \cap A_\alpha$. Since $B_\alpha \cap A_\alpha$ is open in $A_\alpha$, and where $B_\alpha$ is equal to $X_\alpha$, $B_\alpha \cap A_\alpha = A_\alpha$, $B\in \mathcal{B}_p$.\hspace{5in}\n
  \indent Now consider an element $B$ of $\mathcal{B}_p$. Any open element $U_\alpha$ of $A_\alpha$ must be equal to some $V_\alpha \cap A_\alpha$, where $V_\alpha$ is an open set of $A_\alpha$, because $A_\alpha$ is a subspace of $X_\alpha$. Thus, $B=\prod V_\alpha \cap A_\alpha$. For all but a finite number of $\alpha$s, $V_\alpha$ may be $X_\alpha$, since the corresponding $U_\alpha$ is $A_\alpha$ for all but a finite number of $\alpha$s. Hence, $B\in \mathcal{B_s}$, since the corresponding $B_\alpha$ of a $B\in \mathcal{B}$ may be any open set of $X_\alpha$. The proof follows similarly for the box topology.
  \item Prove Theorem 19.4, which is: If each space $X_\alpha$ is a Hausdorff space, then $X=\prod X_\alpha$ is a Hausdorff space in both the box and the product topologies. \hspace{5in} \n
  \indent For any distinct points $x', x'' \in X$ there exist disjoint neighborhoods $U_\alpha', U_\alpha''$ for $x', x''$ respectively, for each $\alpha \in J$ where $x_\alpha' \neq x_\alpha''$, in the other cases, let $U_\alpha'=U_\alpha''$ be any neighborhood of $x_\alpha'$. Thus there exist disjoint neighborhoods $U' = \prod U_\alpha',\; U''=\prod U_\alpha''$, thus $X$ is hausdorff.
  \addtocounter{enumi}{1}
  \item One of the implications stated in Theorem 19.6 holds for the box topology. Which one?\n
  \indent If $f$ is continuous, then $f_\alpha$ is continuous for each $\alpha$.
  \item Let $x_1, x_2, \dots$ be a sequence of points of the product space $X=\prod X_\alpha$. Show that this sequence converges to the point $x$ if and only if the sequence $\pi_\alpha(x_1), \pi_\alpha(x_2), \dots$ converges to $\pi_\alpha(x)$ for each $\alpha$. Is this fact true if one uses the box topology instead of the product topology? \hspace{5in} \n
  \indent Suppose the sequence converges to $x$. Then every neighborhood $U$ of $x$ contains every $x_n$ for all $n$ larger than some $m_U$. Hence, every neighborhood $\pi_\alpha(U)$ of $\pi_\alpha(x)$ contains every $\pi_\alpha(x_n)$ for all $n$ larger than some $m_U$, for each $\alpha$. Furthermore, every neighborhood $V$ of $\pi_\alpha(x)$ has a corresponding neighborhood of $x$: $\pi^{-1}_\alpha(x)$. Thus, since neighborhood $\pi_\alpha(U)$ of $\pi_\alpha(x)$ is every neighorhood $\pi_\alpha(x)$. Therefore, every sequence $\pi_\alpha(x_1), \pi_\alpha(x_2),\dots$ converges to $\pi_\alpha(x)$.\n
  \indent Now suppose every sequence $\pi_\alpha(x_1), \pi_\alpha(x_2)\dots$ converges to $\pi_\alpha(x)$. Any open neighborhood $U_\alpha$ of $\pi_\alpha(x)$ in $X_\alpha$ contains all $\pi_\alpha(x_n)$ greater than some $m_{U_\alpha}$. Use finitely many $U_\alpha$s to construct an open neighborhood of $x$; label these by $i$. $U=\prod U_\alpha$, where $U_\alpha=U_i$ for finitely many values of $\alpha$, and $U_\alpha=X_\alpha$ for the others. Then, for all values $n$ over $\max_\alpha(m_{U_\alpha})$, $U$ will contain $x_n$, thus the sequence converges to $x$.\n
  \indent In the box topology, the second result does not hold, since if the value $U_{m_\alpha}$ increases unboundedly, the final sequence does not converge. For example, in the space $\mathbb{R}^\omega$, consider $x=(0,0,\dots)$, its neighborhood $((-1,1),\allowbreak (-1/2,1/2),\allowbreak (-1/4,1/4),\allowbreak (-1/8,1/8),\allowbreak (-1/16,1/16)\dots)$, and the sequence $x_n = (1/n,1/n,1/n,\dots)$.
  \item Let $\mathbb{R}^\infty$ be the subset of $\mathbb{R}^\omega$ consisting of all sequences that are ``eventually zero,'' that is, all sequences $(x_1, x_2,\dots)$ such that $x_i\neq 0$ for only finitely many values of $i$. What is the closure of $\mathbb{R}^\infty$ in $\mathbb{R}^\omega$ in the box and product topologies?\n
  \indent GO BACK TO THIS, NOT SURE HOW TO APPROACH IT.
  \item Given sequences $(a_1, a_2, \dots)$ and $(b_1, b_2,\dots)$ of real numbers $a_i > 0$ for all $i$, define $h: \mathbb{R}^\omega \rightarrow \mathbb{R}^\omega$ by the equation $h((x_1, x_2, \dots)) = (a_1x_1+b_1, a_2x_2+b_2, \dots)$. Show that if $\mathbb{R}^\omega$ is given the product topology, $h$ is a homeomorphism of $\mathbb{R}^\omega$ with itself. What happens if $\mathbb{R}^\omega$ is given the box topology?\hspace{5in}\n
  \indent First, to prove that $h$ is a homeomorphism, we prove that it is bijective. To do that, we first show that it is injective. Suppose the opposite, there exist distinct $x$ and $y$ such that $h(x)=h(y)$. Then $a_1x_1+b_1 = a_1y_1 + b_1$, so $x=y$, a contradiction. Now we show that $h$ is surjective, again by contradiction. Suppose there is a point $y$ such that there exists no $x$ such that $h(x)=y$. However, the point $x=((y_1-b_1)/a_1, (y_2-b_2)/a_2, \dots)$ is such a point. Thus, $h$ is bijective. \hspace{5in}\n
  \indent Now we show that $h$ is continuous. Each function $h_1(x) = a_1x_1+b_1, h_2(x)=a_2x_2+b_2, \dots$ is clearly continuous, thus, by \hyperref[thm:MapsProducts19.6]{theorem 19.6}, $h$ is continous. Similarly, $h^{-1}$ is continuous, thus $h$ is homeomorphic.\hspace{5in}\n
  \indent In the box topology, theorem 19.6 does not hold, $h$ would not be continuous, and so would not be a homeomorphism. For example, if $a=1$ and $b=0$, we get exactly the function Munkres uses to show that this theorem does not hold for the box topology.
  \item Show that the \hyperref[thm:AxiomChoice]{axiom of choice} is equivalent to the statement that for any indexed family $\{A_\alpha\}_{\alpha\in J}$ of nonempty sets, with $J\neq 0$, the cartesian product $\prod_{\alpha\in J} A_\alpha$ is not empty.\n
  \indent Intuitively, each element of the cartesian product requires the choice of an arbitrary element from each set $A_\alpha$.
  \item Let $A$ be a set; let $(X_\alpha)_{\alpha\in J}$ be an indexed family of spaces; and let $(f_\alpha)_{\alpha\in J}$ be an indexed family of functions $f_\alpha: A \rightarrow X_\alpha$.
  \begin{enumerate}
    \item Show that there is a unique coarsest topology $\T$ on $A$ relative to which each of the functions $f_\alpha$ is continuous.\hspace{5in}\n
    \indent Suppose there were two such coarsest topologies, $\T$ and $\T'$. Each would contain at least one open set, $U$ and $U'$ respectively which the other did not contain. There then exist two $\alpha$, $i$ and $j$, and two open sets $V$ and $V'$ of $X_i$ and $X_j$ such that $f_i^{-1}(V)=U$ and $f_j^{-1}(V')=U'$. However, for $f_j$ to be continuous, $T'$ must contain $U$ and vice versa, hence there is a single unique coarsest topology. NOTE: This shows that the subbasis for such a topology must be unique, not the topology itself. \n
    \indent Consider the intersection of $\T$ and $\T'$. Each $f_\alpha$ is continuous in this intersection, since it is continuous in each topology, but the intersection is necessarily coarser. Thus there is only one possible coarsest topology. 
    \item Let $\mathcal{S}_\beta=\{f_\beta^{-1}(U_\beta) | U_\beta \text{ is open in } X_\beta\}$, and let $\mathcal{S}=\bigcup \mathcal{S}_\beta$. Show that $\mathcal{S}$ is a subbasis for $\T$.\hspace{5in}\n
    \indent For each function to be continuous, it is necessary that each $f_\alpha^{-1}(U_\alpha)$ be open, where $U_\alpha$ is an open set of $X_\alpha$. This subbasis allows for exactly that, without including any other sets. The inverse of each open set must be part of a subbasis to be sure that it forms a topology, they cannot simply be the topology directly.
    \item Show that a map $g: Y\rightarrow A$ is continuous relative to $\T$ if and only if each map $f_\alpha \circ g$ is continuous.\hspace{5in}\n
    \indent If $g$ is continuous, then each $f_\alpha \circ g$ is continuous, since the composition of continuous functions is continuous. Conversely, suppose each $f_\alpha \circ g$ is continuous, but $g$ is not continous. %Then there exists some open set $V\in A$ such that $g^{-1}(V)$ is not open in $Y$. Denote $f_\alpha(V)$ as $U_\alpha$. Then, each $g^{-1}(f^{-1}_\alpha(U_\alpha))$ is open in $Y$.
    % $V$ is some union of some finite intersection sets in $\mathcal{S}$.
    However, consider each subbasis element $U\in\mathcal{S}$. For each one, there exists some $\alpha$ such that there exists an open set $V\in X_\alpha$ with $f_\alpha^{-1}(V)=U$. $g^{-1}(f^{-1}_\alpha(V))$ is open by hypothesis, so $g^{-1}(U)$ is open. It is sufficient to show that the inverse image of each subbasis element is open to show that a function is continuous, thus $g$ is continuous.
    \item Let $f: A\rightarrow \prod X_\alpha$ be defined by the equation $f(a) = (f_\alpha(a))_{\alpha\in J}$; let $Z$ denote the subspace $f(A)$ of the product space $\prod X_\alpha$. Show that the image under $f$ of each element of $\T$ is an open set of $Z$. \hspace{5in}\n
    \indent %In the subspace $f(A)$, $f$ is obviously surjective. Identically, each $\pi_\alpha\circ f = f_\alpha$ maps to the space $\pi_\alpha(f(A)) = f_\alpha(A)$, so each $f_\alpha$ is surjective.
    Let $U$ belong to the subbasis $\mathcal{S}$ of $\T$, $U$ is the preimage of some open set $V\in X_\beta$ for some $\beta$. $f_\beta(U) = f_\beta(f_\beta^{-1}(V)) = V \cap f_\beta(A)$, thus it is an open set in the subspace $f_\beta(A)$.\n
    % Now, consider the set $B=\prod_{\alpha\in J} V_\alpha$, where $V_\alpha = f_\alpha(A)$ for all $\alpha \neq \beta$, and $U_\alpha = f_\alpha(U)$ when $\alpha = \beta$.
    %Let $B=f(U)$, and let $x$ be an element of $B$. %$B$ must be open in $\prod X_\alpha$, because
    The set $\prod V_\alpha$ where $V_\alpha = X_\alpha$ for all $\alpha \neq \beta$ and $V_\alpha=V$ when $\alpha=\beta$ is open in $\prod X_\alpha$. The intersection $B = \prod V_\alpha\cap f(A)$ is exactly $f(U)$. Proof proceeds: consider an $x\in U$. $f_\beta(x) \in V$, and for any other $\alpha$, $f_\alpha(x)$ is obviously in $X_\alpha$, so $f(x)\in B$. In the other direction, consider an $x\in B$. $\pi_\beta(x) \in f_\beta(U)$. Suppose there were a $y$ such that $f_\beta(y)=\pi_\beta(x)$, but $f(y)\not\in f(U)$. However, $y\in f^{-1}_\beta(\pi_\beta(x))$, equivalently, $y\in f^{-1}_\beta(f_\beta(U))$, so $y\in U$. Thus, $\pi_\beta(x)$ being an element of $f_\beta(U)$ is a sufficient condition for $x\in U$. %However, $y$ must be in $f^{-1}_\beta(V)$, and therefore in $U$, since $U=f^{-1}_\beta(V)$, and $\pi_\beta(x) \in f^{-1}(U)$, so $f(\pi_\beta(x)) \in U$ \n
    $B$ is clearly an open set of $Z$, so $f(U)$ is an open element of $Z$, where $U$ is an arbitrary subbasis element of $\T$. To extend this to arbitrary basis elements, simply note that these are finite intersections of subbasis elements, so if $U$ were a basis element, then $U$ would be intersection of the preimage of some finite sets $V_1, V_2,\dots$ of $X_{\beta_1},X_{\beta_2},\dots$ for some finite number of subscripts $\beta$. Since a finite number of the open sets comprising a basis element of the product topology can be any open set, the proof proceeds similarly. The function of a union is the union of the function of each set, thus we are done.
  \end{enumerate}
\end{enumerate}
\end{document}
\textbf{Chapter 2.20} \label{sec:chapter2.20}
\documentclass[12pt,letterpaper]{article}
\usepackage[pdftex]{graphicx}
\usepackage{alltt}
\usepackage[margin=1in]{geometry}
\usepackage{amsmath, amsthm, amssymb}
\usepackage{verbatim}
\usepackage{ragged2e}
\usepackage{enumitem}
\usepackage{xfrac}
\setlist{parsep=0pt,listparindent=\parindent}
\setlength{\RaggedRightParindent}{\parindent}
\newcommand{\degree}{\ensuremath{^\circ}}
\newcommand{\n}{\break}
\let\oldemptyset\emptyset
\let\emptyset\varnothing
\newcommand{\Wlog}{without loss of generality}
\newcommand{\WLOG}{Without loss of generality}
\usepackage{accents}
\let\thinbar\bar
\newcommand\thickbar[1]{\accentset{\rule{.4em}{.8pt}}{#1}}
\let\bar\thickbar
\usepackage{standalone}
\usepackage{hyperref}
\newcommand{\R}{\ensuremath{\mathbb{R}}}
\usepackage{mathtools}
\DeclarePairedDelimiter{\ceil}{\lceil}{\rceil}
\DeclarePairedDelimiter{\floor}{\lfloor}{\rfloor}
\DeclarePairedDelimiter\abs{\lvert}{\rvert}
\DeclarePairedDelimiter\norm{\lVert}{\rVert}
%%%%%%%%%%%%%%%%%%%%%%%%%%%%%%%%%%%%%%%%%%%%%%%%%%%%%
%TOPOLOGY DOCUMENTS ONLY%
\newcommand{\T}{\ensuremath{\mathcal{T}}}
%%%%%%%%%%%%%%%%%%%%%%%%%%%%%%%%%%%%%%%%%%%%%%%%%%%%%

\begin{document}
\RaggedRight
\begin{enumerate}
  \item \begin{enumerate}
    \item In $\R^n$, define $d'(x,y)=|x_1-y_1|+\dots +|x_n-y_n|$. Show that $d'$ is a metric that induces the usual topology of $\R^n$. Describe the basis elements under $d'$ when $n=2$.\hspace{5in}\n
    \indent It is trivial to see that the first two conditions of a metric are satisfied by $d'$. The third can be seen as follows. We assert that $|x_1-y_1|+|y_1-z_1|+\dots + |x_n-y_n|+|y_n-z_n| \geq |x_1-z_1|+\dots + |x_n-z_n|$. For each $i$ between $1$ and $n$, $|x_i-y_i| + |y_i-z_i| \geq |x_i-z_i|$, which is true - it is the triangle inequality on $\R$. Thus it is true for the sum.\hspace{5in}\n
    \indent The basis elements in $\R^2$ are squares with vertices at $(0,\epsilon), (\epsilon, 0), (0,-\epsilon), (-\epsilon,0)$. \hspace{5in}\n
    \indent Consider an element $B=(a_1,b_1)\times\dots (a_n,b_n)$ of the basis for the standard topology on $\R^n$, and let $x$ be an element of $B$. For each $i$ between $1$ and $n$, let $\epsilon_i=\min(|x_i-b_i|, |x_i-a_i|)$. Then, let $\epsilon = \min_i(\epsilon_i)$. The $\epsilon$-ball $b_{d'}(x,\epsilon)$ contains $x$ and lies within $B$. Thus, the topology induced by $d'$ is finer than the usual topology. \hspace{5in}\n
    \indent In the other direction, consider an $\epsilon$-ball, $B=B_{d'}(x,\epsilon)$, for some $x\in \R^n$ and $\epsilon \in \R$. A basis element $(x_1-\epsilon/4, x_1+\epsilon/4)\times\dots \times (x_n-\epsilon/4, x_n+\epsilon/4)$ will contain $x$, and will lie within $B$. Therefore the usual topology is finer than the topology induced by the metric, and together with the previous part this shows that they are the same topology.
    \item More generally, given $p\geq 1$, define $$d'(x,y)=\left(\sum_{i=1}^n|x_i-y_i|^p\right)^{1/p}$$ for $x,y\in \R^n$. Assume that $d'$ is a metric. Show that it induces the usual topology on $\R^n$.\hspace{5in}\n
    \indent Consider an element $B=(a_1,b_1)\times\dots (a_n,b_n)$ of the basis for the standard topology on $\R^n$, and let $x$ be an element of $B$. For each $i$ between $1$ and $n$, let $\epsilon_i=\min(|x_i-b_i|,|x_i-a_i|)$, and let $\epsilon = \left(\displaystyle\sum_{i=1}^{n}\epsilon_i^p\right)^{1/p}$. Then, $B_{d'}(x,\epsilon)$ lies within $B$ and contains $x$, thus the induced metric topology is finer than the usual topology. \hspace{5in}\n
    \indent In the other direction, the proof proceeds exactly as in the previous problem, because the $\epsilon$-ball for larger $p$ contains the ones for smaller $p$.
  \end{enumerate}
  \item Show that $\R \times \R$ in the dictionary order topology is metrizable.\hspace{5in}\n
  \indent Let $d$ be the square metric on $\R^2$, and let $\bar{d}$ be the standard bounded metric on $d$. Consider the metric $d'((x_1\times y_1), (x_2\times y_2))= \begin{cases} 1 \quad & x_1\neq x_2 \\ \bar{d}((x_1\times y_1),(x_2\times y_2)) \quad & x_1=x_2\end{cases}$. For $\epsilon<1$, an $\epsilon$ ball with this metric will be a vertical line, exactly the basis of the dictionary order metric.
  \item Let $X$ be a metric space with metric $d$. \begin{enumerate}
    \item Show that $d:X\times X \rightarrow \R$ is continuous. \hspace{5in}\n
    \indent Suppose $d$ is not continuous; there is an open basis element $V=(y_1,y_2)$ of $\R$ such that $U=d^{-1}(V)$ is not open. $U$ is the set of points that are a distance between $y_1$ and $y_2$ away from one another. This set could also be expressed as the union of all $B_x$ such that $B_x = B_d(x,y_2) - B_d(x,y_1)$, for any $x\in X$. For each $p\in B_x$, there is a $\delta = \min(d(x,p)-y_1, y_2-d(x,p))$, and $B_d(p,\delta)$ lies in $B_x$, therefore $B_x$ is open by the \hyperref[dfn:metricTopology]{definition of the metric topology}. Since each $B_x$ is open, their union is open, a contradiction, since by hypothesis $U$ is closed. 
    \item Let $X'$ denote a space having the same underlying set as $X$. Show that if \\\noindent$d: X'\times X'\rightarrow \R$ is continuous, then the topology of $X'$ is finer than the topology of $X$.\\
    Let $A$ be the underlying set of both $X$ and $X'$. Suppose $d$ is continuous relative to $X'$. Then for 
  \end{enumerate}
  The result can be summarized as follows: the topology induced by $d$ on $X$ is the coarsest topology relative to which $d$ is continuous.
  \item Consider the product, uniform, and box topologies on $\R^\omega$.
  \begin{enumerate}
    \item In which topologies are the following functions from $\R$ to $\R^\omega$ continuous?
    \begin{itemize}
      \item $f(t) = (t,2t,3t,\dots)$\\
      % Consider an element of the basis of the product topology, $B=V_1\times\dots\times V_n\times\R\times\R\cdots$.
      In summary, $f$ is continuous only in the product topology. \\
      Product topology: Let $f_\alpha : \R \rightarrow \R$ be defined as $f_\alpha(t) = \alpha*t$, then $f$ can also be defined as $f(t) = (f_\alpha(t))_{\alpha\in \mathbb{N}}$. By \hyperref[thm:MapsProducts19.6]{Theorem 19.6}, $f$ is continuous if and only if each $f_\alpha$ is continuous, which they clearly are. \\
      Uniform topology: Let $B$ be a basis element of the uniform topology, $B=B_{\bar{p}}(p,\epsilon)$. Suppose $x\in f^{-1}(B)$ and $y=f(x)$. Then, if $f^{-1}(B)$ is open, there exists some positive real $\delta$ such that $A=(x-\delta, x+\delta)\subset f^{-1}(B)$. Then, $\pi_n(f(A)) = (y_n-n\delta, y_n+n\delta) \subset \pi_n(B)$ Since $n\delta$ increases as $n$ increases, there will be some $n$ for which $n\delta > 2\epsilon = \text{diam } B$, a contradiction, since this means $\bar{d}(y, y+n\delta) > \text{diam } B$, so $(y+n\delta)$ is outside the $\epsilon$-ball. \\
      Box topology: Consider the basis element: $B = (-1, 1) \times (-1/2, 1/2) \times (-1/4, 1/4) \times (-1/8, 1/8) \cdots$ Suppose $f^{-1}(B)$ were open. $0\in f^{-1}(B)$, so for $f^{-1}(B)$ to be open it would have to contain some interval $(-\delta, \delta)$ for some positive real $\delta$. Then $f((-\delta, \delta)) \in B$, so applying $\pi_n$ to each side, $f_n((-\delta,\delta)) = (-n\delta, n\delta) \subset (\sfrac{-1}{2^n}, \sfrac{1}{2^n})$ for all $n$, a contradiction.
      \item $g(t) = (t, t, t, \dots)$ \\
      By the same argument as for $f$, $g$ is continuous in the product topology, and is not continuous in the box topology.
      The proof proceeds similarly as for $f$. If $f^{-1}(B)$ where not open, then for some element $x\in g^{-1}(B)$ there would exist no positive real $\delta$ such that the image of $A = (x-\delta, x+\delta)$ under $g$ would be contained within $B$. Let $y=g(x)$. Then, $\pi_n(g(A)) = (y_n - \delta, y_n+\delta)$. $\delta$ has an upper bound, but no lower bound, so it is possible to choose a $\delta$ that arbitrarily small, including one such that $\bar{d}(y_n, y_n-\delta)=\bar{d}(y_n, y_n+\delta)=\delta < \epsilon-\bar{d}(p, y_n)$. For such a $\delta$, $g(A)\in B$, a contradiction. Thus, $g$ is continuous in the uniform topology.
      \item $h(t) = (t, t/2, t/3, \dots)$ \\
      By the same argument as for $g$, $h$ is continuous in the product and uniform topologies, and is not continuous in the box topology. 
    \end{itemize}
    \item In which topologies do the following sequences converge? \\
    If a sequence converges in a topology $\T$ which is coarser than topology $\T'$, then it clearly converges in $\T'$.
    \begin{itemize}
      \item $w_1 = (1,1,1,\dots), w_2=(0,2,2,2,\dots), w_3=(0,0,3,3,3,\dots)$ \\
      This sequence converges only in the product topology. Let $B=B_D(x, \epsilon)$. Suppose $x=(0,0,0,\dots)$. The distance between $x$ and an element of $w$ is $d_n = D(x,w_n) = 1/n$, so every $\epsilon$-ball around $0$ will eventually contain all $w$. More simply, for every open set of the product topology, there will eventually come a point where all coordinates are in the set.
      \item $x_1 = (1,1,1,\dots), x_2=(0,1/2,1/2,1/2,\dots), x_3=(0,0,1/3,1/3,1/3,\dots)$\\
      % This sequence converges in the box, product, and uniform topologies to $(0,0,0,\dots)$.
      This sequence does not converge in the box topology: consider the open set $U=(-1,1)\times (-1/2,1/2)\times (-1/4,1/4)\times (-1/8,1/8)\cdots$. There is no such $N$ such that every $x_{n>N}\in U$. However, in the uniform topology, it does converge - for an $\epsilon$-ball around $(0,0,0,\dots)$, define $N$ such that $1/N<\epsilon$. Then, every $w_{n>N}$ will lie within the given $\epsilon$-ball.
      \item $y_1 = (1,0,0,0,\dots), y_2=(1/2,1/2,0,0,0,\dots), y_3=(1/3,1/3,1/3,0,0,\dots)$. \\
      The argument proceeds identically to the one for $x$. Thus, $y$ converges in the uniform and product topologies only.
      \item $z_1 = (1,1,0,0,\dots), z_2=(1/2,1/2,0,0,0,\dots), z_3=(1/3,1/3,0,0,0,\dots)$. \\
      This converges in each topology to $(0,0,0,\dots)$. Every open set $U$ in the box topology must contain an interval around $0$ for each coordinate, and there will be a $z$ within that interval.
    \end{itemize}
  \end{enumerate}
  \item Let $\R^\infty$ be the subset of $\R^\omega$ consisting of all sequences that are eventually zero. What is the closure of $\R^\infty$ in $\R^\omega$ in the uniform topology?
  \item Let $\bar{p}$ be the uniform metric on $\R^\omega$. Given $x = (x_1, x_2, \dots) \in \R^\omega$, and $0<\epsilon < 1$, let $U(x,\epsilon) = (x_1 - \epsilon, x_1+\epsilon) \times \dots \times (x_n-\epsilon, x_n+\epsilon)\times \cdots$
  \begin{enumerate}
    \item Show that $U(x,\epsilon)$ is not equal to the $\epsilon$-ball $B_{\bar{p}}(x,\epsilon)$. \\
    The point $y=(x_1+\epsilon/2 \times x_2+2\epsilon/3 \times x_3+3\epsilon/4\times\cdots$ is in $U(x,\epsilon)$ since each coordinate is less than $x_n+\epsilon$, but it is not in $B_{\bar{p}}(x,\epsilon)$, since the supremum $\sup(\{\bar{d}(x_\alpha, y_\alpha) | \alpha \in \mathbb{N}\})$ is equal to $\epsilon$.
    \item Show that $U(x,\epsilon)$, is not even open in the uniform topology. \\
    There is no $\epsilon$-ball containing $y$ that is in $U(x,\epsilon)$, so $U(x,\epsilon)$ is not a basis element and is not the union of basis elements.
    \item Show that $B_{\bar{p}}(x,\epsilon) = \displaystyle\bigcup_{\delta<\epsilon}U(x,\delta)$. \\
    This definition contains only those elements up to $\epsilon$ away from $x$, because the union will never contain the problematic element $U(x,\epsilon)$, so no element in the union can be a sequence approaching a coordinate $\epsilon$ away from $x$.
  \end{enumerate}
  \item Given sequences $a=(a_1,a_2,\dots)$ and $b=(b_1,b_2,\dots)$ in $\R^{omega}$, define the map $h: \R^{\omega} \rightarrow \R^{\omega}$, by the equation $h((x_1,x_2,\dots)) = (a_1x_1+b_1,a_2x_2+b_2,\dots)$. Give $\R^{\omega}$ the uniform topology. Under what conditions on the numbers $a_i$ and $b_i$ is $h$ continuous? A homeomorphism? \\
  For continuity, the sequence $a$ must be bounded above by some real $N$, and bounded below by $-N$. Consider an element $B=B_{\bar{p}}(p,e)$. If there exists a $B$ such that $h^{-1}(B)$ was not open, then for some element $x\in h^{-1}(B)$ there would exist no positive real $\delta<|\epsilon|$ such that the image of $A=(x-\delta, x+\delta)$ under $h$ would be contained within $B$. (Intuitively, this means that there there is an $x$ that is at the ``edge'' of $B$) Let $y=h(x)$. Then, the set $\pi_n(h(A)) = (y_n - a_n\delta, y_n+a_n\delta)$ must be contained within $\pi_n(B)$. Equivalently, $a_n\delta < \epsilon - \bar{d}(p,y_n)$. If $a_n$ were to increase without bound, then eventually there would be some $a_n$ for which the previous inequality were false. However, if $a_n$ is bounded above by $N$, then one simple chooses a $\delta<\epsilon$ that is also less than $(\epsilon-\bar{p}(p,y))/N$. (If $N=0$, i.e. $a_n=0$ for all $n$, then $h$ always yields $b$, and the preimage of $b$ is simply $\R^\omega$, which is open.) For such a $\delta$, $A$ is open, thus there can be no $B$ such that $h^{-1}(B)$ is not open, so $h$ is continuous.\\
  Bijection is a property of the function, it is independent of the topologies of the function's domain and range, thus the proof of bijection in exercise 8 of section 19 is sufficient. Note that $a_i\neq 0$ is a further requirement in this case. It remains to show that $h^{-1}$ is continuous. $h^{-1}$ is equivalent to $h$ with the sequences $a_n' = 1/a_n$ and $b_n' = b_n/a_n$. Thus the same condition is needed to make $h^{-1}$ continuous: $a'$ must be bounded above and below. Thus, $1/a_n$ must be bounded above and below. Therefore, the limit of $a_n$ must not be $0$. In fact, the sequence cannot approach $0$ in any way, for example $\begin{cases} a_n=1/n \quad & n \text{ is odd} \\ a_n = -1 \quad & n \text{ is even}\end{cases}$ is also unacceptable. This is equivalent to requiring a lower and upper bound on $|a_n|$.
  \item Let $X$ be the subset of $\R^\omega$ consisting of all sequences $x$ such that $\sum x_i^2$ converges. Then the formula $$d(x,y) = \left(\sum_{i=1}^\infty (x_i-y_i)^2\right)^{1/2}$$ defines a metric on $X$ (See exercise 10). On $X$ we have the three topologies it ineherits from the box, uniform, and product topologies on $R^\omega$. We also have the topology given by the metric $d$, which we call the $\ell^2$-topology, read ``little ell two.''
  \begin{enumerate}
    \item Show that on $X$ we have the inclusions: box topology $\supset$ $\ell^2$-topology $\supset$ uniform topology. \\
    For each $x$, show that there is basis element of the little ell two topology containing $x$ that is in turn contained within an element of the uniform topology \\
    Consider an $x\in U$, where $U$ is a basis element of the product topology on $X$ inherited from $\R^\omega$, of the form $(U_1\times\dots\times U_n\times \R\times R\cdots)\cap X$, where $U_1,\dots,U_n$ are open sets in $\R$.
    In a convergent series, a finite number of the elements may be arbitrary - the sum of a finite number of elements is always finite. Therefore, the first $n$ coordinates of $U$ are unchanged by the intersection with $X$, they are just open sets in $\R$. In each one, $x_i$ is contained within an interval $(a_i, b_i)$. Let $\epsilon_i = \min((x_i-a_i)^2, (b_i-x_i)^2)$. Let $\epsilon=(\min_i(\epsilon_i))^{1/2}$. Then $B_d(x,\epsilon)\in U$. Therefore, $\ell^2$-topology $\supset$ product topology. Well damn it I did the wrong problem.\\
    Consider an $x\in U$, where $U$ is a basis element of the uniform topology on $X$ inherited from $\R^w$, of the form $B_{\bar{p}}(p,\epsilon)\cap X$. Let $\epsilon' = \min(\bar{p}(p,x), \epsilon - \bar{p}(p,x))$. Then $V=B_{\bar{p}}(x,\epsilon')\cap X \subset U$. Furthermore, $A=B_d(x, \epsilon') \in V$, since in this case, no single coordinate $y_i$ of an arbitrary point $y\in A$ can be $\epsilon_i$ away from $x$, even if all the other coordinates were equal to the corresponding coordinate of $x$. \hyperref[thm:basisFiner]{Therefore}, the $\ell^2$-topology is finer than the uniform topology. \\
    Now, consider an $x\in U$, where $U=B_d(x,\epsilon)$, a basis element of the $\ell^2$-topology. Consider the geometric series: $\sum_{k=1}^\infty(r^k) = r/(1-r)$. Suppose we set this series equal to $\epsilon$. Algebra gets us the result: $r=\epsilon/(1+\epsilon)$, i.e. $\sum_{k=1}^\infty(\epsilon/(1+\epsilon))^k = \epsilon$ Now consider the infinite sequence $y=((\epsilon/(1+\epsilon))^{1/2}, (\epsilon/(1+\epsilon))^{2/2},\dots (\epsilon/(1+\epsilon))^{n/2},\dots)$. Clearly, the sum of each element squared is $\epsilon$, and so $d((0,0,0,\dots),y)=\sqrt{\epsilon}$. Therefore, the set $V = (x_1-(\epsilon/(1+\epsilon))^{1/2}, (x_1+\epsilon/(1+\epsilon))^{1/2})\times \dots \times(x_n-(\epsilon/(1+\epsilon))^{n/2}, (x_n+\epsilon/(1+\epsilon))^{n/2})\times \cdots$. This is clearly a subset of $U$, because each coordinate is at most an element of a sequence that is $\sqrt{\epsilon}$ away from $x$. $V$ is also clearly an element of the box topology. Hence, by the same lemma as used in the previous paragraph, the box topology is finer than the $\ell^2$-topology.
    \item The set $\R^\infty$ of all sequences that are eventually zero is contained in $X$. Show that the four topologies that $\R^\infty$ inherits as a subspace of $X$ are all distinct. \\
    Consider a basis element $U$ of the topology on $R^\infty$ inherited as a subspace of the product topology, an let $x\in U$. $U=\prod U_i \cap \R^\infty$, where finitely many $U_i$ are not $R$. For the remaining $U_i$, let $(a_i,b_i)$ be a subset of $U_i$, and let $\epsilon = \min_i(\min(\bar{d}(a_i,x), \bar{d}(b_i,x)))$. Then $B_{\bar{p}}(x,\epsilon) \cap R^\infty \subset U$. Therefore the subspace of the uniform topology is finer than the subspace of the product topology. \\
    Now we show that the uniform topology is distinct from the product topology. An open basis element of the product topology on $\R^\infty$ cannot lie within an open basis element of the uniform topology. Consider an aribitrary $\epsilon$-ball, $U=B_{\bar{p}}(x,\epsilon) \cap \R^\infty$, and any basis element $V$ of the product topology on $\R^\infty$ that is a neighborhood of $x$. For infinitely many $n$, $\pi_n(V)$ is a set that contains elements farther than $\epsilon$ from $x_n$, because any one of those coordinates may be non-zero. Thus, the topology on $\R^\infty$ inherited from the uniform topology is \emph{strictly} finer than the one inherited from the product topology. \\
    The proof that the inherited $\ell^2$-topology is finer than the inherited uniform topology is identical to the one in part $A$, the change from $X$ to $\R^\infty$ does not affect the proof. Furthermore, it is strictly finer: in the uniform topology, a basis element $B_{\bar{p}}(x,\delta) \cap R^\infty$ contains a point with any arbitrarily large finite amount of coordinates that are up to $\delta$ away from $x$ using the $\bar{d}$ metric. A point with $\ceil{(\epsilon+1)/\delta}$ such coordinates will be at least $\epsilon$ away from $x$ under the $d$ metric, but still only $\delta$ away under the $\bar{p}$ metric. Therefore there is no basis element in the uniform topology containing $x$ and lying within the basis element of the $\ell^2$-topology, proving that the $\ell^2$-topology is strictly finer.\\
    The box topology is finer than the $\ell^2$-topology - each basis element of the $\ell^2$-topology clearly is an element of the box topology; each coordinate is an open set of $\R$. It is strictly finer: consider an open set $U=((-1,1)\times(-1/2,1/2)\times(-1/4,1/4)\times(-1/8,1/8)\times\cdots)\cap \R^\infty$. Let $x=0\times0\times0\cdots \in U$, and suppose there where a basis element of the $\ell^2$-topology, $V=B_d(x,\epsilon)\cap \R^\infty$ that lies within $U$. But, for any $\epsilon$, there will be a finite $n$ such that the supremum of $\pi_n(U)$ is smaller than $\epsilon$, thus $V$ does not lie inside of $U$.
    \item The set $H=\displaystyle\prod_{n\in\mathbb{Z}_+} [0,\sfrac{1}{n}]$ is contained in $X$; it is called the Hilbert cube. Compare the four topologies that $H$ inherits as a subspace of $X$. \\
    The inherited product topology lies within the inherited uniform topology, the proof is the same as the first paragraph of part b. However, on the Hilbert cube, the product topology also contains the uniform topology; they are the same topology. Let $B=B_{\bar{p}}(x,\epsilon)\cap H$ be a basis element of the uniform topology. Define $n$ such that $1/n<\epsilon$, for example, $n=\ceil{2/(\epsilon)}$. Let $U_{1\leq i \leq n} = (x-\epsilon/2, x+\epsilon/2)$, and let $U_{i>n} = \R$. Finally, let $U=(\prod U_i) \cap H$ be a basis element of the topology inherited by the Hilbert cube as a subspace of the product topology. This $U$ lies within $B$ and contains $x$, because for all $i\leq n$, there is no point farther than $\epsilon$ from $x_i$, and for all $i$ greater than $n$, $\pi_i(H)$ is a set that contains no point farther than $\epsilon$ away from $x_i$. \\
    Now we consider the inherited $\ell^2$-topology. It is finer than the uniform topology, the proof again proceeds the same way as the proof in part a. However, the same technique as was used in the previous paragraph shows that the uniform topology is also finer than the $\ell^2$-topology, they are the same topology in the Hilbert Cube.\\
    The same argument as used in part b shows that the box topology is still strictly finer than the $\ell^2$ topology on the Hilbert cube.
  \end{enumerate}
  \item Show that the euclidean metric $d$ on $\R^n$ is a metric, as follows: if $x,y\in \R^n$ and $c\in \R$, define $x+y=(x_1+y_1,\dots,x_n+y_n)$; $cx = (cx_1, \dots, cx_n)$; and $x\cdot y = x_1y_1 + \dots + x_ny_n$.\begin{enumerate}
    \item Show that $x\cdot (y+z) = (x\cdot y) + (x\cdot z)$
    $$x_1(y_1+z_1)+\dots + x_n(y_n+z_n) = x_1y_1+\dots+x_ny_n + x_1z_1+\dots+x_nz_n$$
    \item Show that $\abs{x\cdot y} \leq \norm{x}\norm{y}$. Hint: If $x,y \neq 0$, let $a=1/\norm{x}$ and $b=1/\norm{y}$, and use the fact that $\norm{ax\pm by} \geq 0$. \\
    Squaring both sides, we have $(x_1y_1 + \dots + x_ny_n)^2 \leq (x_1^2+\dots+x_2^2)(y_1^2+\dots+y_2^2)$. Fuck vectors. \\
    \item Show that $\norm{x+y} \leq \norm{x} + \norm{y}$. Hint: compute $(x+y)\cdot(x+y)$ and apply (b). \\
    See the last sentence of above.
    \item Verify that $d$ is a metric.
  \end{enumerate}
  \item Let $X$ denote the subset of $R^\omega$ consisting of all sequences $(x_1,x_2,\dots)$ such that $\sum x_i^2$ converges. You may assume the standard facts about infinite series, listed in exercise 11 of the following section. \begin{enumerate}
    \item Show that if $x,y\in X$, then $\sum \abs{x_iy_i}$ converges. Hint: Use (b) of exercise 9 to show that the partial sums are bounded.
    \item Let $x\in \R$. Show that if $x,y\in X$, then so are $x+y$ and $cx$. \\
    \item Show that $d(x,y) = \left(\displaystyle\sum_{i=1}^\infty(x_i-y_i)^2\right)^{1/2}$ is a well-defined metric on $X$.
  \end{enumerate}
  \item [*11.] Show that if $d$ is a metric for $X$, then $d'(x,y) = d(x,y)/(1+d(x,y))$ is a bounded metric that gives the topology of $X$. Hint: if $f(x) = x/(1+x)$ for $x>0$, use the mean-value theorem to show that $f(a+b)-f(b)\leq f(a)$. \\
  The first two conditions for a metric clearly hold. For the third, we must show that $$\frac{d(x,y)}{1+d(x,y)} + \frac{d(y,z)}{1+d(y,z)} \leq \frac{d(x,z)}{1+d(x,z)}$$.
  %\begin{equation} \begin{split}
  %    d(x,y)(1 + d(y,z) + d(x,z) + d(y,z)d(x,z)) + \\
  %    d(y,z)(1 + d(x,y) + d(x,z) + d(x,y)d(x,z)) \leq \\
  %    d(x,z)(1 + d(x,y) + d(y,z) + d(x,y)d(y,z))
  %  \end{split} \end{equation}
  %$$\frac{1+d(x,y)+d(y,z)+d(x,y)d(y,z)}{d(x,y)+d(y,z)+2d(y,z)d(x,y)} \geq \frac{1+d(x,z)}{d(x,z)}$$
  %$$\frac{1+d(x,z)+d(x,y)d(y,z)}{d(x,z)+2d(x,y)d(y,z)} \geq \frac{1+d(x,z)}{d(x,z)}$$
  %Let $d(x,z)=u$ and $d(x,y)d(y,z)=c$ Then we have $$\frac{1+u+c}{u+2c} \geq \frac{1+u}{u}$$
  %$$u(1+u)+cu \geq u(1+u)+2c+2cu$$
  %$$cu \geq 2c+2cu$$
  %Oops.
\end{enumerate}
\end{document}
\textbf{Chapter 2.21} \label{sec:chapter2.21}
\documentclass[12pt,letterpaper]{article}
\usepackage[pdftex]{graphicx}
\usepackage{alltt}
\usepackage[margin=1in]{geometry}
\usepackage{amsmath, amsthm, amssymb}
\usepackage{verbatim}
\usepackage{ragged2e}
\usepackage{enumitem}
\usepackage{xfrac}
\setlist{parsep=0pt,listparindent=\parindent}
\setlength{\RaggedRightParindent}{\parindent}
\newcommand{\degree}{\ensuremath{^\circ}}
\newcommand{\n}{\break}
\let\oldemptyset\emptyset
\let\emptyset\varnothing
\newcommand{\Wlog}{without loss of generality}
\newcommand{\WLOG}{Without loss of generality}
\usepackage{accents}
\let\thinbar\bar
\newcommand\thickbar[1]{\accentset{\rule{.4em}{.8pt}}{#1}}
\let\bar\thickbar
\usepackage{standalone}
\usepackage{hyperref}
\newcommand{\R}{\ensuremath{\mathbb{R}}}
\usepackage{mathtools}
\DeclarePairedDelimiter{\ceil}{\lceil}{\rceil}
\DeclarePairedDelimiter{\floor}{\lfloor}{\rfloor}
\DeclarePairedDelimiter\abs{\lvert}{\rvert}
\DeclarePairedDelimiter\norm{\lVert}{\rVert}
%%%%%%%%%%%%%%%%%%%%%%%%%%%%%%%%%%%%%%%%%%%%%%%%%%%%%
%TOPOLOGY DOCUMENTS ONLY%
\newcommand{\T}{\ensuremath{\mathcal{T}}}
%%%%%%%%%%%%%%%%%%%%%%%%%%%%%%%%%%%%%%%%%%%%%%%%%%%%%

\begin{document}
\RaggedRight
\begin{enumerate}
  \item Let $X$ be a metric space with metric $d$, and let $A$ be a subset of $X$. Show that $d|A\times A$ is a metric for the subspace topology on $A$.\\
  Let $U$ be a basis element of $A$ in the subspace topology that contains $x$, $U$ is of the form $U=B_d(x,\epsilon) \cap A$. There exists some $\delta$ such that $B_{d|A\times A}(y,\delta)\subset U$, where $y\in U$, for instance, $\delta = \epsilon - d(x,y)$. Therefore, the topology created by the restricted metric is finer than the subspace topology.\\
  In the other direction, given an $x\in A$ and $U=B_{d|A\times A}(x,\epsilon)$, $U$ is equivalent to $B_d(x, \epsilon)\cap A$, so the subspace topology is finer than the metric topology. Therefore, the two topologies are equal.
  \item Let $X$ and $Y$ be metric spaces with metrics $d_x$ and $d_Y$. Let $f:X\rightarrow Y$ have the property that for every pair of points $x_1,x_2\in X$, $d_Y(f(x_1),f(x_2))=d_X(x_1,x_2)$. Show that $f$ is an embedding. $f$ is called an \emph{isometric embedding} of $X$ in $Y$. \\
  First we show that $f$ is injective. Suppose the opposite, there exist distinct $x_1,x_2$ such that $f(x_1)=f(x_2)$. But then, by the properties of metrics, $d_Y(f(x_1),f(x_2)) = 0$, whereas $d_X(x_1, x_2)\neq 0$. Therefore $f$ is injective. \\
  Now we show that $f$ is continuous.  Suppose the opposite, there exists an open basis element $V\subset Y$, $V=B_{d_Y}(y,\epsilon)$, such that $f^{-1}(V)$ is not open in $X$, where $y\in Y$. Let $x=f^{-1}(y)$, $x$ is a single element since $f$ is injective. For every point $y'$ in $V$,  $d_Y(y,y')=d_X(x,f^{-1}(y'))<\epsilon$, and so $B_{d_X}(x,\epsilon)$ is open in $X$. Therefore, $f$ is continuous. \\
  Let $Z=f(X)$ and let $f': X\rightarrow Z$ be obtained by restrincting the range of $f$ to $Z$. $f'$ is a bijection. We show that $f$ is an embedding, that is, that $f'$ is a homeorphism, or equivalently, that $f'^{-1}$ is continuous. Assume the opposite, there exists some basis element $U\in X$, $U=B_{d_X}(x,\epsilon)$ such that $f(U)$ is not open in $Y$. The argument that showed that $f$ is continuous also shows that this claim is true, therefore $f'^{-1}$ is continuous, $f'$ is homeomorphic, and $f$ is an embedding of $X$ in $Y$.
  \item Let $X_n$ be a metric space with metric $d_n$, for $n\in \mathbb{Z}_+$.
  \begin{enumerate}
    \item Show that $p(x,y) = \max\{d_1(x_1,y_1),\dots,d_n(x_n,y_n)\}$ is a metric for the product space $X_1\times\dots\times X_n$. \\
    Let $B=U_1\times\dots\times U_n$ be a basis element of the given product space, where each $U_i=B_{d_i}(x_i,\epsilon_i)$ is a basis element of $X_i$, with $x\in X_i$. Then, the basis element $B_p((x_1,\dots,x_n), min_i(\epsilon_i))$ lies within $B$, therefore the metric for the product space is finer than the product topology on that space. \\
    In the other direction, let $x=(x_1,\dots,x_n)$, and $V=B_p(x, \epsilon)$ be a basis element of the metric topology. Then, $B_{d_1}(x_1,\epsilon)\times\dots\times B_{d_n}(x_n, \epsilon)$ lies within $V$. Therefore the product topology is finer than the metric topology, and so the two topologies are equivalent. 
    \item Let $d_i=\min\{d_i,1\}$. Show that $D(x,y)=\sup\{\bar{d}_i(x_i,y_i)/i\}$ is a metric for the product space $\prod(X_i)$ \\
    The first two properties of a metric are satisfied trivially. The triangle inequality is proved as follows. $\bar{d}(x_i,z_i)/i\leq \bar{d}(x_i,y_i)/i + \bar{d}(y_i,z_i)/i \leq D(x,y)+D(y,z)$. Therefore, $D(x,z) = \sup\{ \bar{d}(x_i,z_i)/i\} \leq D(x,y)+D(y,z)$. \\
    Now we show that $D$ gives the product topology $\prod(X_i)$. Let $B=B_D(x,\epsilon)$ be a basis element of the metric topology. Let $N$ be large enough that $1/N < \epsilon$, then, the basis element of the product topology $B_{d_1}(x_1,\epsilon)\times\dots\times B_{d_N}(x_N,\epsilon)\times X_{N+1}\times\cdots$ lies within $B$. For the first $N$ elements, each one is no more than $\max(\epsilon,1)$ away from $x$, and for the rest, $\sup\{\bar{d}(x,y)\}=1$, so $\bar{d}(x,y)/i < 1/N < \epsilon$, since $i>N$. Therefore, overall the metric topology is finer than the product topology. \\
    In the other direction, Let $U=\prod U_i$ be a basis element of the product topology, where $U_i=B_{d_i}(x_i,\epsilon_i)$ is a basis element of $X_i$ and $x_i\in X_i$ for $i=\alpha_1,\dots,\alpha_n$ and $U_i=X_i$ for all other $i\in \mathbb{Z}_+$. Then let $\epsilon = min(\epsilon/i)$ for $i=\alpha_i,\dots,\alpha_n$. $x\in B_D(x,\epsilon)\subset U$. Thus, the two topologies are equivalent.
  \end{enumerate}
  \item Show that \hyperref[dfn:lowerLimitTopology]{$\R_\ell$} and the \hyperref[dfn:orderedSquare]{ordered square} satisfy the first countability axiom. (This result does not, of course, imply that they are metrizable.) \\
  For each element $x\in\R_\ell$ of the lower limit topology, consider the sequence $U_i=[x,x+1/i)$, where each $U_i$ is a basis element of $\R_\ell$. The countable collection of such $U_i$ for all $i\in\mathbb{Z}_+$ forms a countable basis for $x$; any neighborhood of $x$ contains a $U_i$ for at least one $i$. Therefore $\R_\ell$ satisfies the first countability axiom. \\
  For each element $(x\times y)\in I_o^2$, the ordered square, let $U_i=(x \times y - \frac{y}{2i}, x\times y + \frac{1-y}{2i})$. $y-\frac{y}{2i}>0$ and $y+\frac{1-y}{2i} < 1$, and by the same arguments as above, $I_o^2$ satisfies the first countability axiom.
  \item{} [prove] Theorem. Let $x_n\rightarrow x$ and $y_n \rightarrow y$ in the space $\mathbb{R}$. Then, $x_n+y_n \rightarrow x+y,\; x_n-y_n\rightarrow x-y,\; x_ny_n\rightarrow xy$ and provided each $y_n\neq 0$ and $y\neq 0$, $x_n/y_n\rightarrow x/y$. [Hint: apply lemma 21.4, recall from the exercises of section 19 that if $x_n\rightarrow x$ and $y_n\rightarrow y$, then $x_n\times y_n \rightarrow x\times y.$]
  \item Define $f_n : [0,1]\rightarrow \R$ by the equation $f_n(x)=x^n$. Show that the sequence $(f_n(x))$ converges for each $x\in [0,1]$, but that the sequence $(f_n)$ does not converge uniformly. \\
  Calculus shows that for each such $x$, the sequence $(f_n(x))$ converges to 0, except $x=1$ where it converges to 1. Now, suppose that the sequence does converge uniformly; then there is a function $f:X\rightarrow Y$ such that given $\epsilon>0$ there exists an integer $N$ such that $d(f_n(x),f(x))<\epsilon$ for all $n>N$ and for all $x\in X$. Therefore, for all $n>N$, $d(x^n,f(x))<\epsilon$. For all $0\leq x<1$, as $n$ increases $x^n$ decreases to $0$, thus $f(x)$ must be less than $\epsilon$ for the condition to hold. However, for $x=1$ $f(x)$ would have to be 1, so the condition fails for any $\epsilon < 1/2$, so $(f_n)$ does not converge uniformly.
  \item Let $X$ be a set, and let $f_n:X\rightarrow \R$ be a sequence of functions. Let $\bar{p}$ be the uniform metric on the space $\R^X$. [Note: $\R^X$ is the set of all functions mapping $X$ to $\R$.] Show that the sequence $(f_n)$ converges uniformly to the function $f: X\rightarrow \R$ if and only if the sequence $(f_n)$ converges to $f$ as elements of the metric space $(R^X,\bar{p})$.
  \item Let $X$ be a topological space and let $Y$ be a metric space. Let $f_n: X\rightarrow Y$ be a sequence of continuous functions. Let $x_n$ be a sequence of points of $X$ converging to $x$. Show that if the sequence $(f_n)$ converges uniformly to $f$, then $(f_n(x_n))$ converges to $f(x)$. \\
  %We are given the fact that $d(f_n(x_i), f(x_i)) < \epsilon$ for all $n > N_i$, for each $i$.
  %Let $V_\epsilon=B(f(x), \epsilon)$ be a basis element of $Y$. Each $U_\epsilon = f^{-1}(V_\epsilon)$ is a neighborhood of $x$ in $X$, since each element of $(f_n)$ is continuous, so $f$ is continuous. Let $M_\epsilon$ be the value such that for all $n>M_\epsilon$, $x_n\in U$, and so for all $n>M_\epsilon$, $d(f(x), f(x_n)) < \epsilon$.\\
  %Suppose $(f_n(x_n))$ does not converge to $f(x)$, that is, there exists a neighborhood $B=B(f(x),\epsilon)$ such that there is no $M$ for which $f_n(x_n)\in B$ for each $n>M$. \\
  %For each $i$, there is an $N_i$ such that $U=d(f_n(x_i),f(x_i))<\epsilon$ for all $n>N_i$. Furthermore, there is an $M$ such that each $x_{i>M}\in U$, or equivalently, $d(f(x_i),f(x)) < \epsilon$. Therefore, combining these, $d(f_n(x_i), f(x))<2\epsilon$ when $n>N_i$ and $i>M$. \\
  Let $U=f^{-1}(B_d(f(x), \epsilon/2))$. $U$ is open since $f$ is continuous, so there is some $N$ for which $x_{n>N}\in U$, and so $d(f(x_{n>N}),f(x))<\epsilon/2$ Since $(f_n)$ converges uniformly, there is an $M$ such that $d(f_n(y),f(y)) < \epsilon/2$, for all $y\in X$ and all $n>M$. Consider those $y$ that are in $U$, and particularly those that happen to be an $x_n$. Then, $d(f_n(x_n),f(x_n))<\epsilon/2$. Using the triangle inequality, $d(f_n(x_n),f(x_n)) + d(f(x_n), f(x)) < d(f_n(x_n), f(x) < \epsilon$, for all $n > \max(M,N)$. Thus, for any neighborhood $B = B_d(f(x), \epsilon)$, for all $n>\max(M,N)$ every element of the sequence $(f_n(x_n))\in B$; so by definition $(f_n(x_n))$ converges to $f(x)$.
  \item Let $f_n: \R \rightarrow \R$ be the function $$f_n(x) = \frac{1}{n^3(x-\sfrac{1}{n})^2+1}$$ Let $f: \R \rightarrow \R$ be the zero function.
  \begin{enumerate}
    \item Show that $f_n(x)\rightarrow f(x)=0$ for each $x\in R$.\\
    If $x>0$, then for some $N$, $(x-1/n)$ will be greater than $0$ for all $n>N$, so $f_n(x)>0$ for all such $n$. Furthermore, for all such $n$, $1/n > x-1/n$, if $x<1$, so $1/(n^3/n^2+1) = 1/n > f_n(x)$. $(1/n)_{n>N}$ converges to $0$, so $f_n(x)$ converges to $0$. If $x\geq 1$, it simly increases the denominator, which decreases the value of the function further, so the logic holds. \\
    If $x\leq 0$, then $(x-1/n)^2>0$ anyway, the same logic shows that it converges to $0$.
    \item Show that $f_n$ does not converge uniformly to $f$. (This shows that the converse of theorem 21.6 does not hold; the limit function $f$ may be continuous even though the convergence is not uniform.) \\
    Suppose  the sequence does converge uniformly to $f$; for any $\epsilon$ there is an $N$ such that for all $x\in\R$, $f_n(x)<\epsilon$. However, as the figure shows, when $x=1/n$, $f_n(x)=1$ for any $n$. Thus if $\epsilon<1$, there can be no such $N$, so $f_n$ does not converge uniformly.
  \end{enumerate}
  \item Using the \hyperref[dfn:continuous3]{closed set formulation of continuity}, (Theorem 18.1), show that the following are closed subsets of $\R^2$: $A = \{ x\times y\; |\; xy=1\}$; $S^1 = \{ x\times y\;|\; x^2+y^2=1\}$; and $B^2=\{x\times y\;|\;x^2+y^2\leq 1$. The set $B^2$ is the closed unit ball in $\R^2$. \\
  % Define $f(x) = 1/x$, where $f:\R - \{0\} \rightarrow \R - {0}$ then $A$ is equivalent to the set $x\times f(x)$ for all $x\in \R - \{0\}$, so define $h(x) = i(x) \times f(x)$ where $i$ is the identity function. $i$ and $f$ are continuous, so $f$ is continuous, and so $h: \R-\{0\}\rightarrow A$ is continuous.
  Let $\pi_1: A \rightarrow \R - {0}$ be defined $\pi_1(x\times y) = x$. This function is continuous, and it is bijective, it's inverse is the continuous function $f(x) = x\times 1/x$. $\pi_1^{-1}(\R-\{0\}) = A$, $\R - \{0\}$ is closed, since $\pi_1$ is a homeomorphism it preserves topological properties, so $A$ is closed. \\
  \item Skipping this too, for now
  \item Prove continuity of the algebraic operations on $\R$ as follows: use the metric $d(a,b)=\abs{a-b}$ on $\R$ and the square metric $p((x\times y),(x_0\times y_0)) = \max(\abs{x-x_0}, \abs{y-y_0})$ on $\R^2$.
  \begin{enumerate}
    \item Show that addition is continuous. [Hint: Given $\epsilon$, let $\delta = \epsilon/2$ and note that $d(x+y, x_0+y_0) \leq \abs{x-x_0} + \abs{y-y_0}.$]\\
    Let addition be encapsulated in the function $f(x,y) = x+y$, which maps $\R^2$ to $\R$. \\
    In $\R\times\R$, a point $x\times y$ represents two numbers which are to be added up to $x+y$. Consider an $\epsilon$-ball around this point $x\times y$, $B=B_p(x\times y, \epsilon/2)$. Let $x_0\times y_0$ be a point in $B$. Then, consider the distance $d(f(x\times y), f(x_0\times y_0)) = d(x+y, x_0+y_0) = \abs{x+y-x_0-y_0} \leq \abs{x-x_0}+\abs{y-y_0}$ (Proof of the inequality proceeds by case study of the components $x-x_0$ and $y-y_0$. If both are positive or both are negative the LHS and RHS are equal, if the have opposite signs then the RHS increases while the LHS decreases.) We have that $\max(\abs{x-x_0},\abs{y-y_0})<\epsilon/2$, therefore $d(x+y, x_0+y_0)<2*\max(\abs{x-x_0},\abs{y-y_0})<\epsilon$. \\
    Let $z\in \R$, and let $U=B_d(z,\epsilon)$, and $V=f^{-1}(U)$.  Then for each point $x\times y$ in $f^{-1}(z)$, $B$ is an open element of $\R\times\R$ that contains $x\times y$. Each point in $B$ represents a point whose sum is no more than $\epsilon$ away $f(x\times y)=z$, and therefore is within $U$. The union of all such sets $B$ is exactly $U$, and the arbitrary union of basis elements is open, so for every open $U\subset \R$, the preimage is open, so $f$ is continuous.
    \item Show that multiplication is continuous. [Hint: Given $(x_0,y_0)$ and $0<\epsilon<1$, let $3\delta = \epsilon/(|x_0|+|y_0|+1)$ and note that $d(xy,x_0y_0)\leq \abs{x_0}\abs{y-y_o} + \abs{y_0}\abs{x-x_0} + \abs{x-x_0}\abs{y-y_0}$]. \\
    Let multiplication be encapsulated in the function $f(x,y) = xy$, which maps $\R^2$ to $\R$. Let $z\in \R$, $U=B_d(z,\epsilon)$, and $V=f^{-1}(U)$. Then for each point $x\times y$ in $f^{-1}(z)$, we wish to construct a neighborhood of that point which contains every point whose image under $f$ (product of its coordinates) is less than $\epsilon$ away from $z$. \\
    To do this, consider $B=B_p(x\times y, \delta)$, and $x_0\times y_0\in B$. $d(f(x\times y), f(x_0\times y_0)) = \abs{xy-x_0y_0}  \leq \abs{x_0}\abs{y-y_0} + \abs{y_0}\abs{x-x_0} + \abs{x-x_0}\abs{y-y_0}$. Also, $\max(\abs{x-x_0},\abs{y-y_0})<\delta$. Therefore, $d(f(x\times y),f(x_0\times y_0)) < (\abs{x_0}+\abs{y_0})\delta+\delta^2$
    \item Show that the operation of taking reciprocals is a continuous map from $\R-\{0\}$ to $\R$. [Hint: Show the inverse image of the interval $(a,b)$ is open. Consider five cases, according as $a$ and $b$ are positive, negative, or zero.]
    \item Show that the subtraction and quotient operations are continuous.
  \end{enumerate}
\end{enumerate}
\end{document}
\textbf{Chapter 2.22} \label{sec:chapter2.22}
\documentclass[12pt,letterpaper]{article}
\usepackage[pdftex]{graphicx}
\usepackage{alltt}
\usepackage[margin=1in]{geometry}
\usepackage{amsmath, amsthm, amssymb}
\usepackage{verbatim}
\usepackage{ragged2e}
\usepackage{enumitem}
\usepackage{xfrac}
\setlist{parsep=0pt,listparindent=\parindent}
\setlength{\RaggedRightParindent}{\parindent}
\newcommand{\degree}{\ensuremath{^\circ}}
\newcommand{\n}{\break}
\let\oldemptyset\emptyset
\let\emptyset\varnothing
\newcommand{\Wlog}{without loss of generality}
\newcommand{\WLOG}{Without loss of generality}
\usepackage{accents}
\let\thinbar\bar
\newcommand\thickbar[1]{\accentset{\rule{.4em}{.8pt}}{#1}}
\let\bar\thickbar
\usepackage{standalone}
\usepackage{hyperref}
\newcommand{\R}{\ensuremath{\mathbb{R}}}
\usepackage{mathtools}
\DeclarePairedDelimiter{\ceil}{\lceil}{\rceil}
\DeclarePairedDelimiter{\floor}{\lfloor}{\rfloor}
\DeclarePairedDelimiter\abs{\lvert}{\rvert}
\DeclarePairedDelimiter\norm{\lVert}{\rVert}
%%%%%%%%%%%%%%%%%%%%%%%%%%%%%%%%%%%%%%%%%%%%%%%%%%%%%
%TOPOLOGY DOCUMENTS ONLY%
\newcommand{\T}{\ensuremath{\mathcal{T}}}
%%%%%%%%%%%%%%%%%%%%%%%%%%%%%%%%%%%%%%%%%%%%%%%%%%%%%

\begin{document}
\RaggedRight
\begin{enumerate}
  \item Check the details of Example 3, repeated here. Let $p$ be the map of the real line $\R$ onto the three point set $A=\{a,b,c\}$ defined by $p(x)=\begin{cases} a &\quad \text{if } x>0 \\ b &\quad \text{if } x<0 \\ c &\quad \text{if } x=0 \end{cases}$. Check that the quotient topology on $A$ induced by $p$ is the following set of subsets of $A$: $\{\{a\},\{b\},\{a,b\},\{a,b,c\}\}$. \\
  We wish to create a topology such that a set $U$ in $A$ is open if and only if $p{-1}(U)$ is open in $\R$. $p^{-1}(\{a\}) = (0,\infty)$, which is open in $\R$, so $\{a\}$ must be open. Similarly, $\{b\}$ and $\{a,b\}$ must be open. $p^{-1}(\{c\}) = \{0\}$, which is not open, so $\{c\}$ is not open. Similarly for other sets which include $c$, except the entire set $A$, since $p^{-1}(A)=\R$, $A$ is open. 
  \item \begin{enumerate}
    \item Let $p: X\rightarrow Y$ be a continuous map. Show that if there is a continuous map $f: Y\rightarrow X$ such that $p\circ f$ equals the identity map of $Y$, then $p$ is a quotient map. \\
    Suppose there is such a map $f$. Let $V\subset Y$ be open; since $p$ is continuous, $p^{-1}(V)$ is open in $X$. In the other direction, we must show that if $p^{-1}(V)$ is open, then $V$ is open. $f$ is continuous, therefore assuming $p^{-1}(V)$ is open, $f^{-1}(p^{-1}(V))$ will also be open. Since $p\circ f = i$, $f^{-1}\circ p^{-1} = (p\circ f)^{-1} = i^{-1} = i$, $f^{-1}(p^{-1}(V) = V$, so $V$ is open. Hence $p$ is a quotient map.
    \item If $A \subset X$, a retraction of $x$ onto $A$ is a continuous map $r: X\rightarrow A$ such that $r(a)=a$ for each $a\in A$. Show that a retraction is a quotient map. \\
    A retraction is clearly surjective. %$r$ is continuous by definition, so for each open subset $V$ of $A$, $U=r^{-1}(V)$ is open.
    We must show that if $r^{-1}(V)$ is open, then $V$ is open, where $V$ is a subset of $A$. Because of the way $r$ is defined, clearly $r^{-1}(V)$ is some set $U$ such that $U\cap A = V$. Therefore, if $U$ is open, in the subspace topology $U\cap A$ will be open, so $V$ is open, so $r$ is a quotient map.
  \end{enumerate}
  \item Let $\pi_1:\R\times\R \rightarrow \R$ be the projection on the first coordinate. Let $A$ be the subspace of $\R\times\R$ consisting of all points $x\times y$ for which either $x\geq 0$ or $y=0$ (or both); let $q: A\rightarrow \R$ be obtained by restricting $\pi_1$. Show that $q$ is a quotient map that is neither open not closed. \\
  $q$ is clearly surjective. We show that $q$ is continuous. Let $V$ be an open set of $\R$, $V=(a,b)$, with $a,b\in\R$, then $U=q^{-1}(V)=(a,b)\cup (\max(0,a),b)\times\R$ is open, since $U=(a,b)\times\R \cap A$. Now we show the converse, that if $p^{-1}(V)=U$ is open, then $V$ must be open. $U$ is open, therefore it is of the form $(a,b)\times(b,c)\cap A$, and $V$ will be of the form $(a,b)$, an open set. Therefore $q$ is a quotient map.
  \item \begin{enumerate}
    \item Define an equivalence relation on the plane $X=\R^2$ as follows: $x_0\times y_0 \sim x_1\times y_1 \text{ if } x_0+y_0^2 = x_1 + y_1^2$. Let $X^*$ be the corresponding quotient space. It is homeomorphic to a familiar space, what is it? [Hint: Set $g(x\times y) = x+y^2$]
    \item Repeat $(a)$ for the equivalence relation $x_0\times y_0 \sim x_1\times y_1 \quad \text{if } x_0^2+y_0^2 = x_1^2 + y_1^2$.
  \end{enumerate}
  \item Let $p: X\rightarrow Y$ be an open map. Show that if $A$ is open in $X$, then the map $q: A\rightarrow p(A)$ obtained by restricting $p$ is an open map.\\
  Suppose $U$ is open in $X$, then if $A$ is open in $X$, $U\cap A$ is open. Furthermore, $p(U)$ is open, $p(A)$ is open, so $p(U\cap A)$ is open, or equivalently $q(U\cap A)$ is open. $U\cap A$ is a basis for $A$, so $q$ is an open map.
  \item Recall that $\R_K$ denotes the real line in the \hyperref[dfn:KTopology]{$K$-topology}. Let $Y$ be the quotient space obtained from $\R_K$ by collapsing the set $K$ to a point; let $p: \R_K\rightarrow Y$ be the quotient map.
  \begin{enumerate}
    \item Show that $Y$ satisfies the $T_1$ axiom but is not Hausdorff. \\
    The $K$-Topology is Hausdorff, since it is finer than the standard topology on $\R$, which is Hausdorff. To show that $Y$ satisfies the $T_1$ axiom, we show that each element of the partition $Y$ is closed. Each element besides the point $p(K)$ is clearly closed, it is a one point set in $\R_K$. In the $K$-topology, $K$ is closed, it contains all of it's limit points ($0$ is not a limit point, $(-\epsilon, \epsilon)-K$ is a neighborhood of $0$ that does not intersect $K$). Thus, $Y$ satisfies the $T_1$ axiom.\\
    Suppose $Y$ were a Hausdorff space, then every pair of distinct points $y_1,y_2\in Y$ have disjoint neighborhoods $V_1, V_2$. Let $y_1=p(K)$ and $y_2=p(0)$. $V_1$ and $V_2$ are disjoint open sets, and $p$ is a quotient map, so $U_1=p^{-1}(V_1)$ and $U_2=p^{-1}(V_2)$ are disjoint open sets of $R_K$. $U_2$ is a neighborhood of $0$, it has an open subset of the form $U_2'=(-\epsilon,\epsilon)-K$. $U_1$ is a neighborhood of $K$, and there exists some point $1/n$ in $K$ such that $1/n<\epsilon$. Any neighborhood of $K$ must have as a subset a neighborhood of this $1/n$, but this neighborhood intersects $U_2'$, and so intersects $U_2$, a contradiction, since these sets should be disjoint. Therefore, $Y$ is not Hausdorff.
    \item Show that $p\times p : \R_K\times \R_K \rightarrow Y\times Y$ is not a quotient map. [Hint: The diagonal is not closed in $Y\times Y$, but its inverse image is closed in $\R_k\times\R_K$.] \\
    Consider the set $D=\{y\times y\;|\;y\in Y\}$.  By the previous part, every neighborhood of $p(0)$ intersects some neighborhood of $p(K)$. Therefore, the point $p(K)\times p(0)$ will have neighborhoods $U\times V$, where $U$ is a neighborhood of $f(K)$, and $V$ is a neighborhood of $f(0)$ and so necessarily intersects the $U$ at some point $y$. Therefore, $U\times V$ contains a point $y\times y\in D$, so $f(K)\times f(0)$ is a limit point of $D$ not contained in $D$, so $D$ is not closed.\\
    Now consider $p^{-1}(D)$. It is Hausdorff, therefore any $x,y$ where $x\neq y$ have disjoint neighborhoods $U,V$, so $x\times Y$ has a neighborhood $U\times V$ which does not intersect $p^{-1}(D)$, so there are no limit points of $p^{-1}(D)$ which are not in $p^{-1}(D)$, so it is closed. Since $D$ is not closed in $Y\times Y$ but its inverse image is closed in $\R_k\times \R_K$, $p\times p$ is not a quotient map.
  \end{enumerate}
\end{enumerate}
\end{document}
\textbf{Chapter 3.23} \label{sec:chapter3.23}
\documentclass[12pt,letterpaper]{article}
\usepackage[pdftex]{graphicx}
\usepackage{alltt}
\usepackage[margin=1in]{geometry}
\usepackage{amsmath, amsthm, amssymb}
\usepackage{verbatim}
\usepackage{ragged2e}
\usepackage{enumitem}
\usepackage{xfrac}
\setlist{parsep=0pt,listparindent=\parindent}
\setlength{\RaggedRightParindent}{\parindent}
\newcommand{\degree}{\ensuremath{^\circ}}
\newcommand{\n}{\break}
\let\oldemptyset\emptyset
\let\emptyset\varnothing
\newcommand{\Wlog}{without loss of generality}
\newcommand{\WLOG}{Without loss of generality}
\usepackage{accents}
\let\thinbar\bar
\newcommand\thickbar[1]{\accentset{\rule{.4em}{.8pt}}{#1}}
\let\bar\thickbar
\usepackage{standalone}
\usepackage{hyperref}
\newcommand{\R}{\ensuremath{\mathbb{R}}}
\usepackage{mathtools}
\DeclarePairedDelimiter{\ceil}{\lceil}{\rceil}
\DeclarePairedDelimiter{\floor}{\lfloor}{\rfloor}
\DeclarePairedDelimiter\abs{\lvert}{\rvert}
\DeclarePairedDelimiter\norm{\lVert}{\rVert}
%%%%%%%%%%%%%%%%%%%%%%%%%%%%%%%%%%%%%%%%%%%%%%%%%%%%%
%TOPOLOGY DOCUMENTS ONLY%
\newcommand{\T}{\ensuremath{\mathcal{T}}}
%%%%%%%%%%%%%%%%%%%%%%%%%%%%%%%%%%%%%%%%%%%%%%%%%%%%%

\begin{document}
\RaggedRight
\begin{enumerate}
  \item Let $\T$ and $\T'$ be two topologies on $X$. If $\T'\supset\T$, what does connectedness of $X$ in one topology imply about connectedness in the other?\\
  If $\T$ is connected, then the courser $\T'$ is clearly connected. The reverse relationship does not imply anything about connectivity, because a finer set can always be made by adjoining a separation of $X$ to $\T$.
  \item Let $\{A_n\}$ be a sequence of connected subspaces of $X$, such that $A_n\cap A_n+1 \neq \emptyset$ for all $n$. Show that $\bigcup A_n$ is connected. \\
  $A_1$ and $A_2$ are connected by \hyperref[thm:unionConnected]{Theorem 23.3}, since their intersection is not null. Likewise, the union of the sets $(A_!\cup A_2)$ and $A_3$ is connected, since $A_2\cap A_3$ is connected. Proceeding by induction, $\bigcup A_n$ is connected.
  \item Let $\{A_\alpha\}$ be a collection of connected subspaces of $X$; let $A$ be a connected subspace of $X$. Show that if $A\cap A_\alpha\neq\emptyset$ for all $\alpha$, then $A\cup(\bigcup(A_\alpha))$ is connected.\\
  For each $\alpha$, $A$ intersects $A_\alpha$, so $A\cup A_\alpha$ is connected. Similarly, for $\beta\neq\alpha$, $A\cup A_\alpha \cup A_\beta$ is connected, and so by induction we get that $A\cup(\bigcup A_\alpha)$ is connected.
  \item Show that if $X$ is an infinite set, it is connected in the \hyperref[dfn:finiteComplementTopology]{finite complement topology}. \\
  Suppose it is not connected, there exists a subset $U$ of $X$ which is both open and closed. However, for $X-U$ to be finite, $U$ must be an infinite set. Therefore, $X-U$ is not open, since $X-U$ is finite, so $U$ is not closed, a contradiction. $X$ is connected.
  \item A space is totally disconnected if its only connected subspaces are one-point sets. Show that if $X$ has the discrete topology, then $X$ is totally disconnected. Does the converse hold? \\
  Let $A$ be a connected subspace of $X$, suppose $A$ contains more than one distinct point, let $x$ be an element $A$. $\{x\}$ and $X-\{x\}$ are open sets of $X$, and when intersected with $A$ remain open sets. They are nonempty, and so form a separation of $X$. A one point subset is obviously connected, and it is the only type of connected subspace of $X$. FIX!%The converse does hold, if there is no subset with more than one point in it that is connected, then every one point set must be open: Suppose there is a set $A\subset X$ with more than one point. By hypothesis, it is disconnected, it has a separation $C$ and $D$, so $C$ and $D$ are open. If $C$ is not a one point set, repeat. Eventually, this procedure yields $C$ and $D$ that are one-point sets, and they are open. Hence, all one point sets are open in a totally disconnected space, which means that the topology is the discrete topology. Messy.
  \item Let $A\subset X$. Show that if $C$ is a connected subspace of $X$ that intersects both $A$ and $X-A$, then $C$ intersects $\text{Bd } A$. [\hyperref[dfn:boundary]{Boundary}].\\
  Consider the subsets of $C$, $U=A\cap C$ and $V=(X-A)\cap C$. $U$ and $V$ are cleary a pair of disjoint nonempty sets whose union is $Y$. If $x$ is a limit point of both $A$ and $X-A$, then $x$ will be in the closure of each of these sets, and thus in the intersection of their closure, and so in the boundary of $A$. $C$ must contain $x$; it must be in one of $U$ or $V$, otherwise neither will contain a limit point of the other which would imply that $C$ is not connected.
  \item Is the space \hyperref[dfn:lowerLimitTopology]{$\R_\ell$} connected? \\
  It is not. The set $(-\infty,0)$ is open in $\R_\ell$, it is the union of basis elements $[-n,-n+1)$ for $n\in\mathbb{Z}_+$. $[0,\infty)$ is also open, constructed similarly. Together these form a separation of $\R_\ell$.
  \item Determine whether or not $\R^\omega$ is connected in the uniform topology. \\
  %If $\R^\omega$ is not closed, then it contains a set that is open and closed. If a typical open set, $B=B_{\bar{p}}(x,\epsilon)$ is also closed, then $A = \R^\omega - B$ is open. $A$ contains some point $y$ such that $\bar{p}(x,y)=\epsilon$. $y$ is a limit point of $B$, every neighborhood of $y$ intersects $B$, therefore $A$ cannot be open and disjoint from $B$, so $B$ is not closed, so $\R^\omega$ is connected in the uniform topology. WRONG
  \item Let $A$ be a proper subset of $X$, and let $B$ be a proper subset of $Y$. If $X$ and $Y$ are connected, show that $(X\times Y) - (A\times B)$ is connected.
  \item Let $\{X_\alpha\}_{\alpha\in J}$ be an indexed family of connected spaces; let $X$ be the product space $X=\prod_{\alpha\in J}X_\alpha$. Let $a=(a_\alpha)$ be a fixed point of $X$.
  \begin{enumerate}
    \item Given any finite subset $K$ of $J$, let $X_K$ denote the subspace of $X$ consisting of all points $x=(x_\alpha)$ such that $x_\alpha=a_\alpha$ for $a\not\in K$. Show that $X_K$ is connected.
    \item Show that the union $Y$ of spaces $X_K$ is connected.
    \item Show that $X$ equals the closure of $Y$; conclude that $X$ is connected.
  \end{enumerate}
  \item Let $p:X\rightarrow Y$ be a quotient map. Show that if each set $p^{-1}(\{y\})$ is connected, and $Y$ is connected, then $X$ is connected.
  \item Let $Y\subset X$; let $X$ and $Y$ be connected. Show that if $A$ and $B$ form a separation of $X-Y$, then $Y\cup A$ and $Y\cup B$ are connected.
\end{enumerate}
\end{document}
\end{document}